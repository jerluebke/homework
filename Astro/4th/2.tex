\documentclass[11pt,a4paper]{scrartcl}
\usepackage[top=2cm,bottom=5.5cm,left=2cm,right=2cm]{geometry}
\usepackage{fontspec}
\usepackage{polyglossia}
    \setdefaultlanguage{english}
\usepackage{fancyhdr}
\usepackage{csquotes}
\usepackage{enumitem}
% \usepackage{mathabx}
\usepackage{mathtools}
\usepackage{amssymb}
\usepackage{amsfonts}
\usepackage{siunitx}
    \sisetup{range-units=brackets}
    \DeclareSIUnit{\year}{a}
    \DeclareSIUnit{\jansky}{Jy}
    \DeclareSIUnit{\parsec}{pc}
    \DeclareSIUnit{\lightyear}{ly}
    \DeclareSIUnit{\Rsol}{\ensuremath{R_{\astrosun}}}
    \DeclareSIUnit{\Lsol}{\ensuremath{L_{\astrosun}}}
    \DeclareSIUnit{\Msol}{\ensuremath{M_{\astrosun}}}
\usepackage{physics}
\usepackage{wasysym}
\usepackage[version=4]{mhchem}
\usepackage{booktabs}
\usepackage[%
    labelformat=simple,
    labelsep=none,
    textformat=empty,
    font={small,sc}
]{caption}
\usepackage{graphicx}
    \graphicspath{img}
\usepackage{tikz}
    \usetikzlibrary{calc,external}
    \tikzexternalize[prefix=extern/]
    \tikzexternaldisable
\usepackage{pgfplots}
    \pgfplotsset{%
        compat=1.16,
        table/search path={data},
        label style={font=\tiny},
        tick label style={font=\tiny}
    }
\usepackage{todonotes}
\usepackage[%
    % colorlinks=true, linkcolor=blue,
    hidelinks
]{hyperref}


\newcommand{\tablehead}[1]{\multicolumn{1}{c}{#1}}
\newcommand*{\figref}[1]{(see fig.~\ref{#1})}
\newcommand{\eg}{e.\,g.}

\newcommand{\course}{\textbf{Introduction to Astrophysics}}
\newcommand{\hwnumber}{4}
\newcommand{\nameA}{Jeremiah Lübke}
\newcommand{\nameB}{Andreas Menzel}
\newcommand{\matnumA}{108015230366}
\newcommand{\matnumB}{108015226385}
\newcommand{\groupnum}{Exercise Group 3}


\pagestyle{fancyplain}

\headheight 7\baselineskip
\lhead{%
    \course \\
    \vspace*{3\baselineskip}
    \nameA, \matnumA \\
    \nameB, \matnumB
}

\chead{%
    \textbf{\Large Homework \hwnumber} \\
    \vspace*{2\baselineskip}
}

\rhead{%
    \today \\
    \vspace*{4\baselineskip}
    \groupnum
}

\cfoot{\small\thepage}
\headsep 1.5em


\newcommand{\Rsol}{\ensuremath{R_{\astrosun}}}
\newcommand{\Lsol}{\ensuremath{L_{\astrosun}}}
\newcommand{\Msol}{\ensuremath{M_{\astrosun}}}
\newcommand{\Rearth}{\ensuremath{R_{\earth}}}
\newcommand{\hmass}{\SI{1.008}{\atomicmassunit}}


\begin{document}

\section*{Task 2}

\begin{enumerate}[label=\textbf{\large(\alph*)}, itemsep=2\baselineskip]

\item
    The total number of fusion processes:
    \begin{equation*}
        N\sim\frac{0.1\Msol}{4\times\hmass}\sim\num{2.97e55}
    \end{equation*}
    And by comparison with the nuclear time scale
    ($\tau_{\mathrm{nuc}}\sim\SI{3e17}{\second}$) the number of processes per
    second - which is equivalent to the neutrino rate:
    \begin{equation*}
        n\equiv\phi\sim\frac{N}{\tau_{\mathrm{nuc}}}\sim\SI{9.90e37}{\per\second}
    \end{equation*}

\item
    The solar neutrino rate on earth can be estimated via:
    \begin{equation*}
        \phi_{\mathrm{Earth}}=\frac{1}{4\pi r^2}\frac{2\Lsol}{\SI{28}{\mega\electronvolt}}
        \approx\SI{6.07e14}{\per\metre\squared\per\second}
    \end{equation*}
    where $r=\SI{1}{\astronomicalunit}=\SI{149.598e9}{\metre}$:

\item
    Now estimate the distance as $r\sim\SI{163}{\kilo\lightyear}$, which
    yields over a time of 12 seconds:
    \begin{equation*}
        \phi_{\mathrm{Earth},\,\SI{12}{\second}}\sim\SI{6.85e-5}{\per\metre\squared}
    \end{equation*}
    which is practically zero.

    For reference:
    \begin{equation*}
        \frac{\phi_{\astrosun}}{\phi_{\mathrm{Supernova}}}\sim\frac{\num{1e38}}{\num{1e58}}=\num{1e-20}
    \end{equation*}

\end{enumerate}


\end{document}
