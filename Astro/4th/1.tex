\documentclass[11pt,a4paper]{scrartcl}
\usepackage[top=2cm,bottom=5.5cm,left=2cm,right=2cm]{geometry}
\usepackage{fontspec}
\usepackage{polyglossia}
    \setdefaultlanguage{english}
\usepackage{fancyhdr}
\usepackage{csquotes}
\usepackage{enumitem}
\usepackage{mathtools}
\usepackage{amssymb}
\usepackage{amsfonts}
\usepackage{siunitx}
    \sisetup{range-units=brackets}
    \DeclareSIUnit{\year}{a}
    \DeclareSIUnit{\jansky}{Jy}
    \DeclareSIUnit{\parsec}{pc}
    \DeclareSIUnit{\lightyear}{ly}
    \DeclareSIUnit{\Rsol}{\ensuremath{R_{\astrosun}}}
    \DeclareSIUnit{\Lsol}{\ensuremath{L_{\astrosun}}}
    \DeclareSIUnit{\Msol}{\ensuremath{M_{\astrosun}}}
\usepackage{physics}
\usepackage{wasysym}
\usepackage[version=4]{mhchem}
\usepackage{booktabs}
\usepackage[%
    labelformat=simple,
    labelsep=none,
    textformat=empty,
    font={small,sc}
]{caption}
\usepackage{graphicx}
    \graphicspath{img}
\usepackage{tikz}
    \usetikzlibrary{calc,external}
    \tikzexternalize[prefix=extern/]
    \tikzexternaldisable
\usepackage{pgfplots}
    \pgfplotsset{%
        compat=1.16,
        table/search path={data},
        label style={font=\tiny},
        tick label style={font=\tiny}
    }
\usepackage{todonotes}
\usepackage[%
    % colorlinks=true, linkcolor=blue,
    hidelinks
]{hyperref}


\newcommand{\tablehead}[1]{\multicolumn{1}{c}{#1}}
\newcommand*{\figref}[1]{(see fig.~\ref{#1})}
\newcommand{\eg}{e.\,g.}

\newcommand{\course}{\textbf{Introduction to Astrophysics}}
\newcommand{\hwnumber}{4}
\newcommand{\nameA}{Jeremiah Lübke}
\newcommand{\nameB}{Andreas Menzel}
\newcommand{\matnumA}{108015230366}
\newcommand{\matnumB}{108015226385}
\newcommand{\groupnum}{Exercise Group 3}


\pagestyle{fancyplain}

\headheight 7\baselineskip
\lhead{%
    \course \\
    \vspace*{3\baselineskip}
    \nameA, \matnumA \\
    \nameB, \matnumB
}

\chead{%
    \textbf{\Large Homework \hwnumber} \\
    \vspace*{2\baselineskip}
}

\rhead{%
    \today \\
    \vspace*{4\baselineskip}
    \groupnum
}

\cfoot{\small\thepage}
\headsep 1.5em


\newcommand{\Rsol}{\ensuremath{R_{\astrosun}}}
\newcommand{\Lsol}{\ensuremath{L_{\astrosun}}}
\newcommand{\Msol}{\ensuremath{M_{\astrosun}}}
\newcommand{\hmass}{\SI{1.008}{\atomicmassunit}}


\begin{document}

\section*{Task 1}

\begin{enumerate}[label=\textbf{\large(\alph*)}, itemsep=2\baselineskip]

\item
    Computing the mass defect with $m(\ce{^{1}H})=\hmass$
    and $m(\ce{^{4}He})=\SI{4.003}{\atomicmassunit}$, where
    $\si{\atomicmassunit}=\SI{931.494}{\mega\electronvolt}$:
    \begin{equation*}
        \Delta
        E=4\times\hmass-\SI{4.003}{\atomicmassunit}=\SI{27.013}{\mega\electronvolt}
    \end{equation*}
    There are $\frac{1}{\hmass}$ \ce{^{1}H} isotopes per kilogram.
    \begin{equation*}
        \implies w=\frac{E}{\si{\kilogram}}=\frac{\Delta
        E}{4\times\hmass}=\SI{6.70e14}{\joule\per\kilogram}
    \end{equation*}

\item
    The portion of the energy received by the neutrinos:
    \begin{equation*}
        \frac{2 E_{\nu_{e}}}{\Delta E}=\frac{2\times 0.42}{28}=3\%
    \end{equation*}

\item
    Estimate the life time of the sun using the nuclear time scale:
    \begin{equation*}
        \tau_{\mathrm{nuc}}=\frac{(1-0.03)\times0.1\,w\,\Msol}{\Lsol}\approx\SI{3.31e17}{\second}\sim\SI{1.05e10}{\year}
    \end{equation*}

\item
    Estimate the life time of the sun using the Kelvin-Helmholtz time scale:
    \begin{equation*}
        \tau_{\mathrm{KH}}=\frac{E_{\mathrm{Grav}}}{\Lsol}\approx\frac{G\Msol^2}{\Rsol\Lsol}\approx\SI{9.91e14}{\second}\sim\pi\times 10^7\si{\year}
    \end{equation*}
\end{enumerate}


\end{document}
