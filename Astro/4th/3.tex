\documentclass[11pt,a4paper]{scrartcl}
\usepackage[top=2cm,bottom=5.5cm,left=2cm,right=2cm]{geometry}
\usepackage{fontspec}
\usepackage{polyglossia}
    \setdefaultlanguage{english}
\usepackage{fancyhdr}
\usepackage{csquotes}
\usepackage{enumitem}
% \usepackage{mathabx}
\usepackage{mathtools}
\usepackage{amssymb}
\usepackage{amsfonts}
\usepackage{siunitx}
    \sisetup{range-units=brackets}
    \DeclareSIUnit{\year}{a}
    \DeclareSIUnit{\jansky}{Jy}
    \DeclareSIUnit{\parsec}{pc}
    \DeclareSIUnit{\lightyear}{ly}
    \DeclareSIUnit{\Rsol}{\ensuremath{R_{\astrosun}}}
    \DeclareSIUnit{\Lsol}{\ensuremath{L_{\astrosun}}}
    \DeclareSIUnit{\Msol}{\ensuremath{M_{\astrosun}}}
\usepackage{physics}
\usepackage{wasysym}
\usepackage[version=4]{mhchem}
\usepackage{booktabs}
\usepackage[%
    labelformat=simple,
    labelsep=none,
    textformat=empty,
    font={small,sc}
]{caption}
\usepackage{graphicx}
    \graphicspath{img}
\usepackage{tikz}
    \usetikzlibrary{calc,external}
    \tikzexternalize[prefix=extern/]
    \tikzexternaldisable
\usepackage{pgfplots}
    \pgfplotsset{%
        compat=1.16,
        table/search path={data},
        label style={font=\tiny},
        tick label style={font=\tiny}
    }
\usepackage{todonotes}
\usepackage[%
    % colorlinks=true, linkcolor=blue,
    hidelinks
]{hyperref}


\newcommand{\tablehead}[1]{\multicolumn{1}{c}{#1}}
\newcommand*{\figref}[1]{(see fig.~\ref{#1})}
\newcommand{\eg}{e.\,g.}

\newcommand{\course}{\textbf{Introduction to Astrophysics}}
\newcommand{\hwnumber}{4}
\newcommand{\nameA}{Jeremiah Lübke}
\newcommand{\nameB}{Andreas Menzel}
\newcommand{\matnumA}{108015230366}
\newcommand{\matnumB}{108015226385}
\newcommand{\groupnum}{Exercise Group 3}


\pagestyle{fancyplain}

\headheight 7\baselineskip
\lhead{%
    \course \\
    \vspace*{3\baselineskip}
    \nameA, \matnumA \\
    \nameB, \matnumB
}

\chead{%
    \textbf{\Large Homework \hwnumber} \\
    \vspace*{2\baselineskip}
}

\rhead{%
    \today \\
    \vspace*{4\baselineskip}
    \groupnum
}

\cfoot{\small\thepage}
\headsep 1.5em


\newcommand{\Rsol}{\ensuremath{R_{\astrosun}}}
\newcommand{\Lsol}{\ensuremath{L_{\astrosun}}}
\newcommand{\Msol}{\ensuremath{M_{\astrosun}}}
\newcommand{\Vsol}{\ensuremath{V_{\astrosun}}}

\newcommand{\drms}{\ensuremath{d_{\mathrm{rms}}}}
\newcommand{\xirms}{\ensuremath{x_{i,\,\mathrm{rms}}}}
\newcommand{\ts}{\ensuremath{\tau_{\mathrm{s}}}}
\newcommand{\td}{\ensuremath{\tau_{\mathrm{d}}}}
\newcommand{\st}{\ensuremath{\sigma_{\mathrm{T}}}}
\newcommand{\rmass}{\ensuremath{\rho_{\mathrm{mass}}}}
\newcommand{\rE}{\ensuremath{\rho_{\mathrm{E}}}}
\newcommand{\mprot}{\ensuremath{m_{{p}}}}
\newcommand{\melec}{\ensuremath{m_{{e}}}}


\begin{document}

\section*{Task 3}

\begin{enumerate}[label=\textbf{\large(\alph*)}, itemsep=2\baselineskip]

\item
    The deviation from the origin in 3 dimensions:
    \begin{equation*}
        \drms=\sqrt{\sum_{i=1}^{3}\xirms^2}=\sqrt{3N}\,l
    \end{equation*}
    The time scale of a single scattering process:
    \begin{equation*}
        \ts\sim\frac{l}{c}
    \end{equation*}
    And the total time scale of the diffusion to reach a given deviation:
    \begin{gather*}
        \td=N\,\ts \\
        \implies\td\sim\frac{3\,\drms^2\,\ts}{l^2}
    \end{gather*}

\item
    A photon reaches the Sun's surface when $\drms=\Rsol$.

    The photons scatter with electrons, while the protons provide the main
    component of density ($\mprot\gg\melec$).
    The mean free path $l=\frac{1}{\st\,n}$ with the Thomson cross section
    \st~and the particle density $n=\frac{\rmass}{\mprot}$,
    $\rmass=\SI{1408}{\kilogram\per\metre\cubed}$ is known.

    \begin{equation*}
        \implies\td\sim\frac{3\,\Rsol^2\,\st\,\rmass}{c\,\mprot}\approx\SI{271}{\giga\second}\sim\SI{8600}{\year}
    \end{equation*}

    The deviation from $\tau_{\mathrm{d,\,Lit}}=\SI{20000}{\year}$ might be
    explained by the rather vague assumptions on which the above calculation is
    based.

\item
    Estimate the Sun's Luminosity via:
    \begin{equation*}
        \Lsol\sim\frac{\rE\Vsol}{\td}=\frac{\frac{4\pi}{3}\Rsol^3\,a\,T^4}{\td}\sim\SI{1.06e27}{\watt}
    \end{equation*}
    where $\td=\SI{20000}{\year}\sim\SI{630}{\giga\second}$.

\end{enumerate}

\end{document}
