\documentclass[11pt,a4paper]{scrartcl}
\usepackage[top=2cm,bottom=5.5cm,left=2cm,right=2cm]{geometry}
\usepackage{fontspec}
\usepackage{polyglossia}
    \setdefaultlanguage{english}
\usepackage{fancyhdr}
\usepackage{csquotes}
\usepackage{enumitem}
\usepackage{mathtools}
\usepackage{amssymb}
\usepackage{amsfonts}
\usepackage{siunitx}
    \sisetup{range-units=brackets}
    \DeclareSIUnit{\year}{a}
    \DeclareSIUnit{\jansky}{Jy}
    \DeclareSIUnit{\parsec}{pc}
    \DeclareSIUnit{\lightyear}{ly}
    \DeclareSIUnit{\Rsol}{\ensuremath{R_{\astrosun}}}
    \DeclareSIUnit{\Lsol}{\ensuremath{L_{\astrosun}}}
    \DeclareSIUnit{\Msol}{\ensuremath{M_{\astrosun}}}
\usepackage{physics}
\usepackage{wasysym}
\usepackage[version=4]{mhchem}
\usepackage{booktabs}
\usepackage[%
    labelformat=simple,
    labelsep=none,
    textformat=empty,
    font={small,sc}
]{caption}
\usepackage{graphicx}
    \graphicspath{img}
\usepackage{tikz}
    \usetikzlibrary{calc,external}
    \tikzexternalize[prefix=extern/]
    \tikzexternaldisable
\usepackage{pgfplots}
    \pgfplotsset{%
        compat=1.16,
        table/search path={data},
        label style={font=\tiny},
        tick label style={font=\tiny}
    }
\usepackage{todonotes}
\usepackage[%
    % colorlinks=true, linkcolor=blue,
    hidelinks
]{hyperref}


\newcommand{\tablehead}[1]{\multicolumn{1}{c}{#1}}
\newcommand*{\figref}[1]{(see fig.~\ref{#1})}
\newcommand{\eg}{e.\,g.}
\newcommand{\ie}{i.\,e.}

\newcommand{\course}{\textbf{Introduction to Astrophysics}}
\newcommand{\hwnumber}{5}
\newcommand{\nameA}{Jeremiah Lübke}
\newcommand{\nameB}{Andreas Menzel}
\newcommand{\matnumA}{108015230366}
\newcommand{\matnumB}{108015226385}
\newcommand{\groupnum}{Exercise Group 3}


\pagestyle{fancyplain}

\headheight 7\baselineskip
\lhead{%
    \course \\
    \vspace*{3\baselineskip}
    \nameA, \matnumA \\
    \nameB, \matnumB
}

\chead{%
    \textbf{\Large Homework \hwnumber} \\
    \vspace*{2\baselineskip}
}

\rhead{%
    \today \\
    \vspace*{4\baselineskip}
    \groupnum
}

\cfoot{\small\thepage}
\headsep 1.5em


\begin{document}

\section*{Task 3}

\begin{enumerate}[label=\textbf{\large(\alph*)}, itemsep=\baselineskip]

\item
    \textbf{MACHOs:} MAssive Compact Halo Objects, \eg~brown dwarfs, black
    holes

    \textbf{WIMPs:} Weakly Interacting Massive Particles, hypothetical heavy
    particles only being subject to the weak interaction and gravity


\item
    \textbf{Problems with MACHOs as explanation for dark matter:}
    \begin{itemize}
        \item too few objects to explain the required mass
        \item usually located in the periphery of galaxies, but dark matter
            effects were detected within massive galaxy clusters far from
            visible matter
    \end{itemize}


\item
    \textbf{Experiments to detect WIMPs:} (\eg~\emph{LUX}, \emph{XENON})
    \begin{itemize}
        \item Idea: WIMPs - even though interacting rarely - should interact
            sometimes
        \item Use a large cylindric underground tank filled with liquid Xenon
            as a detector, with photodetectors at the top and bottom of the
            tank
        \item Xenon: noble gas (no background due to unwanted chemical
            reactions), high density (in liquid form), high atomic number
            $Z=54$ (large nucleus)
        \item (Rare) scattering of WIMPs on Xenon nuclei should cause
            scintillations
    \end{itemize}
    $\to$ the effort remains inconclusive as of today.


\item
    \textbf{Possible Evidence for Dark Matter:}
        \begin{itemize}
            \item Galaxies are observed to have higher radial velocities than the
                visible matter could gravitationaly hold together
            \item In 2006 the Chandra Telescope observed the collision of two
                galaxy clusters. The resulting distribution of gases and
                gravitational lensing effects suggests a spatial seperation of
                visible and dark matter. This is seen as direct evidence of dark
                matter by some scientists.
        \end{itemize}

        \textbf{MOND:} Modified Newtonian Dynamics, modified theories of
        gravitational interaction to explain the high radial velocities of
        galacies without the notion of dark matter.

        \textbf{MOND vs Dark Matter:}
        \begin{itemize}
            \item New technology yields more precise data of the CMB and
                therewith of the Universe's early days
            \item Accoring to the current theory, the Universe went through a
                phase of inflation during that time
            \item In order for this theory to fit the new data, the Universe
                would have needed much more matter than what is visible.
                Without this additional matter some large structures seen today
                would have had not enough time to evolve
        \end{itemize}

\newpage
\item
        \textbf{Alternatives:}
        \begin{itemize}
            \item Axions, hypothetical particles with small masses. The idea
                emerged when trying to explain broken CP symmetry in QCD
            \item Neutrinos
            \item Black Holes (\ie~MACHOs), suggested by a study based on LIGO
                data
            \item topological defects in quantum fields, \ie~unremovable
                discontinuities in the form of stable high energy
                configurations which remain from the early days of the
                Universe. Such defects could manifest themselves as Monopoles,
                Cosmic Strings or Domain Walls
            \item Certain ideas related to the quest for a theory of quantum
                gravity
        \end{itemize}

\end{enumerate}


\end{document}
