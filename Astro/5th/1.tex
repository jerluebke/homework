\documentclass[11pt,a4paper]{scrartcl}
\usepackage[top=2cm,bottom=5.5cm,left=2cm,right=2cm]{geometry}
\usepackage{fontspec}
\usepackage{polyglossia}
    \setdefaultlanguage{english}
\usepackage{fancyhdr}
\usepackage{csquotes}
\usepackage{enumitem}
\usepackage{mathtools}
\usepackage{amssymb}
\usepackage{amsfonts}
\usepackage{siunitx}
    \sisetup{range-units=brackets}
    \DeclareSIUnit{\year}{a}
    \DeclareSIUnit{\jansky}{Jy}
    \DeclareSIUnit{\parsec}{pc}
    \DeclareSIUnit{\lightyear}{ly}
    \DeclareSIUnit{\Rsol}{\ensuremath{R_{\astrosun}}}
    \DeclareSIUnit{\Lsol}{\ensuremath{L_{\astrosun}}}
    \DeclareSIUnit{\Msol}{\ensuremath{M_{\astrosun}}}
\usepackage{physics}
\usepackage{wasysym}
\usepackage[version=4]{mhchem}
\usepackage{booktabs}
\usepackage[%
    labelformat=simple,
    labelsep=none,
    textformat=empty,
    font={small,sc}
]{caption}
\usepackage{graphicx}
    \graphicspath{img}
\usepackage{tikz}
    \usetikzlibrary{calc,external}
    \tikzexternalize[prefix=extern/]
    \tikzexternaldisable
\usepackage{pgfplots}
    \pgfplotsset{%
        compat=1.16,
        table/search path={data},
        label style={font=\tiny},
        tick label style={font=\tiny}
    }
\usepackage{todonotes}
\usepackage[%
    % colorlinks=true, linkcolor=blue,
    hidelinks
]{hyperref}


\newcommand{\tablehead}[1]{\multicolumn{1}{c}{#1}}
\newcommand*{\figref}[1]{(see fig.~\ref{#1})}
\newcommand{\eg}{e.\,g.}
\newcommand{\ie}{i.\,e.}

\newcommand{\course}{\textbf{Introduction to Astrophysics}}
\newcommand{\hwnumber}{5}
\newcommand{\nameA}{Jeremiah Lübke}
\newcommand{\nameB}{Andreas Menzel}
\newcommand{\matnumA}{108015230366}
\newcommand{\matnumB}{108015226385}
\newcommand{\groupnum}{Exercise Group 3}


\pagestyle{fancyplain}

\headheight 7\baselineskip
\lhead{%
    \course \\
    \vspace*{3\baselineskip}
    \nameA, \matnumA \\
    \nameB, \matnumB
}

\chead{%
    \textbf{\Large Homework \hwnumber} \\
    \vspace*{2\baselineskip}
}

\rhead{%
    \today \\
    \vspace*{4\baselineskip}
    \groupnum
}

\cfoot{\small\thepage}
\headsep 1.5em


\newcommand{\tlabel}[1]{\ensuremath{\tau_{\mathrm{#1}}}}


\begin{document}

\section*{Task 1}

\begin{enumerate}[label=\textbf{\large(\alph*)}, itemsep=\baselineskip]

\item
    h and chi Persei being young clusters, one would expect them - according to
    the old modell - to start on the giant branch and travel left in the
    diagramm towards higher temperatures.
    The CMD of h and chi Persei only displays few stars on the giant branch
    with the majority being located on the main sequence. This suggests instead
    a rather old cluster.

    While the giant branch of M3 is more strongly populated, the bulk is
    located at the lower main sequence.
    Again according to the old modell, one would expect them to travel
    leftwards and show a more significant change in color.
    The CMD of M3 however shows movement to the right, indicating lower
    temperatures.

    According to the old modell, stars at the end of their life should evolve
    downwards along the main sequence, while decreasing in luminosity and
    temperature. Taking into account the mass-luminosity-relation, they should
    lose large amounts of their mass during this process.
    The old modell does not seem to refer to the mass at all. However nowadays
    loss of mass is mainly observed on the giant branch.


\item
    The radius of a star is put into relation with its luminosity and
    temperature by the Stefan-Boltzmann-Law:
    \begin{equation*}
        L = A\,\sigma\,T^4 = 4\pi\,r^2\,\sigma\,T^4
    \end{equation*}
    Thus the larger luminosity or temperature the larger the radius.
    In a loglog plot one can draw straight parallel lines indicating constant
    radii \figref{app:fig1}.


\item
    The life time of a star can be estimated via the Kelvin-Helmholtz time
    scale:
    \begin{equation*}
        \tlabel{HK} \propto \frac{M^2}{L} \propto M^{-1.5}
    \end{equation*}
    where the mass-luminosity-relation $L \propto M^{3.5}$ was used.

    Estimating the lower border of star masses with \SI{0.07}{\Msol} and
    the upper bound with \SI{120}{\Msol}:
    \begin{equation*}
        \implies \frac{\tlabel{light}}{\tlabel{heavy}} \propto
        \left(\frac{0.07}{120}\right)^{-1.5} = \num{70978}
    \end{equation*}

    This means, the lightest star has approximately \num{71000} times the life
    time of the heaviest.


\item
    Star clusters usually consist of only one generation of stars.
    Galaxies such as the milkyway - on the other hand - can have multiple
    generations of star formation, causing a wider scattering of the magnitude
    in the main seqence.


\item
    The Hertzsprung gap coincides with the point in stellar development, where
    core hydrogen burning finished and shell hydrogen burning has yet to start.
    Since this is a rather short time compared to their overall life time,
    stars move through this section quite quickly. Thus it is much more loosely
    populated compared to other areas of the diagram.

\end{enumerate}

\newpage

\chead{%
    \textbf{\Large Homework \hwnumber~--~Appendix} \\
    \vspace*{2\baselineskip}
}
\cfoot{}

% \section*{Anhang}

\begin{figure}[h]
    \centering
    \tikzexternalenable
\tikzsetnextfilename{exhrd}

\begin{tikzpicture}[>=stealth,pin distance=3cm,every pin edge/.style={<-}]

\begin{axis}[%
    set layers,
    width=17cm,
    height=22cm,
    title=Hertzsprung-Russel Diagram: Harvard Revised{,} Gliese,
    title style={yshift=1.5cm,font=\large},
    label style={font=\small},
    x label style={yshift=.4em},
    y label style={yshift=-.4em},
    tick label style={font=\small},
    xlabel=Colour Index $(B-V)$,
    ylabel=Absolute Magnitude,
    ymin=-18,
    ymax=20,
    xmin=-0.5,
    xmax=2.5,
    y dir=reverse,
    xtick distance=0.5,
    % xtick={3000,4000,5000,6000,7500,10000,12000,30000},
    % xticklabel style={%
    %     /pgf/number format/.cd,
    %     fixed,precision=0,
    %     1000 sep={\,},
    %     /tikz/.cd,
    % },
    % xticklabel=\pgfmathparse{exp(\tick)}\pgfmathprintnumber{\pgfmathresult},
    grid=major,
    scatter/classes={%
        d={mark=*,lightgray,draw=gray!60!black},
        o={mark=*,blue!60!white,draw=blue!80!black},
        b={mark=*,cyan!80!white,draw=cyan!60!black},
        a={mark=*,white,draw=gray},
        f={mark=*,lime!80!white,draw=lime!60!black},
        g={mark=*,yellow!80!white,draw=yellow!60!black},
        k={mark=*,orange!80!white,draw=orange!60!black},
        m={mark=*,red!60!white,draw=red!60!black}
    },
    colormap={mycm}{%
        color=(blue!60!white),
        color=(cyan),
        color=(white),
        color=(lime),
        color=(yellow),
        color=(orange),
        color=(red),
        color=(lightgray)
    },
    colorbar horizontal,
    colorbar as palette,
    colorbar style={%
        title=Spectral Class,
        title style={yshift=-.2cm,font=\small},
        at={(0.5,1.03)},
        anchor=south,
        xticklabel pos=upper,
        xticklabel interval boundaries,
        xticklabels={,O,B,A,F,G,K,M,D},
        x tick label style={below,font=\bfseries},
        xtick distance=1,
        samples=8
    }
]
    \addplot+ [%
        only marks,
        scatter,
        point meta=explicit symbolic,
    ] table [%
        x=ci,
        % y=lum,
        y=absmag,
        meta=spect,
        col sep=comma
    ] {hygdata_v3_filtered.csv};

    \addplot+ [%
        % orange!80!black,
        % orange!80!yellow,
        yellow!60!black,
        mark=*,
        only marks,
        mark size=3pt,
        mark options={very thick,fill=yellow}
    ] coordinates {%
        (0.63, 4.83)
    };

    \pgfonlayer{axis foreground}
        \draw [->,shorten >=2.5pt,semithick] (axis cs:2.0,4.0) -- (axis cs:0.63,4.83);
        \node at (axis cs:2.0,4.0) [%
                rectangle,rounded corners,
                draw,fill=white,
                align=left,font=\scriptsize]
            {\small\textbf{Sol} \\ $M=4.83$ \\ $B-V=0.63$};
    \endpgfonlayer
\end{axis}

\end{tikzpicture}

\tikzexternaldisable

    \caption{}
    \label{app:fig1}
\end{figure}

\newpage

\begin{figure}[h]
    \centering
    \tikzexternalenable
\tikzsetnextfilename{exthhrd}

\begin{tikzpicture}

\begin{axis}[%
    set layers,
    width=17cm,
    height=22cm,
    title=Theoretical HR Diagram: V/19,
    title style={font=\large},
    xlabel=$T/\si{\kelvin}$,
    ylabel=$L/\si{\solarlum}$,
    label style={font=\small},
    x label style={yshift=.2em},
    y label style={yshift=-.8em},
    ticklabel style={font=\small},
    x dir=reverse,
    xmode=log,
    ymode=log,
    xmax=10e5,
    xmin=10e1,
    ymax=10e8,
    ymin=10e-5,
    grid=major,
    view={0}{90},
    colormap={mycm}{color=(gray),color=(gray)},
    legend entries={V/19, V/19/Hyades, Aldebaran},
    legend cell align=left,
    legend pos=south west,
    legend style={rounded corners,mark size=3pt}
]
    \addplot+ [%
        blue!80!black,
        only marks,
        mark options={fill=blue!60!white,}
    ] table [%
        x=T,
        y=L,
        col sep=comma
    ] {asu-piskunov1980-nolog.csv};

    \addplot+ [%
        yellow!60!black,
        mark=*,
        only marks,
        mark options={thick,fill=yellow!80!white}
    ] table [%
        x=T,
        y=L,
        col sep=comma
    ] {asu-piskunov1980-hyades-nolog.csv};

    \addplot+ [%
        green!60!black,
        mark=*,
        only marks,
        mark size=3pt,
        mark options={thick,fill=green!80!white}
    ] coordinates {%
        (3910, 518)
    };

    \addplot [%
        contour prepared={labels=false},
        contour plot/.style={draw color=grey},
    ] table {%
        1e6     9e-4    1e-6
        5.772e5 1e-4    1e-6

        1e6     9       1e-4
        5.772e4 1e-4    1e-4

        1e6     9e4     1e-2
        5.772e3 1e-4    1e-2

        1e6     9e8     1
        5.772e2 1e-4    1

        1e6     9e12    1e2
        5.772e1 1e-4    1e2

        1e6     9e16    1e4
        5.772   1e-4    1e4

        1e6     9e20    1e6
        5.772e-1    1e-4    1e6
    };

    \pgfonlayer{axis foreground}
        \node [align=center,
            % draw,
            fill=white,
            inner sep=1pt,font=\scriptsize,rotate=-59]
            at (axis cs: 324583, 1e-1) {\SI{1e-4}{\solarradii}};
        \node [align=center, fill=white, inner sep=1pt,font=\scriptsize,rotate=-59]
            at (axis cs: 324583, 1e3) {\SI{1e-2}{\solarradii}};
        \node [align=center, fill=white, inner sep=1pt,font=\scriptsize,rotate=-59]
            at (axis cs: 324583, 1e7) {\SI{1}{\solarradii}};
        \node [align=center, fill=white, inner sep=1pt,font=\scriptsize,rotate=-59]
            at (axis cs: 57720, 1e8) {\SI{1e2}{\solarradii}};
        \node [align=center, fill=white, inner sep=1pt,font=\scriptsize,rotate=-59]
            at (axis cs: 5772, 1e8) {\SI{1e4}{\solarradii}};
        \node [align=center, fill=white, inner sep=1pt,font=\scriptsize,rotate=-59]
            at (axis cs: 577, 1e8) {\SI{1e6}{\solarradii}};
    \endpgfonlayer

\end{axis}

\end{tikzpicture}

\tikzexternaldisable

    \caption{}
    \label{app:fig2}
\end{figure}

\newpage



\end{document}
