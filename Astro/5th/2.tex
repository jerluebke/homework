\documentclass[11pt,a4paper]{scrartcl}
\usepackage[top=2cm,bottom=5.5cm,left=2cm,right=2cm]{geometry}
\usepackage{fontspec}
\usepackage{polyglossia}
    \setdefaultlanguage{english}
\usepackage{fancyhdr}
\usepackage{csquotes}
\usepackage{enumitem}
\usepackage{mathtools}
\usepackage{amssymb}
\usepackage{amsfonts}
\usepackage{siunitx}
    \sisetup{%
        % range-units=brackets,
        range-units=repeat,
        range-phrase={;\,}
    }
    \DeclareSIUnit{\year}{a}
    \DeclareSIUnit{\jansky}{Jy}
    \DeclareSIUnit{\parsec}{pc}
    \DeclareSIUnit{\lightyear}{ly}
    \DeclareSIUnit{\Rsol}{\ensuremath{R_{\astrosun}}}
    \DeclareSIUnit{\Lsol}{\ensuremath{L_{\astrosun}}}
    \DeclareSIUnit{\Msol}{\ensuremath{M_{\astrosun}}}
\usepackage{physics}
\usepackage{wasysym}
\usepackage[version=4]{mhchem}
\usepackage{booktabs}
\usepackage[%
    labelformat=simple,
    labelsep=none,
    textformat=empty,
    font={small,sc}
]{caption}
\usepackage{graphicx}
    \graphicspath{img}
\usepackage{tikz}
    \usetikzlibrary{calc,external}
    \tikzexternalize[prefix=extern/]
    \tikzexternaldisable
\usepackage{pgfplots}
    \pgfplotsset{%
        compat=1.16,
        table/search path={data},
        label style={font=\tiny},
        tick label style={font=\tiny}
    }
\usepackage{todonotes}
\usepackage[%
    % colorlinks=true, linkcolor=blue,
    hidelinks
]{hyperref}


\newcommand{\tablehead}[1]{\multicolumn{1}{c}{#1}}
\newcommand*{\figref}[1]{(see fig.~\ref{#1})}
\newcommand{\eg}{e.\,g.}

\newcommand{\course}{\textbf{Introduction to Astrophysics}}
\newcommand{\hwnumber}{5}
\newcommand{\nameA}{Jeremiah Lübke}
\newcommand{\nameB}{Andreas Menzel}
\newcommand{\matnumA}{108015230366}
\newcommand{\matnumB}{108015226385}
\newcommand{\groupnum}{Exercise Group 3}


\pagestyle{fancyplain}

\headheight 7\baselineskip
\lhead{%
    \course \\
    \vspace*{3\baselineskip}
    \nameA, \matnumA \\
    \nameB, \matnumB
}

\chead{%
    \textbf{\Large Homework \hwnumber} \\
    \vspace*{2\baselineskip}
}

\rhead{%
    \today \\
    \vspace*{4\baselineskip}
    \groupnum
}

\cfoot{\small\thepage}
\headsep 1.5em


\newcommand{\taums}{\ensuremath{\tau_{\mathrm{MS}}}}
\newcommand{\mbold}[1]{\mathrm{\textbf{#1}}}


\begin{document}

\section*{Task 2}

\begin{enumerate}[label=\textbf{\large(\alph*)}, itemsep=\baselineskip]

\item
    In order from youngest to oldest:
    \begin{description}
        \item [Pleiades]
        \item [Hyades]
        \item [M3]
    \end{description}
    The Pleiades have stars only on the main sequence (MS), relatively high up.
    This makes these the youngest of the three.

    The Hyades are mainly located on the MS, predominantly in the lower
    portion, but do have a small cluster on the Red Giant Branch (RGB), making
    them older than the Pleiades.

    Lastly the HRD of M3 shows a significant number of stars on the Asymptotic
    Giant Branch (AGB), the RGB and the Horizontal Branch (HB), indicating a
    much greater age, compared to the other two star clusters.


\item
    The time a star spends on the MS can be estimated with (using the
    mass-luminosity-relation $L \propto M^{3.5}$):
    \begin{equation*}
        \taums \propto
        10^{10}\frac{M}{\si{\Msol}}\left(\frac{L}{\si{\Lsol}}\right)^{-1}
        \propto 10^{10}\left(\frac{L}{\si{\Lsol}}\right)^{-\frac{5}{7}}
    \end{equation*}
    Therewith one can write down an approximate range of how long the stars of
    each cluster will remain on the MS based solely on $y =
    \log{\frac{L}{\si{\Lsol}}}$:
    \begin{align*}
        \mbold{Pleiades}\qquad &y\in\left[\numrange{0}{3}\right] \implies
            \taums\in\left[\SIrange{72e6}{10e10}{\year}\right] \\
        \mbold{Hyades}\qquad &y\in\left[\numrange{-1}{2}\right] \implies
            \taums\in\left[\SIrange{372e6}{50e10}{\year}\right] \\
        \mbold{M3}\qquad &y\in\left[\numrange{-1}{1}\right] \implies
            \taums\in\left[\SIrange{2e9}{50e10}{\year}\right]
    \end{align*}
    Generally speaking, for clusters with stars mainly on the MS, the top of
    the range is an upper bound of the age, whereas for clusters that more
    spread along the evolutionary track, the bottom of the range is a lower
    bound of the age.


\item
    \emph{Actual ages:}

    \begin{tabular}{l r}
        \textbf{Pleiades}   & \SI{100e6}{\year} \\
        \textbf{Hyades}     & \SI{625e6}{\year} \\
        \textbf{M3}         & \SI{11e9}{\year} \\
    \end{tabular}

    We expected this method to be quite uncertain, mainly because the ranges can be
    quite large (depending on how much the cluster is spread along the MS), and
    also because there is no information on how much time the stars spend in
    other areas of the HRD outside the MS. Yet the actual ages of the Pleiades,
    Hyades and M3 fit nicely in these ranges. This can perhaps be explained by
    the fact that stars generally spend most of their life on the MS.


\end{enumerate}


\end{document}
