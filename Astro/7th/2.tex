\documentclass[11pt,a4paper]{scrartcl}
\usepackage[top=2cm,bottom=5.5cm,left=2cm,right=2cm]{geometry}
\usepackage{fontspec}
\usepackage{polyglossia}
    \setdefaultlanguage{english}
\usepackage{lmodern}
\usepackage{fixcmex}
\usepackage{fancyhdr}
\usepackage{csquotes}
\usepackage{enumitem}
\usepackage{mathtools}
\usepackage{amssymb}
\usepackage{amsfonts}
\usepackage{siunitx}
    \sisetup{range-units=brackets}
    \DeclareSIUnit{\year}{a}
    \DeclareSIUnit{\jansky}{Jy}
    \DeclareSIUnit{\parsec}{pc}
    \DeclareSIUnit{\lightyear}{ly}
    \DeclareSIUnit{\Rsol}{\ensuremath{R_{\astrosun}}}
    \DeclareSIUnit{\Lsol}{\ensuremath{L_{\astrosun}}}
    \DeclareSIUnit{\Msol}{\ensuremath{M_{\astrosun}}}
\usepackage{physics}
\usepackage{textcomp}
\usepackage{gensymb}
\usepackage{wasysym}
\usepackage[version=4]{mhchem}
\usepackage{array}
\usepackage{booktabs}
\usepackage[%
    labelformat=simple,
    labelsep=none,
    textformat=empty,
    font={small,sc}
]{caption}
\usepackage{graphicx}
    \graphicspath{img}
\usepackage{tikz}
    \usetikzlibrary{calc,external}
    \tikzexternalize[prefix=extern/]
    \tikzexternaldisable
\usepackage{pgfplots}
    \pgfplotsset{%
        compat=1.16,
        table/search path={data},
        label style={font=\tiny},
        tick label style={font=\tiny}
    }
\usepackage{todonotes}
\usepackage[%
    % colorlinks=true, linkcolor=blue,
    hidelinks
]{hyperref}


\newcommand{\tablehead}[1]{\multicolumn{1}{c}{#1}}
\newcommand*{\figref}[1]{(see fig.~\ref{#1})}
\newcommand{\eg}{e.\,g.}
\newcommand{\ie}{i.\,e.}

\newcommand{\course}{\textbf{Introduction to Astrophysics}}
\newcommand{\hwnumber}{7}
\newcommand{\nameA}{Jeremiah Lübke}
\newcommand{\nameB}{Andreas Menzel}
\newcommand{\matnumA}{108015230366}
\newcommand{\matnumB}{108015226385}
\newcommand{\groupnum}{Exercise Group 3}


\pagestyle{fancyplain}

\headheight 7\baselineskip
\lhead{%
    \course \\
    \vspace*{3\baselineskip}
    \nameA, \matnumA \\
    \nameB, \matnumB
}

\chead{%
    \textbf{\Large Homework \hwnumber} \\
    \vspace*{2\baselineskip}
}

\rhead{%
    \today \\
    \vspace*{4\baselineskip}
    \groupnum
}

\cfoot{\small\thepage}
\headsep 1.5em


\begin{document}

\section*{Task 2}

\begin{enumerate}[label=\textbf{\large(\alph*)}, itemsep=\baselineskip]

\item
    \begin{description}
        \item[Type I] No H lines in their spectra (rather peculiar since
            Hydrogen is the most common element in the universe) \\
            \emph{Subtypes:}
            \begin{description}
                \item[Ia] Strong Si II lines; very uniform properties,
                    therewith very useful for calibration when determing
                    distances. \\
                    Actually involving fundamentally different physics compared
                    to the other types (being \emph{thermo-nuclear} SNe, while
                    the others are \emph{core-collapse} SNe).
                \item[Ib] Notable He lines. \\
                    The outer hull containing the remaining hydrogen of the
                    star has been ejected prior to the explosion.
                \item[Ic] No He lines, weak Si II line. \\
                    In Addition to the hydrogen, the helium has also been
                    ejected prior to the explosion.
            \end{description}

            While Ia SN are generally found in all types of galaxies, Ib/c are
            (so far) only found in spiral galaxies in areas where star
            formation has recently taken place, indicating a connection to
            short-lived massive stars.

        \item[Type II] Strong H lines. \\
            Also core collapse SNe; there have been observed cases, where
            initially Type II SNe evolved into Type Ib. \\
            The two subtypes and their differences are explained below.
    \end{description}


\item
    \begin{description}
        \item[II-P] The light curve exhibits a plateau in the descending
            magnitude $\sim$30 to 80 days after its maximum. \\
            This can be explained by the very high amounts of ejected mass and
            velocities of this type. The decrease in magnitude due to cooling
            is compensated by a quick expansion of the ejected hull and the
            thereby enlarged surface area.
        \item[II-L] After reaching its maximum, the magnitude decreases
            linearly. \\
            Since the amount of ejected mass and its velocity are comparable
            low, the magnitude decreases rather uncompensated.
    \end{description}


\item
    Consider the gravitational potential:
    \begin{gather*}
        V=-\frac{GMm}{r}\qc\dd{V}=-\frac{GM\dd{m}}{r} \\
        \qq*{with}\dd{m}=4\pi\,r^2\rho(r)\dd{r} \\
        \implies V=-4\pi\,G\int\limits_{0}^{R}\dd{r}\,r\,M(r)\,\rho(r)
    \end{gather*}
    Approximating $\rho(r)$ via the mean density
    $\rho\approx\bar{\rho}=\frac{M}{\frac{4}{3}\pi{R}^{3}}$ and
    $M(r)\approx\frac{4}{3}\pi\,r^3\,\bar{\rho}$:
    \begin{equation*}
        \implies V\approx-\frac{16\pi^2G}{3}\bar{\rho}^2\int\limits_{0}^{R}\dd{r}r^4
        =-\frac{16\pi^2G}{15}\bar{\rho}^2R^5
        =-\frac{3}{5}\frac{GM^2}{R}
    \end{equation*}
    And with the Virial Theorem $2\,T=-V$:
    \begin{equation*}
        E=T+V=\frac{1}{2}V\approx-\frac{3}{10}\frac{GM^2}{R}.
    \end{equation*}

    Which yields for the given values (writing the absolute value of the
    energy):
    \begin{equation*}
        \implies E\sim\SI{5e45}{\joule}
    \end{equation*}

    For comparision, the estimated energy of SN1987A:
    \begin{equation*}
        E_{\mathrm{SN1987A}}\sim\SI{1e46}{\joule}
    \end{equation*}


\item
    The given light curve indicates a Type II SN, coming from the linear
    descent after the maximum which becomes more shallow after a kink at
    $\sim$900 days.  In contrast with a Type I curve, one would expect a steeper
    maximum and an undisturbed linear descent afterwards.

\end{enumerate}


\end{document}
