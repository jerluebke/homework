\documentclass[11pt,a4paper]{scrartcl}
\usepackage[top=2cm,bottom=5.5cm,left=2cm,right=2cm]{geometry}
\usepackage{fontspec}
\usepackage{polyglossia}
    \setdefaultlanguage{english}
\usepackage{lmodern}
\usepackage{fixcmex}
\usepackage{fancyhdr}
\usepackage{csquotes}
\usepackage{enumitem}
\usepackage{mathtools}
\usepackage{amssymb}
\usepackage{amsfonts}
\usepackage{siunitx}
    \sisetup{range-units=brackets}
    \DeclareSIUnit{\year}{a}
    \DeclareSIUnit{\jansky}{Jy}
    \DeclareSIUnit{\parsec}{pc}
    \DeclareSIUnit{\lightyear}{ly}
    \DeclareSIUnit{\Rsol}{\ensuremath{R_{\astrosun}}}
    \DeclareSIUnit{\Lsol}{\ensuremath{L_{\astrosun}}}
    \DeclareSIUnit{\Msol}{\ensuremath{M_{\astrosun}}}
\usepackage{physics}
\usepackage{textcomp}
\usepackage{gensymb}
\usepackage{wasysym}
\usepackage[version=4]{mhchem}
\usepackage{array}
\usepackage{booktabs}
\usepackage[%
    labelformat=simple,
    labelsep=none,
    textformat=empty,
    font={small,sc}
]{caption}
\usepackage{graphicx}
    \graphicspath{img}
\usepackage{tikz}
    \usetikzlibrary{calc,external}
    \tikzexternalize[prefix=extern/]
    \tikzexternaldisable
\usepackage{pgfplots}
    \pgfplotsset{%
        compat=1.16,
        table/search path={data},
        label style={font=\tiny},
        tick label style={font=\tiny}
    }
\usepackage{todonotes}
\usepackage[%
    % colorlinks=true, linkcolor=blue,
    hidelinks
]{hyperref}


\newcommand{\tablehead}[1]{\multicolumn{1}{c}{#1}}
\newcommand*{\figref}[1]{(see fig.~\ref{#1})}
\newcommand{\eg}{e.\,g.}
\newcommand{\ie}{i.\,e.}

\newcommand{\course}{\textbf{Introduction to Astrophysics}}
\newcommand{\hwnumber}{7}
\newcommand{\nameA}{Jeremiah Lübke}
\newcommand{\nameB}{Andreas Menzel}
\newcommand{\matnumA}{108015230366}
\newcommand{\matnumB}{108015226385}
\newcommand{\groupnum}{Exercise Group 3}


\pagestyle{fancyplain}

\headheight 7\baselineskip
\lhead{%
    \course \\
    \vspace*{3\baselineskip}
    \nameA, \matnumA \\
    \nameB, \matnumB
}

\chead{%
    \textbf{\Large Homework \hwnumber} \\
    \vspace*{2\baselineskip}
}

\rhead{%
    \today \\
    \vspace*{4\baselineskip}
    \groupnum
}

\cfoot{\small\thepage}
\headsep 1.5em


\begin{document}

\section*{Task 3}

\begin{enumerate}[label=\textbf{\large(\alph*)}, itemsep=\baselineskip]

\item
    Consider:
    \begin{gather*}
        \frac{M_1}{M_2}=\frac{v_2}{v_1}\qc
        M_1+M_2=\frac{P}{2\pi{G}}\left(\frac{v_1+v_2}{\sin{i}}\right)^3 \\
        \implies
        M_1+M_2=\frac{P}{2\pi{G}}\left(\frac{v_1}{M_2}\frac{M_1+M_2}{\sin{i}}\right)^3
        \\
        \iff\frac{P\,v_1^3}{2\pi{G}}=\frac{M_2^3}{(M_1+M_2)^2}\sin^3{i}
    \end{gather*}
    For the given values, one has:
    \begin{equation*}
        \frac{P\,v_1^3}{2\pi{G}}\approx\SI{4.42e30}{\kilogram}=\SI{2.22}{\Msol}
    \end{equation*}

    Since this number is quite large, it follows: $M_2^{3/2}\gg M_1+M_2$, which
    indicates that the mass of the compact object is large compared to the mass
    of the star.

\item
    One finds straightaway from above calculations:
    \begin{equation*}
        M_2=\frac{P\,v_1\,(v_1+v_2)^2}{2\pi{G}\sin^3{i}}
    \end{equation*}

    Therewith:
    $\eval{M_2}_{i=\ang{90}}\approx\SI{5.41e30}{\kilogram}=\SI{2.72}{\Msol}$,
    which would fit for a small stellar black hole.

\item
    Now: $\eval{M_2}_{i=\ang{85}}\approx\SI{5.47e30}{\kilogram}=\SI{2.75}{\Msol}$,
    which still corresponds to a small stellar black hole.

\item
    Problems with this approach of determing the mass is the strong dependence
    on the inclination, which can -- at best -- be roughly estimated, and on the
    radial velocity of the compact object which cannot be measured directly if
    it is not visible (\eg~black holes).

\end{enumerate}


\end{document}
