\documentclass[11pt,a4paper]{scrartcl}
\usepackage[top=2cm,bottom=5.5cm,left=2cm,right=2cm]{geometry}
\usepackage{fontspec}
\usepackage{polyglossia}
    \setdefaultlanguage{english}
\usepackage{lmodern}
\usepackage{fixcmex}
\usepackage{fancyhdr}
\usepackage{csquotes}
\usepackage{enumitem}
\usepackage{mathtools}
\usepackage{amssymb}
\usepackage{amsfonts}
\usepackage{siunitx}
    \sisetup{range-units=brackets}
    \DeclareSIUnit{\year}{a}
    \DeclareSIUnit{\jansky}{Jy}
    \DeclareSIUnit{\parsec}{pc}
    \DeclareSIUnit{\lightyear}{ly}
    \DeclareSIUnit{\Rsol}{\ensuremath{R_{\astrosun}}}
    \DeclareSIUnit{\Lsol}{\ensuremath{L_{\astrosun}}}
    \DeclareSIUnit{\Msol}{\ensuremath{M_{\astrosun}}}
\usepackage{physics}
\usepackage{textcomp}
\usepackage{gensymb}
\usepackage{wasysym}
\usepackage[version=4]{mhchem}
\usepackage{array}
\usepackage{booktabs}
\usepackage[%
    labelformat=simple,
    labelsep=none,
    textformat=empty,
    font={small,sc}
]{caption}
\usepackage{graphicx}
    \graphicspath{img}
\usepackage{tikz}
    \usetikzlibrary{calc,external}
    \tikzexternalize[prefix=extern/]
    \tikzexternaldisable
\usepackage{pgfplots}
    \pgfplotsset{%
        compat=1.16,
        table/search path={data},
        label style={font=\tiny},
        tick label style={font=\tiny}
    }
\usepackage{todonotes}
\usepackage[%
    % colorlinks=true, linkcolor=blue,
    hidelinks
]{hyperref}


\newcommand{\tablehead}[1]{\multicolumn{1}{c}{#1}}
\newcommand*{\figref}[1]{(see fig.~\ref{#1})}
\newcommand{\eg}{e.\,g.}
\newcommand{\ie}{i.\,e.}

\newcommand{\course}{\textbf{Introduction to Astrophysics}}
\newcommand{\hwnumber}{10}
\newcommand{\nameA}{Jeremiah Lübke}
\newcommand{\nameB}{Andreas Menzel}
\newcommand{\matnumA}{108015230366}
\newcommand{\matnumB}{108015226385}
\newcommand{\groupnum}{Exercise Group 3}


\pagestyle{fancyplain}

\headheight 7\baselineskip
\lhead{%
    \course \\
    \vspace*{3\baselineskip}
    \nameA, \matnumA \\
    \nameB, \matnumB
}

\chead{%
    \textbf{\Large Homework \hwnumber} \\
    \vspace*{2\baselineskip}
}

\rhead{%
    \today \\
    \vspace*{4\baselineskip}
    \groupnum
}

\cfoot{\small\thepage}
\headsep 1.5em


\newcommand{\SFR}{\ensuremath{\Sigma_{\mathrm{SFR}}}}
\newcommand{\ISM}{\ensuremath{\Sigma_{\mathrm{gas}}}}
\newcommand{\ISMtotal}{\ensuremath{\Sigma_{\mathrm{total}}}}
\newcommand{\ISMHI}{\ensuremath{\Sigma_{\mathrm{H}_1}}}
\newcommand{\ISMHII}{\ensuremath{\Sigma_{\mathrm{H}_2}}}
\newcommand{\HI}{\ensuremath{\mathrm{H}_1}}
\newcommand{\HII}{\ensuremath{\mathrm{H}_2}}


\begin{document}

\section*{Task 1}

\begin{enumerate}[label=\textbf{\large(\alph*)}, itemsep=\baselineskip]

\item
    The Schmidt Law attempts to describe the relationship between large-scale
    star formation rate and ISM (more specifically: the SFR per unit area and
    the surface density of the ISM)

\item
    Measurements of H$\alpha$ luminosities suffer from errors due to extinction
    variations and extrapolated initial mass functions. Errors in measurements
    of the gas density are largely due to variations in the CO/H conversion
    factor, combined with sometimes limited CO measurements. These influences
    are difficult to reduce, since many times one has to rely on estimates
    instead of accurate data.

\item
    Figure 3 depicts density profiles of 21 galaxies, which were
    obtained by azimuthally averaging \SFR~and \ISM~at different distances from
    the galactic centre. For high densities a Schmidt Law is fulfilled, while
    below a certain threshold the profiles steeply decrease. This is to
    illustrate deviations due to averaging over the whole galaxy, as it is
    done in that paper.

\item
    The \SFR-\ISMHI~relation is similar to the \SFR-\ISMtotal~relation, which
    is not surprising as \HI~makes up roughly half of the total gas
    density. Further, one could read from the correlation that the SFR
    regulates the \HI~density through photodissociation of molecular gas by
    (young) hot stars.

\item
    The \SFR-\ISMHII~correlation is much weaker, possibly as it is based on a
    CO/\HII~conversion factor which is valid for regions with near-solar
    metalicity, but thereby underestimating \HII-mass in metal-poor regions.

\item
    In figure 6, filled circles represent global H$\alpha$ measurements of normal
    spiral galaxies, which are found in regions of low densities, and filled
    squares represent FIR measurements of the central regions of starburst
    galaxies, which fill regions of higher densities. \\
    While both types remarkably agree in their slope, they are separated by an
    empty gap. In order to show that a physical continuity exists between both
    domains, from 25 normal spiral galaxies \SFR~and \ISM~were derived by
    averaging only their central regions. When this data is plotted (as open
    circles), it fills aforementioned gap.

\item
    The Schmidt Law is rather a statistical description averaged over many
    galaxies, since some of them deviated substantially from the mean
    relation. \\
    For concrete models, the size of star-forming regions is required on linear
    scales, which is straightforward for normal galaxies, but might be more
    challenging for starburst galaxies, since in some cases the relevant
    regions are only a few percent of the parent galaxy's radius.

\item



\end{enumerate}


\end{document}
