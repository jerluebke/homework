\documentclass[11pt,a4paper]{scrartcl}
\usepackage[top=2cm,bottom=5.5cm,left=2cm,right=2cm]{geometry}
\usepackage{fontspec}
\usepackage{polyglossia}
    \setdefaultlanguage{english}
\usepackage{fancyhdr}
\usepackage{csquotes}
\usepackage{enumitem}
\usepackage{mathtools}
\usepackage{amssymb}
\usepackage{amsfonts}
\usepackage{siunitx}
    \sisetup{range-units=brackets}
    \DeclareSIUnit{\jansky}{Jy}
\usepackage{physics}
\usepackage{wasysym}
\usepackage{booktabs}
\usepackage[%
    labelformat=simple,
    labelsep=none,
    textformat=none,
    font={small,sc}
]{caption}
\usepackage{graphicx}
    \graphicspath{img}
\usepackage{tikz}
    \usetikzlibrary{calc}
\usepackage{pgfplots}
    \pgfplotsset{%
        table/search path={data},
        label style={font=\tiny},
        tick label style={font=\tiny}
    }
\usepackage{todonotes}
\usepackage[%
    % colorlinks=true, linkcolor=blue,
    hidelinks
]{hyperref}


\newcommand{\tablehead}[1]{\multicolumn{1}{c}{#1}}
\newcommand*{\figref}[1]{(see fig.~\ref{#1})}
\newcommand{\eg}{e.\,g.}

\newcommand{\course}{\textbf{Introduction to Astrophysics}}
\newcommand{\hwnumber}{1}
\newcommand{\nameA}{Jeremiah Lübke}
\newcommand{\nameB}{Andreas Menzel}
\newcommand{\matnumA}{108015230366}
\newcommand{\matnumB}{108015226385}
\newcommand{\groupnum}{Exercise Group 3}


\pagestyle{fancyplain}

\headheight 7\baselineskip
\lhead{%
    \course \\
    \vspace*{3\baselineskip}
    \nameA, \matnumA \\
    \nameB, \matnumB
}

\chead{%
    \textbf{\Large Homeworksheet \hwnumber} \\
    \vspace*{2\baselineskip}
}

\rhead{%
    \today \\
    \vspace*{4\baselineskip}
    \groupnum
}

\cfoot{\small\thepage}
\headsep 1.5em


\newcommand{\lambdamax}{\ensuremath{\lambda_{\mathrm{m}}}}
\newcommand{\numax}{\ensuremath{\nu_{\mathrm{m}}}}
\newcommand{\Temp}{\ensuremath{T}}
\newcommand{\Unu}{\ensuremath{u_{\nu}}}
\newcommand{\Utot}{\ensuremath{u_{\mathrm{tot}}}}
\newcommand{\Bnu}{\ensuremath{B_{\nu}}}


\begin{document}

\section*{Task 1}

\begin{enumerate}[label=\textbf{\large(\alph*)}, itemsep=2\baselineskip]

\item
    Since the CMB (\Temp = \SI{2.728}{\kelvin}) behaves like a Black Body, the maximum of the flux density
    coincides with the maximum of the spectral radiance, it can be determined
    via Wien's displacement law:
    \begin{align*}
        \lambdamax &= 0.0029 T^{-1} \si{\metre} \approx \SI{1.063}{\milli\metre}
        \\
        \numax &= \frac{c}{\lambdamax} \approx \SI{282}{\giga\hertz}
    \end{align*}

\item
    The spectral energy density is $\Unu = \frac{4\pi}{c}\Bnu$. For the total
    energy density, integrate over all frequencies using Planck's Law:
    \begingroup
    \addtolength{\jot}{1em}
    \begin{align*}
        \Unu =& \frac{8\pi h\nu^3}{c^3}\frac{1}{e^{\frac{h\nu}{k\Temp}}-1}
        \\\
        \Utot =&
        \frac{8\pi h}{c^3}\int_{0}^{\infty}\frac{\nu^3\dd\nu}{e^{\frac{h\nu}{k\Temp}}-1}
        \\
        \textrm{Substituting:}\quad &x = \frac{h\nu}{k\Temp},\quad \dd{x} = \frac{h}{k\Temp}\dd{\nu} \\
        \implies \Utot =& \frac{8\pi k^4\Temp^4}{c^3 h^3}
        \underbrace{%
        \int_{0}^{\infty}\frac{u^3 \dd u}{e^u-1}
        }_{=\Gamma(4)\zeta(4)=\frac{\pi^4}{15}}
        = \underbrace{\frac{8\pi^5 k^4}{15 c^3 h^3}}
        _{=\frac{\sigma}{c}}
        \Temp^4 \\
        \eval{\Utot}_{\Temp=\SI{2.728}{\kelvin}}\approx&\SI{4.190e-14}{\joule\per\metre\cubed}
    \end{align*}
    \endgroup
    For comparision, measurements yield
    $\Temp=\SI{4.005e-14}{\joule\per\metre\cubed}$.

\item
    The radiant flux is given by the Stefan-Boltzmann law:
    \begin{equation*}
        j=\sigma\,\Temp\overset{\Temp=\SI{2.728}{\kelvin}}{\approx}\SI{3.140e-6}{\watt\per\metre\squared}
    \end{equation*}
    To find the power received by the earth, one can use:
    \begin{equation*}
        P = A_{\earth}\,j \approx \SI{1601.83}{\watt}
    \end{equation*}

\end{enumerate}


\end{document}
