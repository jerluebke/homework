\documentclass[11pt,a4paper]{scrartcl}
\usepackage[top=2cm,bottom=5.5cm,left=2cm,right=2cm]{geometry}
\usepackage{fontspec}
\usepackage{polyglossia}
    \setdefaultlanguage{english}
\usepackage{fancyhdr}
\usepackage{csquotes}
\usepackage{enumitem}
\usepackage{mathtools}
\usepackage{amssymb}
\usepackage{amsfonts}
\usepackage{siunitx}
    \sisetup{range-units=brackets}
    \DeclareSIUnit{\jansky}{Jy}
    \DeclareSIUnit{\parsec}{pc}
    \DeclareSIUnit{\lightyear}{ly}
\usepackage{physics}
\usepackage{wasysym}
\usepackage{booktabs}
\usepackage[%
    labelformat=simple,
    labelsep=none,
    textformat=none,
    font={small,sc}
]{caption}
\usepackage{graphicx}
    \graphicspath{img}
\usepackage{tikz}
    \usetikzlibrary{calc}
\usepackage{pgfplots}
    \pgfplotsset{%
        table/search path={data},
        label style={font=\tiny},
        tick label style={font=\tiny}
    }
\usepackage{todonotes}
\usepackage[%
    % colorlinks=true, linkcolor=blue,
    hidelinks
]{hyperref}


\newcommand{\tablehead}[1]{\multicolumn{1}{c}{#1}}
\newcommand*{\figref}[1]{see fig.~\ref{#1}}
\newcommand{\eg}{e.\,g.}

\newcommand{\course}{\textbf{Introduction to Astrophysics}}
\newcommand{\hwnumber}{2}
\newcommand{\nameA}{Jeremiah Lübke}
\newcommand{\nameB}{Andreas Menzel}
\newcommand{\matnumA}{108015230366}
\newcommand{\matnumB}{108015226385}
\newcommand{\groupnum}{Exercise Group 3}


\pagestyle{fancyplain}

\headheight 7\baselineskip
\lhead{%
    \course \\
    \vspace*{3\baselineskip}
    \nameA, \matnumA \\
    \nameB, \matnumB
}

\chead{%
    \textbf{\Large Homeworksheet \hwnumber} \\
    \vspace*{2\baselineskip}
}

\rhead{%
    \today \\
    \vspace*{4\baselineskip}
    \groupnum
}

\cfoot{\small\thepage}
\headsep 1.5em


\begin{document}

\section*{Task 3}

\begin{enumerate}[label=\textbf{\large(\alph*)}, itemsep=2\baselineskip]

\item
    For a scematic sketch of synchrotron radiation, \figref{app:fig1}.
    \begin{figure}[h]
        \centering
        \newcommand{\eradius}{3pt}
\newcommand{\blineouterradius}{7pt}
\newcommand{\blineinnerradius}{1pt}

\newcommand{\bline}[1]{%
    \draw[thick,blue!80] (#1) circle (\blineouterradius);
    \fill[blue!80] (#1) circle (\blineinnerradius);
}

\newcommand{\radiationcone}{%
    (e) -- (10, 5.5) to [out=14.03624, in=90] (13.25, 4) to [out=-90,
    in=-14.03624] (10, 2.5) -- (e);
}

\begin{tikzpicture}[scale=1.2,>=stealth]

% gyration center at (4, -4), radius 8

% electron coordinates
\coordinate (e) at (4, 4);

% bline coordinates
\coordinate (B1) at (2, 2);
\coordinate (B2) at (2, 6);
\coordinate (B3) at (6, 6);
\coordinate (B4) at (6, 2);

% dashed radiation cone border
\draw [semithick,dashed,gray] (e) -- (13, 6.25);
\draw [semithick,dashed,gray] (e) -- (13, 1.75);
% radiation cone filling
% \fill [orange!30] \radiationcone;
% radiation direction
\draw [dashed,thick,gray,->] (e) -- (14, 4);
% electron trajectory
\draw [thick,->] (0.93853,3.39104) arc (112.5:45:8);
% radiation cone outline
\draw [very thick,orange!80] \radiationcone;

% draw electron and blines
\fill [blue!80!black] (e) circle (\eradius);
\foreach \bcoord in {B1, B2, B3, B4}
    \bline{\bcoord};

% opening angle of cone
\draw [semithick,|<-] (8.85071,5.21268) arc (14:4.5:5);
\draw [semithick,|<-] (8.85071,2.78732) arc (-14:0.5:5);
% Efield oscillation
\draw [thick,red!80!black,|->] (10.5,3.985) -- (10.5,5.59);

% labels
\node [right=1] at (B2) {$\vec{B}$};
\node [above] at (e) {$e^-$};
\node [right] at (10.5, 4.8) {$\vec{E}$};
% \node [right] at (10.6, 6.2) {$\Theta$};
\node  at (9, 4.2) {$\Theta$};

\end{tikzpicture}

        \caption{}
        \label{app:fig1}
    \end{figure}

\item
    The relativistic gyroradius is given as:
    \begin{equation*}
        r_{L} = \frac{p}{e\,B_{\perp}}.
    \end{equation*}

    Using the relativistic energy-momentum relation
    \begin{equation*}
        E^2 = (m\,c)^2 + (p\,c)^2,
    \end{equation*}

    we obtain:
    \begin{equation*}
        r_{L} = \frac{\sqrt{E^2 - (m\,c)^2}}{c\,e\,B_{\perp}}.
    \end{equation*}

    Where $m_e\,c^2 = \SI{511e3}{\electronvolt}$ and the prefactor has the value:
    \begin{equation*}
        \frac{1}{c\,e\,B_{\perp}} \approx \SI{16.68}{\metre\per\electronvolt}
    \end{equation*}

    We then receive the results:
    \begin{table}[h]
        \centering
        \begin{tabular}{SSS}
            \toprule
            \tablehead{Energies/\si{\giga\electronvolt}} &
            \tablehead{$r_{L}$/\SI{16.68}{\giga\metre}} &
            \tablehead{$r_{L}$/\SI{540.56}{\nano\parsec}} \\
            \midrule
            1 & 1 & 1 \\
            1e3 & 1e3 & 1e3 \\
            1e6 & 1e6 & 1e6 \\
            1e9 & 1e9 & 1e9 \\
            \bottomrule
        \end{tabular}
    \end{table}


\item
    For the opening angle of the radiation cone, we use:
    \begin{equation*}
        \Theta = \frac{1}{\gamma} = \frac{m\,c^2}{E}
    \end{equation*}

    and find:
    \begin{table}[h]
        \centering
        \begin{tabular}{SS}
            \toprule
            \tablehead{Energies/\si{\giga\electronvolt}} &
            \tablehead{$\Theta$/\SI{511}{\radian}} \\
            \midrule
            1 & 1 \\
            1e3 & 1e-3 \\
            1e6 & 1e-6 \\
            1e9 & 1e-9 \\
            \bottomrule
        \end{tabular}
    \end{table}

\end{enumerate}


\end{document}
