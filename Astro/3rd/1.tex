\documentclass[11pt,a4paper]{scrartcl}
\usepackage[top=2cm,bottom=5.5cm,left=2cm,right=2cm]{geometry}
\usepackage{fontspec}
\usepackage{polyglossia}
    \setdefaultlanguage{english}
\usepackage{fancyhdr}
\usepackage{csquotes}
\usepackage{enumitem}
\usepackage{mathtools}
\usepackage{amssymb}
\usepackage{amsfonts}
\usepackage{siunitx}
    \sisetup{range-units=brackets}
    \DeclareSIUnit{\parsec}{pc}
    \DeclareSIUnit{\Lsol}{\ensuremath{L_{\astrosun}}}
    \DeclareSIUnit{\Rsol}{\ensuremath{R_{\astrosun}}}
\usepackage{physics}
\usepackage{wasysym}
\usepackage{booktabs}
\usepackage[%
    labelformat=simple,
    labelsep=none,
    textformat=empty,
    font={small,sc}
]{caption}
\usepackage{graphicx}
    \graphicspath{img}
\usepackage{tikz}
    \usetikzlibrary{calc,external}
    \tikzexternalize[prefix=extern/]
    \tikzexternalize
    \tikzexternaldisable
\usepackage{pgfplots}
    \pgfplotsset{%
        compat=1.16,
        table/search path={data},
        % label style={font=\tiny},
        % tick label style={font=\tiny}
    }
\usepackage{todonotes}
\usepackage[%
    % colorlinks=true, linkcolor=blue,
    hidelinks
]{hyperref}


\newcommand{\tablehead}[1]{\multicolumn{1}{c}{#1}}
\newcommand*{\figref}[1]{(see fig.~\ref{#1})}
\newcommand{\eg}{e.\,g.}

\newcommand{\course}{\textbf{Introduction to Astrophysics}}
\newcommand{\hwnumber}{3}
\newcommand{\nameA}{Jeremiah Lübke}
\newcommand{\nameB}{Andreas Menzel}
\newcommand{\matnumA}{108015230366}
\newcommand{\matnumB}{108015226385}
\newcommand{\groupnum}{Exercise Group 3}


\pagestyle{fancyplain}

\headheight 7\baselineskip
\lhead{%
    \course \\
    \vspace*{3\baselineskip}
    \nameA, \matnumA \\
    \nameB, \matnumB
}

\chead{%
    \textbf{\Large Homework \hwnumber} \\
    \vspace*{2\baselineskip}
}

\rhead{%
    \today \\
    \vspace*{4\baselineskip}
    \groupnum
}

\cfoot{\small\thepage}
\headsep 1.5em


\DeclareSIUnit{\solarradii}{\ensuremath{R_{\astrosun}}}
\DeclareSIUnit{\solarlum}{\ensuremath{L_{\astrosun}}}


\newcommand{\La}{\ensuremath{L_{\textrm{A}}}}
\newcommand{\Ta}{\ensuremath{T_{\textrm{A}}}}
\newcommand{\Ra}{\ensuremath{R_{\textrm{A}}}}


\begin{document}

\section*{Task 1}

\begin{enumerate}[label=\textbf{\large(\alph*)}, itemsep=2\baselineskip]

\item
    In figure~\ref{app:fig1} Absolute Magnitude vs Color Index of \num{11890}
    stars from the \emph{Yale Bright Star Catalogue} and the
    \emph{Gliese Catalogue of Nearby Stars} is plotted. The stars are coloured
    according to their spectral class, where in addition to the standard
    classes O to M the class D for white dwarfs was included. \\
    The data is available here: \url{http://www.astronexus.com/hyg}. \\

    Since the above mentioned data set lacks temperature of the stars, the
    \emph{Theoretical Hertzsprung-Russel Diagram} is illustrated in
    figure~\ref{app:fig2}, where Luminosity in Solar Units vs Effective
    Temperature of \num{9879} stars from the \emph{V/19} catalogue is plotted.
    This data set also contains the \emph{Hyades} open star cluster. \\
    The data can be obtained from the \emph{VizieR} database:
    \url{http://vizier.u-strasbg.fr/viz-bin/VizieR}. \\
    For comparison the red giant \emph{Aldebaran} was also plotted.

\item
    Aldebaran's distance from the sun:
    \begin{equation*}
        d=\frac{1}{p}=\SI{20.83}{\parsec}
    \end{equation*}
    where $p=0.48''$.
    It is therefore not part of the Hyades, rather roughly halfway the
    distance. \\

    Wien's displacement law gives the temperature:
    \begin{equation*}
    \lambda_{\textrm{max}}\,T=\SI{0.0029}{\metre\kelvin}\implies
    \Ta\approx\SI{3973}{\kelvin}
    \end{equation*}

    Therewith the luminosity is:
    \begin{equation*}
        \La = 4\,\pi\,\Ra^2\,\sigma\,\Ta^4 \approx \SI{167.17e27}{\watt} \approx
        \SI{437}{\Lsol}
    \end{equation*}
    where $\Ra = \SI{44.13}{\Rsol}\approx\SI{30.70e9}{\metre}$. \\

    Aldebaran is currently on the red giant branch, wich means that it evolved
    off the mainsequence after burning the hydrogen in it's core, leaving it
    with a helium core.  It will continue to move on this branch towards higher
    luminosity and lower temperature until, through a helium-flash, the core is
    able to burn the helium.  Once helium burning started Aldebaran will start
    to move on the horizontal branch with near constant luminosity but
    increasing temperature.

\end{enumerate}

\newpage

\chead{%
    \textbf{\Large Homework \hwnumber~--~Appendix} \\
    \vspace*{2\baselineskip}
}
\cfoot{}

% \section*{Anhang}

\begin{figure}[h]
    \centering
    \begin{tikzpicture}

\begin{axis}[%
    set layers,
    width=17cm,
    height=23cm,
    title=Theoretical HR Diagram: h and $\chi$ Persei,
    title style={font=\large},
    xlabel=$\log T$,
    ylabel=$L/\si{\Lsol}$,
    label style={font=\small},
    x label style={yshift=.2em},
    y label style={yshift=-.8em},
    ticklabel style={font=\small},
    x tick label style={%
        /pgf/number format/.cd,
            fixed,
            fixed zerofill,
            precision=1,
        /tikz/.cd
    },
    x dir=reverse,
    ymode=log,
    xmax=4.7,
    xmin=3.4,
    ymax=5e5,
    ymin=1e2,
    grid=major,
    view={0}{90},
    colormap={mycm}{color=(gray),color=(gray)},
]
    \addplot+ [%
        blue!80!black,
        only marks,
        mark options={fill=blue!60!white,}
    ] table [%
        x=logT,
        % y=Mbol,
        y expr=10^(0.4*(4.74-\thisrow{Mbol})),
        col sep=comma
    ] {asu-h-chi-per.tsv};

    \addplot [%
        contour prepared={labels=false},
        contour plot/.style={draw color=grey},
    ] table {%
        4.7     5685    1
        4.26133 1e2     1

        4.7     568.5e3 10
        3.76122 1e2     10

        4.7     56.85e6 100
        3.26133 1e2     100
    };

    \pgfonlayer{axis foreground}
        \node [align=center, fill=white, inner
            sep=1pt,font=\scriptsize,rotate=-62.5]
            at (axis cs: 4.35, 226) {\SI{1}{\Rsol}};
        \node [align=center, fill=white, inner
            sep=1pt,font=\scriptsize,rotate=-62.5]
            at (axis cs: 3.85, 226) {\SI{10}{\Rsol}};
        \node [align=center, fill=white, inner
            sep=1pt,font=\scriptsize,rotate=-62.5]
            at (axis cs: 3.55, 1428) {\SI{100}{\Rsol}};
    \endpgfonlayer

\end{axis}

\end{tikzpicture}


    \caption{}
    \label{app:fig1}
\end{figure}



\end{document}
