\documentclass[11pt,a4paper]{scrartcl}
\usepackage[top=2cm,bottom=5.5cm,left=2cm,right=2cm]{geometry}
\usepackage{fontspec}
\usepackage{polyglossia}
    \setdefaultlanguage{english}
\usepackage{fancyhdr}
\usepackage{csquotes}
\usepackage{enumitem}
\usepackage{mathtools}
\usepackage{amssymb}
\usepackage{amsfonts}
\usepackage{siunitx}
    \sisetup{range-units=brackets}
    \DeclareSIUnit{\parsec}{pc}
    \DeclareSIUnit{\Lsol}{\ensuremath{L_{\astrosun}}}
    \DeclareSIUnit{\Rsol}{\ensuremath{R_{\astrosun}}}
\usepackage{physics}
\usepackage{wasysym}
\usepackage{booktabs}
\usepackage[%
    labelformat=simple,
    labelsep=none,
    textformat=empty,
    font={small,sc}
]{caption}
\usepackage{graphicx}
    \graphicspath{img}
\usepackage{tikz}
    \usetikzlibrary{calc,external}
    \tikzexternalize[prefix=extern/]
    \tikzexternalize
    \tikzexternaldisable
\usepackage{pgfplots}
    \pgfplotsset{%
        compat=1.16,
        table/search path={data},
        % label style={font=\tiny},
        % tick label style={font=\tiny}
    }
\usepackage{todonotes}
\usepackage[%
    % colorlinks=true, linkcolor=blue,
    hidelinks
]{hyperref}


\newcommand{\tablehead}[1]{\multicolumn{1}{c}{#1}}
\newcommand*{\figref}[1]{(see fig.~\ref{#1})}
\newcommand{\eg}{e.\,g.}

\newcommand{\course}{\textbf{Introduction to Astrophysics}}
\newcommand{\hwnumber}{3}
\newcommand{\nameA}{Jeremiah Lübke}
\newcommand{\nameB}{Andreas Menzel}
\newcommand{\matnumA}{108015230366}
\newcommand{\matnumB}{108015226385}
\newcommand{\groupnum}{Exercise Group 3}


\pagestyle{fancyplain}

\headheight 7\baselineskip
\lhead{%
    \course \\
    \vspace*{3\baselineskip}
    \nameA, \matnumA \\
    \nameB, \matnumB
}

\chead{%
    \textbf{\Large Homework \hwnumber} \\
    \vspace*{2\baselineskip}
}

\rhead{%
    \today \\
    \vspace*{4\baselineskip}
    \groupnum
}

\cfoot{\small\thepage}
\headsep 1.5em


\DeclareSIUnit{\solarradii}{\ensuremath{R_{\astrosun}}}
\DeclareSIUnit{\solarlum}{\ensuremath{L_{\astrosun}}}


\newcommand{\La}{\ensuremath{L_{\textrm{A}}}}
\newcommand{\Ta}{\ensuremath{T_{\textrm{A}}}}
\newcommand{\Ra}{\ensuremath{R_{\textrm{A}}}}


\begin{document}

\section*{Task 1}

\begin{enumerate}[label=\textbf{\large(\alph*)}, itemsep=2\baselineskip]

\item
    In figure~\ref{app:fig1} Absolute Magnitude vs Color Index of \num{11890}
    stars from the \emph{Yale Bright Star Catalogue} and the
    \emph{Gliese Catalogue of Nearby Stars} is plotted. The stars are coloured
    according to their spectral class, where in addition to the standard
    classes O to M the class D for white dwarfs was included. \\
    The data is available here: \url{http://www.astronexus.com/hyg}. \\

    Since the above mentioned data set lacks temperature of the stars, the
    \emph{Theoretical Hertzsprung-Russel Diagram} is illustrated in
    figure~\ref{app:fig2}, where Luminosity in Solar Units vs Effective
    Temperature of \num{9879} stars from the \emph{V/19} catalogue is plotted.
    This data set also contains the \emph{Hyades} open star cluster. \\
    The data can be obtained from the \emph{VizieR} database:
    \url{http://vizier.u-strasbg.fr/viz-bin/VizieR}. \\
    For comparison the red giant \emph{Aldebaran} was also plotted.

\item
    Aldebaran's distance from the sun:
    \begin{equation*}
        d=\frac{1}{p}=\SI{20.83}{\parsec}
    \end{equation*}
    where $p=0.48''$.
    It is therefore not part of the Hyades, rather roughly halfway the
    distance. \\

    Wien's displacement law gives the temperature:
    \begin{equation*}
    \lambda_{\textrm{max}}\,T=\SI{0.0029}{\metre\kelvin}\implies
    \Ta\approx\SI{3973}{\kelvin}
    \end{equation*}

    Therewith the luminosity is:
    \begin{equation*}
        \La = 4\,\pi\,\Ra^2\,\sigma\,\Ta^4 \approx \SI{167.17e27}{\watt} \approx
        \SI{437}{\Lsol}
    \end{equation*}
    where $\Ra = \SI{44.13}{\Rsol}\approx\SI{30.70e9}{\metre}$. \\

    Aldebaran is currently on the red giant branch, wich means that it evolved
    off the mainsequence after burning the hydrogen in it's core, leaving it
    with a helium core.  It will continue to move on this branch towards higher
    luminosity and lower temperature until, through a helium-flash, the core is
    able to burn the helium.  Once helium burning started Aldebaran will start
    to move on the horizontal branch with near constant luminosity but
    increasing temperature.

\end{enumerate}

\newpage

\chead{%
    \textbf{\Large Homework \hwnumber~--~Appendix} \\
    \vspace*{2\baselineskip}
}
\cfoot{}

% \section*{Anhang}

\begin{figure}[h]
    \centering
    \tikzexternalenable
\tikzsetnextfilename{exhrd}

\begin{tikzpicture}[>=stealth,pin distance=3cm,every pin edge/.style={<-}]

\begin{axis}[%
    set layers,
    width=17cm,
    height=22cm,
    title=Hertzsprung-Russel Diagram: Harvard Revised{,} Gliese,
    title style={yshift=1.5cm,font=\large},
    label style={font=\small},
    x label style={yshift=.4em},
    y label style={yshift=-.4em},
    tick label style={font=\small},
    xlabel=Colour Index $(B-V)$,
    ylabel=Absolute Magnitude,
    ymin=-18,
    ymax=20,
    xmin=-0.5,
    xmax=2.5,
    y dir=reverse,
    xtick distance=0.5,
    % xtick={3000,4000,5000,6000,7500,10000,12000,30000},
    % xticklabel style={%
    %     /pgf/number format/.cd,
    %     fixed,precision=0,
    %     1000 sep={\,},
    %     /tikz/.cd,
    % },
    % xticklabel=\pgfmathparse{exp(\tick)}\pgfmathprintnumber{\pgfmathresult},
    grid=major,
    scatter/classes={%
        d={mark=*,lightgray,draw=gray!60!black},
        o={mark=*,blue!60!white,draw=blue!80!black},
        b={mark=*,cyan!80!white,draw=cyan!60!black},
        a={mark=*,white,draw=gray},
        f={mark=*,lime!80!white,draw=lime!60!black},
        g={mark=*,yellow!80!white,draw=yellow!60!black},
        k={mark=*,orange!80!white,draw=orange!60!black},
        m={mark=*,red!60!white,draw=red!60!black}
    },
    colormap={mycm}{%
        color=(blue!60!white),
        color=(cyan),
        color=(white),
        color=(lime),
        color=(yellow),
        color=(orange),
        color=(red),
        color=(lightgray)
    },
    colorbar horizontal,
    colorbar as palette,
    colorbar style={%
        title=Spectral Class,
        title style={yshift=-.2cm,font=\small},
        at={(0.5,1.03)},
        anchor=south,
        xticklabel pos=upper,
        xticklabel interval boundaries,
        xticklabels={,O,B,A,F,G,K,M,D},
        x tick label style={below,font=\bfseries},
        xtick distance=1,
        samples=8
    }
]
    \addplot+ [%
        only marks,
        scatter,
        point meta=explicit symbolic,
    ] table [%
        x=ci,
        % y=lum,
        y=absmag,
        meta=spect,
        col sep=comma
    ] {hygdata_v3_filtered.csv};

    \addplot+ [%
        % orange!80!black,
        % orange!80!yellow,
        yellow!60!black,
        mark=*,
        only marks,
        mark size=3pt,
        mark options={very thick,fill=yellow}
    ] coordinates {%
        (0.63, 4.83)
    };

    \pgfonlayer{axis foreground}
        \draw [->,shorten >=2.5pt,semithick] (axis cs:2.0,4.0) -- (axis cs:0.63,4.83);
        \node at (axis cs:2.0,4.0) [%
                rectangle,rounded corners,
                draw,fill=white,
                align=left,font=\scriptsize]
            {\small\textbf{Sol} \\ $M=4.83$ \\ $B-V=0.63$};
    \endpgfonlayer
\end{axis}

\end{tikzpicture}

\tikzexternaldisable

    \caption{}
    \label{app:fig1}
\end{figure}

\newpage

\begin{figure}[h]
    \centering
    \tikzexternalenable
\tikzsetnextfilename{exthhrd}

\begin{tikzpicture}

\begin{axis}[%
    set layers,
    width=17cm,
    height=22cm,
    title=Theoretical HR Diagram: V/19,
    title style={font=\large},
    xlabel=$T/\si{\kelvin}$,
    ylabel=$L/\si{\solarlum}$,
    label style={font=\small},
    x label style={yshift=.2em},
    y label style={yshift=-.8em},
    ticklabel style={font=\small},
    x dir=reverse,
    xmode=log,
    ymode=log,
    xmax=10e5,
    xmin=10e1,
    ymax=10e8,
    ymin=10e-5,
    grid=major,
    view={0}{90},
    colormap={mycm}{color=(gray),color=(gray)},
    legend entries={V/19, V/19/Hyades, Aldebaran},
    legend cell align=left,
    legend pos=south west,
    legend style={rounded corners,mark size=3pt}
]
    \addplot+ [%
        blue!80!black,
        only marks,
        mark options={fill=blue!60!white,}
    ] table [%
        x=T,
        y=L,
        col sep=comma
    ] {asu-piskunov1980-nolog.csv};

    \addplot+ [%
        yellow!60!black,
        mark=*,
        only marks,
        mark options={thick,fill=yellow!80!white}
    ] table [%
        x=T,
        y=L,
        col sep=comma
    ] {asu-piskunov1980-hyades-nolog.csv};

    \addplot+ [%
        green!60!black,
        mark=*,
        only marks,
        mark size=3pt,
        mark options={thick,fill=green!80!white}
    ] coordinates {%
        (3910, 518)
    };

    \addplot [%
        contour prepared={labels=false},
        contour plot/.style={draw color=grey},
    ] table {%
        1e6     9e-4    1e-6
        5.772e5 1e-4    1e-6

        1e6     9       1e-4
        5.772e4 1e-4    1e-4

        1e6     9e4     1e-2
        5.772e3 1e-4    1e-2

        1e6     9e8     1
        5.772e2 1e-4    1

        1e6     9e12    1e2
        5.772e1 1e-4    1e2

        1e6     9e16    1e4
        5.772   1e-4    1e4

        1e6     9e20    1e6
        5.772e-1    1e-4    1e6
    };

    \pgfonlayer{axis foreground}
        \node [align=center,
            % draw,
            fill=white,
            inner sep=1pt,font=\scriptsize,rotate=-59]
            at (axis cs: 324583, 1e-1) {\SI{1e-4}{\solarradii}};
        \node [align=center, fill=white, inner sep=1pt,font=\scriptsize,rotate=-59]
            at (axis cs: 324583, 1e3) {\SI{1e-2}{\solarradii}};
        \node [align=center, fill=white, inner sep=1pt,font=\scriptsize,rotate=-59]
            at (axis cs: 324583, 1e7) {\SI{1}{\solarradii}};
        \node [align=center, fill=white, inner sep=1pt,font=\scriptsize,rotate=-59]
            at (axis cs: 57720, 1e8) {\SI{1e2}{\solarradii}};
        \node [align=center, fill=white, inner sep=1pt,font=\scriptsize,rotate=-59]
            at (axis cs: 5772, 1e8) {\SI{1e4}{\solarradii}};
        \node [align=center, fill=white, inner sep=1pt,font=\scriptsize,rotate=-59]
            at (axis cs: 577, 1e8) {\SI{1e6}{\solarradii}};
    \endpgfonlayer

\end{axis}

\end{tikzpicture}

\tikzexternaldisable

    \caption{}
    \label{app:fig2}
\end{figure}

\newpage



\end{document}
