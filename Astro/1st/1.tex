\documentclass[11pt,a4paper]{scrartcl}
\usepackage[top=2cm,bottom=5.5cm,left=2cm,right=2cm]{geometry}
\usepackage{fontspec}
\usepackage{polyglossia}
    \setdefaultlanguage{german}
\usepackage{fancyhdr}
\usepackage{csquotes}
\usepackage{enumitem}
\usepackage{mathtools}
\usepackage{amssymb}
\usepackage{amsfonts}
\usepackage{siunitx}
    \sisetup{range-units=brackets}
    \DeclareSIUnit{\jansky}{Jy}
\usepackage{physics}
\usepackage{wasysym}
\usepackage{booktabs}
\usepackage[%
    labelformat=simple,
    labelsep=none,
    textformat=none,
    font={small,sc}
]{caption}
\usepackage{graphicx}
    \graphicspath{img}
\usepackage{pgfplots}
    \pgfplotsset{%
        table/search path={data},
        label style={font=\tiny},
        tick label style={font=\tiny}
    }
\usepackage{todonotes}
\usepackage[%
    % colorlinks=true, linkcolor=blue,
    hidelinks
]{hyperref}

\newcommand{\tablehead}[1]{\multicolumn{1}{c}{#1}}
\newcommand*{\figref}[1]{(siehe Abb.~\ref{#1})}
\newcommand{\zB}{z.\,B.}

\newcommand{\course}{\textbf{Einführung in die Astrophysik}}
\newcommand{\hwnumber}{1}
\newcommand{\nameA}{Jeremiah Lübke}
\newcommand{\nameB}{Andreas Menzel}
\newcommand{\matnumA}{108015230366}
\newcommand{\matnumB}{108015226385}
\newcommand{\groupnum}{Übungsgruppe 3}


\pagestyle{fancyplain}
\headheight 7\baselineskip
\lhead{%
    \course \\
    \vspace*{3\baselineskip}
    \nameA, \matnumA \\
    \nameB, \matnumB
}

\chead{%
    \textbf{\Large Hausaufgabenblatt \hwnumber} \\
    \vspace*{2\baselineskip}
}

\rhead{%
    \today \\
    \vspace*{4\baselineskip}
    \groupnum
}

\cfoot{\small\thepage}
\headsep 1.5em


\newcommand{\Lum}{\mathrm{L}}
\newcommand{\Area}{\mathrm{A}}
\newcommand{\Temp}{\mathrm{T}}
\newcommand{\lambdamax}{\lambda_{\mathrm{max}}}


\begin{document}

\section*{Aufgabe 1}

\begin{enumerate}[label=\textbf{\large(\alph*)}]

\item
    Mit dem Stefan-Boltzmann-Gesetz in der Form $\Lum = \Area\,\sigma\,\Temp^4$:
    \begin{equation*}
        \frac{\Lum}{\Lum_{\astrosun}} =
        \frac{\Area\,\Temp^4}{\Area_{\astrosun}\,\Temp_{\astrosun}^4}
        \overset{\Area\approx\Area_{\astrosun}}{\approx}
        \frac{\Temp^4}{\Temp_{\astrosun}^4} =
        \frac{\SI{23112}{\kelvin}}{\SI{5778}{\kelvin}} = 4
    \end{equation*}

\vspace*{\baselineskip}

\item
    Aus dem Wienschen Verschiebungsgesetz erhält man:
    $\lambdamax = 0.0029\,\Temp^{-1}\,\si{\metre}$
    \begin{equation*}
        \implies
        \begin{cases}
            \lambdamax(\Temp=\SI{5778}{\kelvin}) = \SI{501.9}{\nano\metre} &
            \quad \mathrm{sichtbar (blau-grün)} \\
            \lambdamax(\Temp=\SI{23112}{\kelvin}) = \SI{125.5}{\nano\metre} &
            \quad \mathrm{Vakuum-UV}
        \end{cases}
    \end{equation*}

\vspace*{\baselineskip}

\item
    $\Lum_{\astrosun}\approx\SI{3.8e26}{\watt}$,
    $\SI{1}{\astronomicalunit}\approx\SI{149.6e9}{m}$
    \begin{enumerate}[label=\textbf{(\roman*)}]
        \item Solarkonstante Erde
            \begin{equation*}
                S_{\earth} = \dv{\Lum_{\astrosun}}{\Area} \approx
                \frac{\Lum_{\astrosun}}{4\pi r_{\mathrm{S-E}}^2}
                \underset{r_{\mathrm{S-E}}=\SI{1}{\astronomicalunit}}{\approx}
                \SI{1351.2}{\watt\per\metre\squared}
            \end{equation*}

        \item Solarkonstante Saturn
            \begin{equation*}
                S_{\saturn} \approx
                \SI{14.72}{\watt\per\metre\squared}
            \end{equation*}
    \end{enumerate}


\end{enumerate}


\end{document}
