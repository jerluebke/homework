\documentclass[11pt,a4paper]{scrartcl}
\usepackage[top=2cm,bottom=5.5cm,left=2cm,right=2cm]{geometry}
\usepackage{fontspec}
\usepackage{polyglossia}
    \setdefaultlanguage{german}
\usepackage{fancyhdr}
\usepackage{csquotes}
\usepackage{enumitem}
\usepackage{mathtools}
\usepackage{amssymb}
\usepackage{amsfonts}
\usepackage{siunitx}
    \sisetup{range-units=brackets}
    \DeclareSIUnit{\jansky}{Jy}
\usepackage{physics}
\usepackage{wasysym}
\usepackage{booktabs}
\usepackage[%
    labelformat=simple,
    labelsep=none,
    textformat=none,
    font={small,sc}
]{caption}
\usepackage{graphicx}
    \graphicspath{img}
\usepackage{pgfplots}
    \pgfplotsset{%
        table/search path={data},
        label style={font=\tiny},
        tick label style={font=\tiny}
    }
\usepackage{todonotes}
\usepackage[%
    % colorlinks=true, linkcolor=blue,
    hidelinks
]{hyperref}

\newcommand{\tablehead}[1]{\multicolumn{1}{c}{#1}}
\newcommand*{\figref}[1]{(siehe Abb.~\ref{#1})}
\newcommand{\zB}{z.\,B.}

\newcommand{\course}{\textbf{Einführung in die Astrophysik}}
\newcommand{\hwnumber}{1}
\newcommand{\nameA}{Jeremiah Lübke}
\newcommand{\nameB}{Andreas Menzel}
\newcommand{\matnumA}{108015230366}
\newcommand{\matnumB}{108015226385}
\newcommand{\groupnum}{Übungsgruppe 3}


\pagestyle{fancyplain}
\headheight 7\baselineskip
\lhead{%
    \course \\
    \vspace*{3\baselineskip}
    \nameA, \matnumA \\
    \nameB, \matnumB
}

\chead{%
    \textbf{\Large Hausaufgabenblatt \hwnumber} \\
    \vspace*{2\baselineskip}
}

\rhead{%
    \today \\
    \vspace*{4\baselineskip}
    \groupnum
}

\cfoot{\small\thepage}
\headsep 1.5em


\DeclareSIUnit{\erg}{erg}
\DeclareSIUnit{\mag}{mag}
\newcommand{\Fnu}{\ensuremath{\mathrm{F}_{\nu}}}
\newcommand{\Inu}{\ensuremath{\mathrm{I}_{\nu}}}
\newcommand{\Mab}{\ensuremath{m_{\mathrm{AB}}}}
\newcommand{\Pow}{\ensuremath{\mathrm{P}}}
\newcommand{\Area}{\ensuremath{\mathrm{A}}}


\begin{document}

\section*{Aufgabe 2}

Definition der AB-Magnitude: $\Mab = -2.5\,\log(\Fnu) - 48.60$

\begin{enumerate}[label=\textbf{\large(\alph*)}, itemsep=2\baselineskip]

\item
    Strahlungsflussdichte bei nullter AB-Magnitude:
    \begin{equation*}
        \implies \eval{\Fnu}_{\Mab=0} =
        \eval{10^{\frac{\Mab+48.60}{-2.5}}}_{\Mab=0} = 10^{\frac{48.60}{-2.5}}
        \approx \SI{3630.78}{\jansky}
    \end{equation*}


\item
    Betrachte im Abstand von \SI{2}{\metre} eine Lampe mit
    \begin{itemize}
        \item \SI{2}{\watt} Strahlungsleistung
        \item Konstantem Spektrum für Wellenlängen in
            \SIrange[range-units=repeat]{30e3}{300}{\nano\metre}
            $\implies \Delta\nu=\SI{297e-7}{\metre}$
    \end{itemize}
    \medskip
    Die Strahlungsflussdichte ist gegeben als:
    \begin{align*}
        \Fnu &= \int_{\mathrm{Halbraum}}\Inu\dd{\Omega} \\
        \Inu &\propto \frac{\dd\Pow}{\dd\Area\,\dd\Omega\,\dd\nu}
        \overset{\mathrm{hier}}{\propto}
        \dv{\left(\frac{\Pow}{\Area\,\Delta\nu}\right)}{\Omega} \\
        &\implies \Fnu = \frac{\Pow}{\Area\,\Delta\nu}
    \end{align*}

    und mit $\Area = \Area_\mathrm{Halbraum} = 2\pi\,r^2$:
    \begin{align*}
        &\Fnu(r=\SI{2}{\metre}) =
        \SI{2679.38}{\watt\per\metre\squared\per\hertz} =
        \SI{2.68}{\erg\per\second\per\centi\metre\squared\per\hertz} \\
        &\implies \Mab = \SI{-49.67}{\mag}
    \end{align*}


\item
    Die Lampe befinde sich nun auf dem Mond (Abstand \SI{385000}{\kilo\metre}):
    \begin{align*}
        &\Fnu(r=\SI{385e6}{\metre}) =
        \SI{72.31e-15}{\watt\per\metre\squared\per\hertz} =
        \SI{72.31e-18}{\erg\per\second\per\centi\metre\squared\per\hertz} \\
        &\implies \Mab = \SI{-8.25}{\mag}
    \end{align*}

\end{enumerate}


\end{document}
