\documentclass[11pt,a4paper]{scrartcl}
\usepackage{fontspec}
\usepackage{polyglossia}
    \setdefaultlanguage{german}
\usepackage{mathtools}
\usepackage{amssymb}
\usepackage{booktabs}
\usepackage{siunitx}
    \sisetup{range-units=brackets}
\usepackage{pgfplots}
    \pgfplotsset{%
        table/search path={data},
    }
\usepackage[colorlinks=true]{hyperref}

\newcommand{\head}[1]{\multicolumn{1}{c}{#1}}

\DeclareSIUnit{\jansky}{Jy}

\title{Einführung in die Astrophysik}
\subject{Hausaufgabenblatt 1}
\author{Jeremiah Lübke}
\date{\today}

\begin{document}

\section*{Aufgabe 3}
\subsection*{a}

\begin{table}[h]
\centering
\begin{tabular}{rrrc}
    \toprule
    \head{E} & \head{$\lambda$} & \head{$\nu$} & \head{Frequenzbereich} \\
    \midrule
    \SIrange{0.1}{2.4}{\kilo\electronvolt} & \SIrange{12.4}{0.5}{\nano\metre} &
    \SIrange{24.2}{580.3}{\peta\hertz} & Röntgenstrahlen \\
    2.8 \si{\electronvolt} & 442.0 \si{\nano\metre} & 678.3 \si{\giga\hertz} &
    Blaue Licht \\
    0.6 \si{\electronvolt} & 2.2 \si{\micro\metre} & 136.3 \si{\giga\hertz} &
    nahes Infrarot \\
    44.3 \si{\micro\electronvolt} & 28.0 \si{\milli\metre} & 10.7 \si{\giga\hertz}
    &  Millimeterwellen \\
    5.8 \si{\micro\electronvolt} & 214.1 \si{\milli\metre} & 1.4 \si{\giga\hertz}
    & Millimeterwellen \\
    306.0 \si{\nano\electronvolt} & 4.1 \si{\metre} & 74.0 \si{\mega\hertz} &
    Ultrakurzwelle \\
    \bottomrule
\end{tabular}
\caption{Einige Messwerte von Quasar 3C138}
\label{tab:tab1}
\end{table}

\subsection*{b}
\begin{figure}[h]
    \centering
    \begin{tikzpicture}[%
        scale=1.5,
        every mark/.append style={%
            scale=0.8,
        }
    ]
    \begin{loglogaxis}[%
        title=Spektrale Energieverteilung des Quasars 3C138,
        xlabel=$\nu$/\si{\hertz},
        ylabel=$\mathrm{F}_{\nu}$/\si{\jansky},
        grid=major,
        xmin=1e7, xmax=1e18,
        ymin=1e-7, ymax=1e2,
        % max space between ticks=20
    ]
        \addplot+ [%
            only marks,
            error bars/.cd,
            y dir=both,
            y explicit,
            error bar style={black},
        ] table [%
            x=frequency,
            y=flux_density,
            y error=uncertainty,
            col sep=comma
        ] {sed-3c138.csv};
    \end{loglogaxis}
    \end{tikzpicture}
    \caption{SED Plot 3C138, Quelle der Daten:
        \href{https://ned.ipac.caltech.edu/}{NED}}
    \label{fig:fig1}
\end{figure}

\end{document}
