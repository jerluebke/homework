\documentclass[11pt,a4paper]{scrartcl}
\usepackage[top=2cm,bottom=5.5cm,left=2cm,right=2cm]{geometry}
\usepackage{fontspec}
\usepackage{polyglossia}
    \setdefaultlanguage{german}
\usepackage{fancyhdr}
\usepackage{csquotes}
\usepackage{enumitem}
\usepackage{mathtools}
\usepackage{amssymb}
\usepackage{amsfonts}
\usepackage{booktabs}
\usepackage{siunitx}
    \sisetup{range-units=brackets}
    \DeclareSIUnit{\jansky}{Jy}
\usepackage{physics}
\usepackage[%
    labelformat=simple,
    labelsep=none,
    textformat=none,
    font={small,sc}
]{caption}
\usepackage{graphicx}
    \graphicspath{img}
\usepackage{pgfplots}
    \pgfplotsset{%
        table/search path={data},
        label style={font=\tiny},
        tick label style={font=\tiny}
    }
    \usepackage[disable]{todonotes}
\usepackage[%
    % colorlinks=true, linkcolor=blue,
    hidelinks
]{hyperref}

\newcommand{\head}[1]{\multicolumn{1}{c}{#1}}
\newcommand*{\figref}[1]{(siehe Abb.~\ref{#1})}
\newcommand{\zB}{z.\,B.}

\newcommand{\course}{\textbf{Einführung in die Astrophysik}}
\newcommand{\hwnumber}{1}
\newcommand{\nameA}{Jeremiah Lübke}
\newcommand{\nameB}{Andreas Menzel}
\newcommand{\matnumA}{108015230366}
\newcommand{\matnumB}{108015226385}
\newcommand{\groupnum}{Übungsgruppe 3}


\pagestyle{fancyplain}
\headheight 7\baselineskip
\lhead{%
    \course \\
    \vspace*{3\baselineskip}
    \nameA, \matnumA \\
    \nameB, \matnumB
}

\chead{%
    \textbf{\Large Hausaufgabenblatt \hwnumber} \\
    \vspace*{2\baselineskip}
}

\rhead{%
    \today \\
    \vspace*{4\baselineskip}
    \groupnum
}

\cfoot{\small\thepage}
\headsep 1.5em


\newcommand{\Fnu}{\ensuremath{{F}_{\nu}}}


\begin{document}

\section*{Aufgabe 3}

\begin{enumerate}[label=\textbf{\large(\alph*)}]

\item
Einige Messwerte von Quasar 3C138

\begin{table}[h]
\centering
\begin{tabular}{rrrc}
    \toprule
    \head{E} & \head{$\lambda$} & \head{$\nu$} & \head{Frequenzbereich} \\
    \midrule
    \SIrange{0.1}{2.4}{\kilo\electronvolt} & \SIrange{12.4}{0.5}{\nano\metre} &
    \SIrange{24.2}{580.3}{\peta\hertz} & Röntgenstrahlen \\
    2.8 \si{\electronvolt} & 442.0 \si{\nano\metre} & 678.3 \si{\giga\hertz} &
    Blaue Licht \\
    0.6 \si{\electronvolt} & 2.2 \si{\micro\metre} & 136.3 \si{\giga\hertz} &
    nahes Infrarot \\
    44.3 \si{\micro\electronvolt} & 28.0 \si{\milli\metre} & 10.7 \si{\giga\hertz}
    &  Millimeterwellen \\
    5.8 \si{\micro\electronvolt} & 214.1 \si{\milli\metre} & 1.4 \si{\giga\hertz}
    & Millimeterwellen \\
    306.0 \si{\nano\electronvolt} & 4.1 \si{\metre} & 74.0 \si{\mega\hertz} &
    Ultrakurzwelle \\
    \bottomrule
\end{tabular}
\label{tab:tab1}
\end{table}

wobei $E = h\,\nu$ und $\nu = \frac{c}{\lambda}$ mit $h =
\SI{4.135667662e-15}{\electronvolt\second}$.

\vspace*{\baselineskip}

\item
Spektrale Energieverteilung des Quasars 3C138: \Fnu~gegen
$\nu$ \figref{app:fig1}. \\
\textit{\small Die geplotteten Daten stammen von der
    \href{https://ned.ipac.caltech.edu/}{NASA Extragalactic Database}.}

    \begin{itemize}
        \item Der Plot suggeriert einen (auf der doppelt logarithmischen
            Skala) fast linearen Verlauf für Frequenzen $>10^9$ \si{\hertz},
            also einen exponentiellen Anstieg der Strahlungsflussdichte
            \Fnu~für kleiner werdende Frequenzen
        \item Bei $10^8$ \si{\hertz} scheint der Verlauf ein Maximum zu
            erreichen; an dieser Stelle könnten weitere Daten für kleinere
            Frequenzen interessant sein
    \end{itemize}

\vspace*{\baselineskip}

\item
Spektrale Energieverteilung des Quasars 3C138: $\nu\,\Fnu$
gegen $\nu$ \figref{app:fig2}. \\
\textit{\small Die geplotteten Daten basieren auf dem in Teil (b)
    verwendeten Datensatz. Die y-Fehler wurden gemäß
    $\Delta\left(\nu\,\Fnu\right) = \abs{\pdv{(\nu\,\Fnu)}{\Fnu}}\,\Delta\Fnu =
    \nu\,\Delta\Fnu$ berechnet.}

    \begin{itemize}
        \item Die in Teil (b) betrachtete Strahlungsflussdichte
            \Fnu~beschreibt die Strahlungsleistung pro Fläche
            pro Frequenz für \emph{eine spezifische} Frequenz
        \item Das hier betrachtete Produkt $\nu\,\Fnu$
            beschreibt hingegen die Strahlungsleitung pro Fläche
        \item In einer Dimensionsbetrachtung reduziert sich ersteres zu
            Energie pro Fläche, wohingegen letzteres tatsächlich eine Leistung
            pro Fläche beschreibt, und damit die eigentlich interessante Größe
            ist, wenn man Verläufe von Strahlungsleistungen betrachtet
        \item Der Verlauf nimmt eine andere Form an, deren Maximum nun
            deutlich erkennbar bei $\sim 10^{13}$ \si{\hertz} liegt
    \end{itemize}

\end{enumerate}

\newpage

\chead{%
    \textbf{\Large Hausaufgabenblatt \hwnumber~--~Anhang} \\
    \vspace*{2\baselineskip}
}
\cfoot{}

% \section*{Anhang}

\begin{figure}[h]
    \centering
    \begin{tikzpicture}[%
    rotate=270,
    scale=1.5,
    every mark/.append style={%
        scale=1.0,
    }
]
    \begin{loglogaxis}[%
        width=15cm,
        height=10.5cm,
        title=Spektrum des Quasars 3C138: \Fnu~gegen $\nu$,
        xlabel=$\nu$/\si{\hertz},
        ylabel=\Fnu/\si{\jansky},
        grid=major,
        xmin=1e7, xmax=1e18,
        ymin=1e-7, ymax=1e2,
        % max space between ticks=20
    ]
        \addplot+ [%
            only marks,
            error bars/.cd,
                y dir=both,
                y explicit,
                error bar style={gray,line width=0.7pt},
        ] table [%
            x=frequency,
            y=flux_density,
            y error=uncertainty,
            col sep=comma
        ] {sed-3c138.csv};

        % correcting truncated error bars
        \addplot+ [%
            blue,
            mark=*,
            mark options={fill=blue!80!black},
            only marks,
            error bars/.cd,
                y dir=minus,
                y explicit,
                error bar style={gray, line width=0.7pt},
        ] coordinates {%
            (2.34e+13,8.0e-3) -= (0,7.9999e-3)
            (1.42e+13,5.6e-2) -= (0,0.0559999)
            (4.93e+12,1.3e-2) -= (0,0.0129999)
            (2.90e+12,2.5e-2) -= (0,0.0249999)
            (1.72e+12,2.7e-2) -= (0,0.0269999)
        };
    \end{loglogaxis}
\end{tikzpicture}

    \caption{}
    \label{app:fig1}
\end{figure}

\newpage

\begin{figure}[h]
    \centering
    \begin{tikzpicture}[%
    scale=1.5,
    every mark/.append style={%
        scale=1.0,
    }
]
    \begin{loglogaxis}[%
        % title=Spektrale Energieverteilung des Quasars 3C138,
        xlabel=$\nu$/\si{\hertz},
        ylabel=$\mathrm{F}_{\nu}\,\nu$/\si{\jansky\hertz},
        grid=major,
        % xmin=1e7, xmax=1e18,
        % ymin=1e-7, ymax=1e2,
        % max space between ticks=20
    ]
        \addplot+ [%
            only marks,
            % error bars/.cd,
            %     y dir=both,
            %     y explicit,
            %     error bar style={black},
        ] table [%
            x=frequency,
            y expr={\thisrow{flux_density}*\thisrow{frequency}},
            % y error=uncertainty,
            col sep=comma
        ] {sed-3c138.csv};
    \end{loglogaxis}
\end{tikzpicture}

    \caption{}
    \label{app:fig2}
\end{figure}

\newpage



\end{document}
