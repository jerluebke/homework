\documentclass[11pt,a4paper]{scrartcl}
\usepackage[top=2cm,bottom=5.5cm,left=2cm,right=2cm]{geometry}
\usepackage{fontspec}
\usepackage{polyglossia}
    \setdefaultlanguage{german}
\usepackage{fancyhdr}
\usepackage{csquotes}
\usepackage{enumitem}
\usepackage{mathtools}
\usepackage{amssymb}
\usepackage{amsfonts}
\usepackage{booktabs}
\usepackage{siunitx}
    \sisetup{range-units=brackets}
    \DeclareSIUnit{\jansky}{Jy}
\usepackage{physics}
\usepackage[%
    labelformat=simple,
    labelsep=none,
    textformat=none,
    font={small,sc}
]{caption}
\usepackage{graphicx}
    \graphicspath{img}
\usepackage{pgfplots}
    \pgfplotsset{%
        table/search path={data},
        label style={font=\tiny},
        tick label style={font=\tiny}
    }
\usepackage{todonotes}
\usepackage[%
    colorlinks=true, linkcolor=blue,
    % hidelinks
]{hyperref}

\newcommand{\head}[1]{\multicolumn{1}{c}{#1}}
\newcommand*{\figref}[1]{(siehe Abb.~\ref{#1})}
\newcommand{\zB}{z.\,B.}

\newcommand{\course}{\textbf{Einführung in die Astrophysik}}
\newcommand{\hwnumber}{1}
\newcommand{\nameA}{Jeremiah Lübke}
\newcommand{\nameB}{Andreas Menzel}
\newcommand{\matnumA}{108015230366}
\newcommand{\matnumB}{108015226385}
\newcommand{\groupnum}{Übungsgruppe 3}


\pagestyle{fancyplain}
\headheight 7\baselineskip
\lhead{%
    \course \\
    \vspace*{3\baselineskip}
    \nameA, \matnumA \\
    \nameB, \matnumB
}

\chead{%
    \textbf{\Large Hausaufgabenblatt \hwnumber} \\
    \vspace*{2\baselineskip}
}

\rhead{%
    \today \\
    \vspace*{4\baselineskip}
    \groupnum
}

\cfoot{\small\thepage}
\headsep 1.5em


\newcommand{\Fnu}{\ensuremath{\mathrm{F}_{\nu}}}


\begin{document}

\section*{Aufgabe 3}

\begin{enumerate}[label=\textbf{\large(\alph*)}]

\item
Einige Messwerte von Quasar 3C138

\begin{table}[h]
\centering
\begin{tabular}{rrrc}
    \toprule
    \head{E} & \head{$\lambda$} & \head{$\nu$} & \head{Frequenzbereich} \\
    \midrule
    \SIrange{0.1}{2.4}{\kilo\electronvolt} & \SIrange{12.4}{0.5}{\nano\metre} &
    \SIrange{24.2}{580.3}{\peta\hertz} & Röntgenstrahlen \\
    2.8 \si{\electronvolt} & 442.0 \si{\nano\metre} & 678.3 \si{\giga\hertz} &
    Blaue Licht \\
    0.6 \si{\electronvolt} & 2.2 \si{\micro\metre} & 136.3 \si{\giga\hertz} &
    nahes Infrarot \\
    44.3 \si{\micro\electronvolt} & 28.0 \si{\milli\metre} & 10.7 \si{\giga\hertz}
    &  Millimeterwellen \\
    5.8 \si{\micro\electronvolt} & 214.1 \si{\milli\metre} & 1.4 \si{\giga\hertz}
    & Millimeterwellen \\
    306.0 \si{\nano\electronvolt} & 4.1 \si{\metre} & 74.0 \si{\mega\hertz} &
    Ultrakurzwelle \\
    \bottomrule
\end{tabular}
\label{tab:tab1}
\end{table}

wobei $E = h\,\nu$ und $\nu = \frac{c}{\lambda}$.

\vspace*{\baselineskip}

\item
Spektrale Energieverteilung des Quasars 3C138: \Fnu gegen
$\nu$ \figref{app:fig1}. \\
\textit{\small Die geplotteten Daten stammen von der
    \href{https://ned.ipac.caltech.edu/}{NASA Extragalactic Database}.}

    \begin{itemize}
        \item Der Plot suggeriert einen (auf der doppelt logarithmischen
            Skala) fast linearen Verlauf für Frequenzen $>10^9$ \si{\hertz}
        \item Bei $10^8$ \si{\hertz} scheint der Verlauf ein Maximum zu
            erreichen; an dieser Stelle könnten weitere Daten für kleinere
            Frequenzen interessant sein
    \end{itemize}

\vspace*{\baselineskip}

\item
Spektrale Energieverteilung des Quasars 3C138: $\Fnu\,\nu$
gegen $\nu$ \figref{app:fig2}. \\
\textit{\small Die geplotteten Daten basieren auf dem in Teil (b)
    verwendeten Datensatz. Die y-Fehler wurden gemäß
    $\Delta\left(\Fnu\,\nu\right) = \abs{\pdv{(\Fnu\,\nu)}{\Fnu}}\,\Delta\Fnu =
    \nu\,\Delta\Fnu$ berechnet.}

    \begin{itemize}
        \item Die in Teil (b) betrachtete Strahlungsflussdichte
            \Fnu beschreibt die Strahlungsleistung pro Fläche
            pro Frequenz für \emph{eine spezifische} Frequenz
        \item Das hier betrachtete Produkt $\Fnu\,\nu$
            beschreibt hingegen die Strahlungsleitung pro Fläche

        \todo{Worauf will ich hinaus? Was beschreiben beide Größe in
        Hinblick auf die abgestralte Energie?}

        \item Der Verlauf nimmt eine andere Form an, deren Maximum nun
            deutlich erkennbar bei $\sim 10^{13}$ \si{\hertz} liegt
    \end{itemize}

\end{enumerate}

\newpage

\chead{%
    \textbf{\Large Homework \hwnumber~--~Appendix} \\
    \vspace*{2\baselineskip}
}
\cfoot{}

\begin{figure}[h]
    \centering
    \newcommand{\eradius}{3pt}
\newcommand{\blineouterradius}{7pt}
\newcommand{\blineinnerradius}{1pt}

\newcommand{\bline}[1]{%
    \draw[thick,blue!80] (#1) circle (\blineouterradius);
    \fill[blue!80] (#1) circle (\blineinnerradius);
}

\newcommand{\radiationcone}{%
    (e) -- (10, 5.5) to [out=14.03624, in=90] (13.25, 4) to [out=-90,
    in=-14.03624] (10, 2.5) -- (e);
}

\begin{tikzpicture}[scale=1.2,>=stealth]

% gyration center at (4, -4), radius 8

% electron coordinates
\coordinate (e) at (4, 4);

% bline coordinates
\coordinate (B1) at (2, 2);
\coordinate (B2) at (2, 6);
\coordinate (B3) at (6, 6);
\coordinate (B4) at (6, 2);

% dashed radiation cone border
\draw [semithick,dashed,gray] (e) -- (13, 6.25);
\draw [semithick,dashed,gray] (e) -- (13, 1.75);
% radiation cone filling
% \fill [orange!30] \radiationcone;
% radiation direction
\draw [dashed,thick,gray,->] (e) -- (14, 4);
% electron trajectory
\draw [thick,->] (0.93853,3.39104) arc (112.5:45:8);
% radiation cone outline
\draw [very thick,orange!80] \radiationcone;

% draw electron and blines
\fill [blue!80!black] (e) circle (\eradius);
\foreach \bcoord in {B1, B2, B3, B4}
    \bline{\bcoord};

% opening angle of cone
\draw [semithick,|<-] (8.85071,5.21268) arc (14:4.5:5);
\draw [semithick,|<-] (8.85071,2.78732) arc (-14:0.5:5);
% Efield oscillation
\draw [thick,red!80!black,|->] (10.5,3.985) -- (10.5,5.59);

% labels
\node [right=1] at (B2) {$\vec{B}$};
\node [above] at (e) {$e^-$};
\node [right] at (10.5, 4.8) {$\vec{E}$};
% \node [right] at (10.6, 6.2) {$\Theta$};
\node  at (9, 4.2) {$\Theta$};

\end{tikzpicture}

    \caption{}
    \label{app:fig1}
\end{figure}


\end{document}
