\documentclass[11pt,a4paper]{scrartcl}
\usepackage[top=2cm,bottom=5.5cm,left=2cm,right=2cm]{geometry}
\usepackage{fontspec}
\usepackage{polyglossia}
    \setdefaultlanguage{german}
\usepackage{csquotes}
\usepackage{enumitem}
\usepackage{mathtools}
\usepackage{amssymb}
\usepackage{amsfonts}
\usepackage{booktabs}
\usepackage{siunitx}
    \sisetup{range-units=brackets}
    \DeclareSIUnit{\jansky}{Jy}
\usepackage[%
    labelformat=empty,
    font={small,sc}
]{caption}
\usepackage{graphicx}
    \graphicspath{img}
\usepackage{pgfplots}
    \pgfplotsset{%
        table/search path={data},
        label style={font=\tiny},
        tick label style={font=\tiny}
    }
\usepackage{fancyhdr}
\usepackage[%
    % colorlinks=true, linkcolor=blue,
    hidelinks
]{hyperref}

\newcommand{\head}[1]{\multicolumn{1}{c}{#1}}
\newcommand{\zB}{z.\,B.}

\newcommand{\course}{\textbf{Einführung in die Astrophysik}}
\newcommand{\hwnumber}{1}
\newcommand{\nameA}{Jeremiah Lübke}
\newcommand{\nameB}{Andreas Menzel}
\newcommand{\matnumA}{108015230366}
\newcommand{\matnumB}{108015xxxxxx}
\newcommand{\groupnum}{Übungsgruppe 3}


\pagestyle{fancyplain}
\headheight 7\baselineskip
\lhead{%
    \course \\
    \vspace*{3\baselineskip}
    \nameA, \matnumA \\
    \nameB, \matnumB
}

\chead{%
    \textbf{\Large Hausaufgabenblatt \hwnumber} \\
    \vspace*{2\baselineskip}
}

\rhead{%
    \today \\
    \vspace*{4\baselineskip}
    \groupnum
}

\cfoot{\small\thepage}
\headsep 1.5em


\begin{document}

\section*{Aufgabe 3}

\begin{enumerate}[label=\textbf{\large(\alph*)}]

\item
Einige Messwerte von Quasar 3C138

\begin{table}[h!]
\centering
\begin{tabular}{rrrc}
    \toprule
    \head{E} & \head{$\lambda$} & \head{$\nu$} & \head{Frequenzbereich} \\
    \midrule
    \SIrange{0.1}{2.4}{\kilo\electronvolt} & \SIrange{12.4}{0.5}{\nano\metre} &
    \SIrange{24.2}{580.3}{\peta\hertz} & Röntgenstrahlen \\
    2.8 \si{\electronvolt} & 442.0 \si{\nano\metre} & 678.3 \si{\giga\hertz} &
    Blaue Licht \\
    0.6 \si{\electronvolt} & 2.2 \si{\micro\metre} & 136.3 \si{\giga\hertz} &
    nahes Infrarot \\
    44.3 \si{\micro\electronvolt} & 28.0 \si{\milli\metre} & 10.7 \si{\giga\hertz}
    &  Millimeterwellen \\
    5.8 \si{\micro\electronvolt} & 214.1 \si{\milli\metre} & 1.4 \si{\giga\hertz}
    & Millimeterwellen \\
    306.0 \si{\nano\electronvolt} & 4.1 \si{\metre} & 74.0 \si{\mega\hertz} &
    Ultrakurzwelle \\
    \bottomrule
\end{tabular}
\label{tab:tab1}
\end{table}
\vspace*{\baselineskip}


\item
Spektrale Energieverteilung des Quasars 3C138: $\mathrm{F}_{\nu}$ gegen
$\nu$

\begin{figure}[h!]
    \centering
    \begin{tikzpicture}[%
    scale=1.5,
    every mark/.append style={%
        scale=1.0,
    }
]
    \begin{loglogaxis}[%
        % title=Spektrale Energieverteilung des Quasars 3C138,
        xlabel=$\nu$/\si{\hertz},
        ylabel=$\mathrm{F}_{\nu}$/\si{\jansky},
        grid=major,
        xmin=1e7, xmax=1e18,
        ymin=1e-7, ymax=1e2,
        % max space between ticks=20
    ]
        \addplot+ [%
            only marks,
            error bars/.cd,
                y dir=both,
                y explicit,
                error bar style={black},
        ] table [%
            x=frequency,
            y=flux_density,
            y error=uncertainty,
            col sep=comma
        ] {sed-3c138.csv};

        % correcting truncated error bars
        \addplot+ [%
            blue,
            mark=*,
            mark options={fill=blue!80!black},
            only marks,
            error bars/.cd,
                y dir=minus,
                y explicit,
                error bar style={black},
        ] coordinates {%
            (2.34e+13,8.0e-3) -= (0,7.9999e-3)
            (1.42e+13,5.6e-2) -= (0,0.0559999)
            (4.93e+12,1.3e-2) -= (0,0.0129999)
            (2.90e+12,2.5e-2) -= (0,0.0249999)
            (1.72e+12,2.7e-2) -= (0,0.0269999)
        };
    \end{loglogaxis}
\end{tikzpicture}

    \caption{Quelle der Daten: \href{https://ned.ipac.caltech.edu/}{NED}}
\end{figure}
\vspace*{\baselineskip}
\newpage

\item
Spektrale Energieverteilung des Quasars 3C138: $\mathrm{F}_{\nu}\,\nu$
gegen $\nu$

\begin{figure}[h!]
    \centering
    \begin{tikzpicture}[%
    rotate=270,
    scale=1.5,
    every mark/.append style={%
        scale=1.0,
    }
]
    \begin{loglogaxis}[%
        width=15cm,
        height=10.5cm,
        % axis y line*=right,
        title=Spektrum des Quasars 3C138: $\nu\,\Fnu$ gegen
        $\nu$,
        xlabel=$\nu$/\si{\hertz},
        ylabel=$\nu\,\Fnu$/\si{\jansky\hertz},
        grid=major,
        xmin=1e7, xmax=1e18,
        ymin=1e8, ymax=3e12,
        % max space between ticks=20
    ]
        \addplot+ [%
            only marks,
            error bars/.cd,
                y dir=both,
                y explicit,
                error bar style={gray,line width=0.7pt},
        ] table [%
            x=frequency,
            y expr={\thisrow{flux_density}*\thisrow{frequency}},
            y error expr={\thisrow{frequency}*\thisrow{uncertainty}},
            % y error=uncertainty,
            col sep=comma
        ] {sed-3c138.csv};

        \addplot+ [%
            blue,
            mark=*,
            mark options={fill=blue!80!black},
            only marks,
            error bars/.cd,
                y dir=minus,
                y explicit,
                error bar style={gray,line width=0.7pt},
        ] coordinates {%
            (2.34e+13,1.872e+11) -= (0,1.871e+11)
            (1.42e+13,7.952e+11) -= (0,7.951e+11)
            (4.93e+12,6.409e+10) -= (0,6.399e+10)
            (2.90e+12,7.25e+10) -= (0,7.24e+10)
            (1.72e+12,4.644e+10) -= (0,4.634e+10)
        };
    \end{loglogaxis}
\end{tikzpicture}

\end{figure}


\end{enumerate}

\end{document}
