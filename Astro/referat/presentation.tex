\documentclass[mathserif]{beamer}
\usepackage{fontspec}
\usepackage{amsmath}
\usepackage{bbold}
\usepackage{physics}
\usepackage{graphicx}
    \graphicspath{img}
\usepackage{tcolorbox}


\newcommand{\partt}{\partial_{t}}
\newcommand{\Div}{\nabla\cdot}
\newcommand{\Grad}{\nabla}
\newcommand{\Curl}{\nabla\times}
\newcommand{\Lap}{\nabla^{2}}

\newcommand{\uvec}{\underline{u}}
\newcommand{\jvec}{\underline{j}}
\newcommand{\Bvec}{\underline{B}}
\newcommand{\Evec}{\underline{E}}
\newcommand{\alphatensor}{\underline{\underline{\alpha}}}

\newcommand{\avg}[1]{\langle#1\rangle}
\newcommand{\Ren}[1]{\mathcal{R}_{#1}}


\title{Large-Scale Galactic Magnetic Fields}
\subject{Introduction to Astrophysics - Minitalk}
\author{Jeremiah Lübke}
\date{Ruhr-Universität Bochum, July 2019}


\begin{document}

\setlength{\abovedisplayskip}{6pt}
\setlength{\belowdisplayskip}{6pt}


\frame{\titlepage}

\begin{frame}
    \frametitle{Table of Contents}
    \tableofcontents
\end{frame}


\section{MHD}
\begin{frame}
    \frametitle{Magnetohydrodynamics (MHD)}
    \begin{columns}[T]
        \begin{column}[T]{.5\textwidth}
            \begin{tcolorbox}
            \begin{center}
                \textbf{Plasma}
            \end{center}
            continuity: \[\partt\rho+\Div(\rho\uvec)=0\]
            momentum:
            \begin{align*}
                \partt(\rho\uvec)+\Div(\rho\uvec\uvec) \\
                =\jvec\times\Bvec-\Grad{p}
            \end{align*}
            adiabatic closure: \[p=2nkT\]
            \end{tcolorbox}
        \end{column}

        \begin{column}[T]{.5\textwidth}
            \begin{tcolorbox}
            \begin{center}
                \textbf{Magnetic Field}
            \end{center}
            Ohm's Law: \[\Evec+\uvec\times\Bvec=\eta\jvec\]
            Faraday's Law:
            \begin{align*}
                \partt\Bvec&=-\Curl\Evec \\
                &=\Curl(\uvec\times\Bvec)+\eta\Lap\Bvec
            \end{align*}
            \end{tcolorbox}
        \end{column}
    \end{columns}
    \medskip
    \begin{itemize}
        \item Plasma \emph{induces} Magnetic Field
        \item Magnetic Field \emph{accelerates} Plasma
    \end{itemize}
\end{frame}


\section{Dynamos}
\begin{frame}
    \frametitle{Dynamos}
    \begin{itemize}
        \item
            Consider the equation governing the magnetic field:
            \begin{equation}
                \partt\Bvec=\Curl(\uvec\times\Bvec)+\eta\Lap\Bvec
                \label{eq:b}
            \end{equation}
    \end{itemize}

    \noindent\rule{\textwidth}{.4pt}

    \begin{itemize}
        \only<1-4>{%
        \item<2-4>
            Splitting up the involved quantities into regular average and random
            fluctuations:
            \begin{equation*}
                % \Bvec=\avg{\Bvec}+\Bvec',\quad\uvec=\avg{\uvec}+\uvec'
                X=\avg{X}+X',\quad\avg{\avg{X}}=\avg{X},\quad\avg{X'}=0
            \end{equation*}

        \item<3-4>
            And working out \eqref{eq:b}:
            \begin{equation*}
                \implies \partt\avg{\Bvec}=\Curl(\avg{\uvec}\times\avg{\Bvec})
                +\eta\Lap\avg{\Bvec}+\Curl\avg{\uvec'\times\Bvec'}
            \end{equation*}

        \item<4-4>
            Investigating the last term more closely:
            \begin{equation*}
                \avg{\uvec'\times\Bvec'}\approx\alphatensor\avg{\Bvec}
            \end{equation*}
            with the \emph{helicity tensor} $\alphatensor=\alpha\mathbb{1}$
            (for the isotropic case)
        }

        \only<5>{%
        \item
            Working out \eqref{eq:b} in spherical coordinates and with a lot of
            galaxy-related assumptions, one finally arrives at:
            \begin{subequations}
            \begin{align}
                \partt{B_r}&=-\Ren{\alpha}\,\partial_{z}(\alpha{B_{\phi}})
                +\partial_{zz}B_r \\
                \partt{B_{\phi}}&=\Ren{\Omega}\,B_{r}\,r\,\partial_{r}\Omega
                +\partial_{zz}B_{\phi}
            \end{align}
            \end{subequations}
        }
    \end{itemize}
\end{frame}


\section{Spatial Structures}
\begin{frame}
    \frametitle{Spatial Structures}

\end{frame}


\section{Observing Magnetic Fields}
\begin{frame}
    \frametitle{Observing Magnetic Fields}

\end{frame}


\section{References}
\begin{frame}
    \frametitle<presentation>{References}

\end{frame}


\end{document}

% vim: set ff=unix tw=79 sw=4 ts=4 et ic ai :
