\documentclass[mathserif]{beamer}
\usepackage{fontspec}
\usepackage{textcomp}
\usepackage{amsmath}
\usepackage{bbold}
\usepackage{physics}
\usepackage{graphicx}
    \graphicspath{{./img/}}
\usepackage{subcaption}
\usepackage{tcolorbox}


\usetheme{Madrid}
\usecolortheme{beaver}


\newcommand{\partt}{\partial_{t}}
\newcommand{\Div}{\nabla\cdot}
\newcommand{\Grad}{\nabla}
\newcommand{\Curl}{\nabla\times}
\newcommand{\Lap}{\nabla^{2}}

\newcommand{\uvec}{\underline{u}}
\newcommand{\jvec}{\underline{j}}
\newcommand{\Bvec}{\underline{B}}
\newcommand{\Evec}{\underline{E}}
\newcommand{\alphatensor}{\underline{\underline{\alpha}}}

\newcommand{\avg}[1]{\langle#1\rangle}
\newcommand{\Ren}[1]{\mathcal{R}_{#1}}


\title{Large-Scale Galactic Magnetic Fields}
\subject{Introduction to Astrophysics - Minitalk}
\author{Jeremiah Lübke}
\date{Ruhr-Universität Bochum, July 2019}


\begin{document}

\setlength{\abovedisplayskip}{6pt}
\setlength{\belowdisplayskip}{6pt}


\frame{\titlepage}

\begin{frame}
    \frametitle{Table of Contents}
    \tableofcontents
\end{frame}


% \setcounter{framenumber}{1}
\addtocounter{framenumber}{-2}


\section{MHD}
\begin{frame}
    \frametitle{Magnetohydrodynamics (MHD)}
    \begin{columns}[T]
        \begin{column}[T]{.5\textwidth}
            \begin{tcolorbox}
            \begin{center}
                \textbf{Plasma}
            \end{center}
            continuity: \[\partt\rho+\Div(\rho\uvec)=0\]
            momentum:
            \begin{align*}
                \partt(\rho\uvec)+\Div(\rho\uvec\uvec) \\
                =\jvec\times\Bvec-\Grad{p}
            \end{align*}
            adiabatic closure: \[p=2nkT\]
            \end{tcolorbox}
        \end{column}

        \begin{column}[T]{.5\textwidth}
            \begin{tcolorbox}
            \begin{center}
                \textbf{Magnetic Field}
            \end{center}
            Ohm's Law: \[\Evec+\uvec\times\Bvec=\eta\jvec\]
            Faraday's Law:
            \begin{align*}
                \partt\Bvec&=-\Curl\Evec \\
                &=\Curl(\uvec\times\Bvec)+\eta\Lap\Bvec
            \end{align*}
            \end{tcolorbox}
        \end{column}
    \end{columns}
    \medskip
    \begin{itemize}
        \item Plasma \emph{induces} Magnetic Field
        \item Magnetic Field \emph{accelerates} Plasma
    \end{itemize}
\end{frame}


\section{Dynamos}
\begin{frame}
    \frametitle{Dynamos}
    \begin{itemize}
        \item
            Consider the equation governing the magnetic field:
            \begin{equation}
                \partt\Bvec=\Curl(\uvec\times\Bvec)+\eta\Lap\Bvec
                \label{eq:b}
            \end{equation}
    \end{itemize}

    \noindent\rule{\textwidth}{.4pt}

    \begin{itemize}
        \only<1-3>{%
        \item<2-3>
            Splitting up the involved quantities into regular average and random
            fluctuations:
            \begin{equation*}
                % \Bvec=\avg{\Bvec}+\Bvec',\quad\uvec=\avg{\uvec}+\uvec'
                X=\avg{X}+X'\qq{with}\avg{\avg{X}}=\avg{X},\quad\avg{X'}=0
            \end{equation*}

        \item<3-3>
            And working out \eqref{eq:b}:
            \begin{equation*}
                \implies \partt\avg{\Bvec}=\Curl(\avg{\uvec}\times\avg{\Bvec})
                +\eta\Lap\avg{\Bvec}+
                \underbrace{%
                    \Curl\avg{\uvec'\times\Bvec'}
                }_{=\mathcal{E}}
            \end{equation*}
            \begin{itemize}
            \begin{tcolorbox}
                \item
                    The large-scale average is influenced by the
                    small-scale fluctuations!
            \end{tcolorbox}
            \end{itemize}
        }

        \only<4-6>{%
        \item
            Working out \eqref{eq:b} in spherical coordinates, with
            $\uvec=r\,\Omega\,\underline{e}_{\phi}$ and a lot of
            galaxy-related assumptions, one finally arrives at:

            \only<4>{%
            \begin{equation}
                \mathcal{E}=0\implies
                \left\{\begin{aligned}
                    \partt{B_r}&=\partial_{zz}B_r \\
                    \partt{B_{\phi}}&=\Ren{\Omega}\,B_{r}\,r\,\partial_{r}\Omega
                    +\partial_{zz}B_{\phi}
                \end{aligned}\right.
            \end{equation}

            \begin{itemize}
                \item
                    The magnetic field dilutes and the dynamo dies \\
                    (\textrightarrow~Antidynamo theorem)
            \end{itemize}

            \vspace{3.6\baselineskip}
            }

            \only<5-6>{%
            \setcounter{equation}{2}
            \begin{equation}
                \mathcal{E}\approx\alpha\avg{\Bvec}\implies
                \left\{\begin{aligned}
                    \partt{B_r}&=-\Ren{\alpha}\,\partial_{z}(\alpha{B_{\phi}})
                    +\partial_{zz}B_r \\
                    \partt{B_{\phi}}&=\Ren{\Omega}\,B_{r}\,r\,\partial_{r}\Omega
                    +\partial_{zz}B_{\phi}
                \end{aligned}\right.
            \end{equation}

            \begin{itemize}
                \item
                    small-scale turbulence ($\alpha\ne0$) sustains the magnetic
                    field \\ (\textrightarrow~$\alpha$- and $\Omega$-effect)
            \end{itemize}

            \only<5>{\vspace{3.6\baselineskip}}
        \only<6>{%
            \vspace{\baselineskip}
            \begin{tcolorbox}
            \item
                Order out of Chaos
            \end{tcolorbox}
        }
            }
        }
    \end{itemize}
\end{frame}


\section{Spatial Structures of Dynamo Solutions}
\begin{frame}
    \frametitle{Spatial Structures of Dynamo Solutions}

    \begin{figure}
        \centering
        \begin{subfigure}[b]{.35\textwidth}
            \includegraphics[width=\textwidth]{mode-0}
        \end{subfigure}
        \begin{subfigure}[b]{.35\textwidth}
            \includegraphics[width=\textwidth]{mode-1}
        \end{subfigure}
        \begin{subfigure}[b]{.35\textwidth}
            \includegraphics[width=\textwidth]{mode-2}
        \end{subfigure}
    \end{figure}

\end{frame}


\section{Observing Magnetic Fields}
\begin{frame}
    \frametitle{Observing Magnetic Fields}

    \begin{columns}[T]
        \begin{column}[T]{.5\textwidth}
            \begin{itemize}
                \item
                    Synchrotron radiation: polarization $\perp$ field
                    orientation
                \item
                    Polarization is changed when passing through magnetized
                    plasma (\emph{Faraday Rotation}):
                    \[\beta=\mathrm{RM}(n_e, B_{\parallel})\,\lambda^2\]
            \end{itemize}
            \includegraphics[width=\textwidth]{faraday-rotation}
        \end{column}

        \begin{column}[T]{.5\textwidth}
            \pause
            \includegraphics[width=\textwidth]{m51}
        \end{column}
    \end{columns}

\end{frame}


\appendix
\newcounter{finalframe}
\setcounter{finalframe}{\value{framenumber}}


\section{References}
\begin{frame}
    \frametitle<presentation>{References}

    \begin{thebibliography}{10}
    \beamertemplatearticlebibitems
    \bibitem{Dorch2007}
        S.~B.~F.~Dorch.
        \newblock {\em Magnetohydrodynamics}.
        \newblock {\em Scholarpedia}, revision \#186287.
    \beamertemplatearticlebibitems
    \bibitem{Beck2007}
        R.~Beck.
        \newblock {\em Galactic Magnetic Fields}.
        \newblock {\em Scholarpedia}, revision \#185129.
    \beamertemplatebookbibitems
    \bibitem{GalMagFields}
        U.~Klein, A.~Fletcher.
        \newblock {\em Galactic and Intergalactic Magnetic Fields}
        \newblock Springer Praxis Books, 2015.
    \beamertemplateonlinebibitems
    \bibitem{Beck2014}
        R.~Beck.
        \newblock {\em Magnetic Fields in Spiral Galaxies, Part I: Disks}.
        \newblock \url{https://www.mpifr-bonn.mpg.de/1281929/Beck_Galaxies.pdf}
    \end{thebibliography}

\end{frame}


\setcounter{framenumber}{\value{finalframe}}


\end{document}

% vim: set ff=unix tw=79 sw=4 ts=4 et ic ai :
