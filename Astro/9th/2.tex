\documentclass[11pt,a4paper]{scrartcl}
\usepackage[top=2cm,bottom=5.5cm,left=2cm,right=2cm]{geometry}
\usepackage{fontspec}
\usepackage{polyglossia}
    \setdefaultlanguage{english}
\usepackage{lmodern}
\usepackage{fixcmex}
\usepackage{fancyhdr}
\usepackage{csquotes}
\usepackage{enumitem}
\usepackage{mathtools}
\usepackage{amssymb}
\usepackage{amsfonts}
\usepackage{siunitx}
    \sisetup{range-units=brackets}
    \DeclareSIUnit{\year}{a}
    \DeclareSIUnit{\jansky}{Jy}
    \DeclareSIUnit{\parsec}{pc}
    \DeclareSIUnit{\lightyear}{ly}
    \DeclareSIUnit{\Rsol}{\ensuremath{R_{\astrosun}}}
    \DeclareSIUnit{\Lsol}{\ensuremath{L_{\astrosun}}}
    \DeclareSIUnit{\Msol}{\ensuremath{M_{\astrosun}}}
\usepackage{physics}
\usepackage{textcomp}
\usepackage{gensymb}
\usepackage{wasysym}
\usepackage[version=4]{mhchem}
\usepackage{array}
\usepackage{booktabs}
\usepackage[%
    labelformat=simple,
    labelsep=none,
    textformat=empty,
    font={small,sc}
]{caption}
\usepackage{graphicx}
    \graphicspath{img}
\usepackage{tikz}
    \usetikzlibrary{calc,external}
    \tikzexternalize[prefix=extern/]
    \tikzexternaldisable
\usepackage{pgfplots}
    \pgfplotsset{%
        compat=1.16,
        table/search path={data},
        label style={font=\tiny},
        tick label style={font=\tiny}
    }
\usepackage{todonotes}
\usepackage[%
    % colorlinks=true, linkcolor=blue,
    hidelinks
]{hyperref}


\newcommand{\tablehead}[1]{\multicolumn{1}{c}{#1}}
\newcommand*{\figref}[1]{(see fig.~\ref{#1})}
\newcommand{\eg}{e.\,g.}
\newcommand{\ie}{i.\,e.}

\newcommand{\course}{\textbf{Introduction to Astrophysics}}
\newcommand{\hwnumber}{9}
\newcommand{\nameA}{Jeremiah Lübke}
\newcommand{\nameB}{Andreas Menzel}
\newcommand{\matnumA}{108015230366}
\newcommand{\matnumB}{108015226385}
\newcommand{\groupnum}{Exercise Group 3}


\pagestyle{fancyplain}

\headheight 7\baselineskip
\lhead{%
    \course \\
    \vspace*{3\baselineskip}
    \nameA, \matnumA \\
    \nameB, \matnumB
}

\chead{%
    \textbf{\Large Homework \hwnumber} \\
    \vspace*{2\baselineskip}
}

\rhead{%
    \today \\
    \vspace*{4\baselineskip}
    \groupnum
}

\cfoot{\small\thepage}
\headsep 1.5em


\newcommand{\mH}{\ensuremath{m_{\mathrm{H}}}}
\newcommand{\mP}{\ensuremath{m_{\mathrm{P}}}}
\newcommand{\nop}{\text{Number of particles}}


\begin{document}

\section*{Task 2}

\begin{enumerate}[label=\textbf{\large(\alph*)}, itemsep=2\baselineskip]

\item
    In order to find the cloud's diameter consider the column density:
    \begin{gather*}
        N=\int{n}\dd{z}\overset{\text{here}}{=}n\int\limits_{0}^{d}\dd{z}=nD \\
        \iff{d}=\frac{N}{n}=\SI{1.5e16}{\metre}=\SI{.49}{\parsec}
    \end{gather*}

\item
    To find the total number of particles:
    \begin{gather*}
        \nop=nV\qc{V}=\frac{\pi{d^3}}{6} \\
        \implies\nop=\frac{\pi{d^3}n}{6}=\num{1.77e56}
    \end{gather*}

\item
    Therewith one can estimate the cloud's total mass:
    \begin{equation*}
        M=\nop\times\mH\overset{\mH\approx\mP}{=}\SI{2.96e29}{\kilogram}=\SI{.15}{\Msol}
    \end{equation*}

\item
    The Luminosity of 21cm photons with given rate of photons $Q$:
    \begin{gather*}
        L=Q\times{h}\nu=\SI{1.47e18}{\watt}=\SI{3.8e-9}{\Lsol}
    \end{gather*}
    where $\nu=\frac{c}{\lambda}=\SI{1.4e9}{\hertz}$.

\item
    And finally the flux of 21cm photons as seen on earth:
    \begin{gather*}
        F_{\nu}=\frac{L}{A\nu}=\frac{L}{\pi{D^2}\nu}=\SI{3.48e-29}{\watt\per\metre\squared\per\hertz}
        =\SI{3.48e-3}{\jansky}
    \end{gather*}
    where $D=\SI{100}{\parsec}\approx\SI{3.1e18}{\metre}$.

\end{enumerate}


\end{document}
