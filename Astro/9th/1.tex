\documentclass[11pt,a4paper]{scrartcl}
\usepackage[top=2cm,bottom=5.5cm,left=2cm,right=2cm]{geometry}
\usepackage{fontspec}
\usepackage{polyglossia}
    \setdefaultlanguage{english}
\usepackage{lmodern}
\usepackage{fixcmex}
\usepackage{fancyhdr}
\usepackage{csquotes}
\usepackage{enumitem}
\usepackage{mathtools}
\usepackage{amssymb}
\usepackage{amsfonts}
\usepackage{siunitx}
    \sisetup{range-units=brackets}
    \DeclareSIUnit{\year}{a}
    \DeclareSIUnit{\jansky}{Jy}
    \DeclareSIUnit{\parsec}{pc}
    \DeclareSIUnit{\lightyear}{ly}
    \DeclareSIUnit{\Rsol}{\ensuremath{R_{\astrosun}}}
    \DeclareSIUnit{\Lsol}{\ensuremath{L_{\astrosun}}}
    \DeclareSIUnit{\Msol}{\ensuremath{M_{\astrosun}}}
\usepackage{physics}
\usepackage{textcomp}
\usepackage{gensymb}
\usepackage{wasysym}
\usepackage[version=4]{mhchem}
\usepackage{array}
\usepackage{booktabs}
\usepackage[%
    labelformat=simple,
    labelsep=none,
    textformat=empty,
    font={small,sc}
]{caption}
\usepackage{graphicx}
    \graphicspath{img}
\usepackage{tikz}
    \usetikzlibrary{calc,external}
    \tikzexternalize[prefix=extern/]
    \tikzexternaldisable
\usepackage{pgfplots}
    \pgfplotsset{%
        compat=1.16,
        table/search path={data},
        label style={font=\tiny},
        tick label style={font=\tiny}
    }
\usepackage{todonotes}
\usepackage[%
    % colorlinks=true, linkcolor=blue,
    hidelinks
]{hyperref}


\newcommand{\tablehead}[1]{\multicolumn{1}{c}{#1}}
\newcommand*{\figref}[1]{(see fig.~\ref{#1})}
\newcommand{\eg}{e.\,g.}
\newcommand{\ie}{i.\,e.}

\newcommand{\course}{\textbf{Introduction to Astrophysics}}
\newcommand{\hwnumber}{9}
\newcommand{\nameA}{Jeremiah Lübke}
\newcommand{\nameB}{Andreas Menzel}
\newcommand{\matnumA}{108015230366}
\newcommand{\matnumB}{108015226385}
\newcommand{\groupnum}{Exercise Group 3}


\pagestyle{fancyplain}

\headheight 7\baselineskip
\lhead{%
    \course \\
    \vspace*{3\baselineskip}
    \nameA, \matnumA \\
    \nameB, \matnumB
}

\chead{%
    \textbf{\Large Homework \hwnumber} \\
    \vspace*{2\baselineskip}
}

\rhead{%
    \today \\
    \vspace*{4\baselineskip}
    \groupnum
}

\cfoot{\small\thepage}
\headsep 1.5em


\newcommand{\trec}{\ensuremath{\tau_{\mathrm{rec}}}}
\newcommand{\rs}{\ensuremath{r_{\mathrm{S}}}}
\newcommand{\Rsol}{\ensuremath{R_{\astrosun}}}
\newcommand{\Lsol}{\ensuremath{L_{\astrosun}}}


\begin{document}

\section*{Task 1}

\begin{enumerate}[label=\textbf{\large(\alph*)}, itemsep=2\baselineskip]

\item
    The volume of the Strömgren sphere - \ie~the area of ionized gas around a
    star - can be estimated via the product of the star's photon rate $Q$ times
    the recombination time $\trec=1/n\alpha$ divided by the particle density
    $n$.
    In case of the Sun, with the given (estimated) values:
    \begin{gather*}
        V=\frac{4\pi}{3}\rs^3=\frac{Q}{n^2\alpha} \\
        \iff\rs=\sqrt[3]{\frac{3Q}{4\pi{n^2}\alpha}}=\SI{568.4e6}{\metre}
    \end{gather*}

\item
    For comparison, the Sun's radius is $\Rsol=\SI{695.7e6}{\metre}$.
    Of course, it doesn't make a lot of sense for \rs~to be smaller then the
    star's radius.

    % In this case the Sun's photon rate was estimated
    % incorrectly.
    In this case the density of the interplanetary medium was esimated
    incorrectly. Instead considering $n=\SI{5}{\per\centi\metre\cubed}$ one
    finds:
    \begin{equation*}
        \rs\approx\SI{2.64e9}{\metre}=\SI{3.79}{\Rsol}
    \end{equation*}

    % Consider instead:
    % \begin{gather*}
    %     Q=\frac{\Lsol}{h\nu}=\frac{\SI{3.828e26}{\watt}}{\SI{13.6}{\electronvolt}}\sim\SI{1.76e44}{\per\second}
    %     \\
    %     \implies\rs\approx\SI{3.97e15}{\metre}=\SI{5.70e6}{\Rsol}
    % \end{gather*}

\item
    In case of a O6 star in a HII region with the provided values:
    \begin{equation*}
        \rs=\SI{8.5e15}{\metre}=\SI{.23}{\parsec}
    \end{equation*}

\item
    The recombination time is given as $\trec=1/n\alpha$. \\
    Firstly, for $n=\SI{5e9}{\per\metre\cubed}$:
    \begin{equation*}
        \trec=\SI{7.69e8}{\second}\approx\SI{24.39}{\year}
    \end{equation*}
    And for $n=\SI{1e8}{\per\metre\cubed}$:
    \begin{equation*}
        \trec=\SI{3.85e10}{\second}\approx\SI{1219.61}{\year}
    \end{equation*}

\item
    The values obtained above indicate that mainly young and hot stars are
    responsible for the development of HII clouds.

\end{enumerate}


\end{document}
