%! TEX program = lualatex
\documentclass[11pt,a4paper]{scrartcl}
\usepackage{fontspec}
\usepackage{polyglossia}
    \setdefaultlanguage{ngerman}
\usepackage{lmodern}
\usepackage{fixmex}
\usepackage{graphicx}
\usepackage{csvsimple}
\usepackage{siunitx}
\usepackage{tikz}   % foreach

\usepackage[%
    backend=biber,
    natbib=true,
    sorting=nyt,
    url=false,
    doi=false,
    isbn=false,
    maxbibnames=3,
    maxcitenames=2,
    date=year,
    style=numeric-verb,
    citestyle=numeric-comp
]{biblatex}
\addbibresource{literatur.bib}

\usepackage[hidelinks]{hyperref}


\title{}
\subject{Bearbeitung von EwA-Blatt 4}
\author{Jeremiah Lübke}
\date{\today}


\begin{document}

\maketitle

\section{Aufgabe 1: Engauge}
\subsection{Vorgehen}
\begin{itemize}
    \item Lade Abb. 2 von \cite{Godfrey2013} herunter (siehe
        \hyperref[Link]{https://doi.org/10.1088\%2F0004-637x\%2F767\%2F1\%2F12})
    \item Öffne Abbildung in \textsw{Engauge Digitizer}
    \item Spezifiziere vier Punktkurven \textsw{NLRG, BLRG, Quasar, LERG}
    \item Erstelle drei Referenzpunkte bspw.~%
        $(10^{25}, 10^{47}); (10^{28}, 10^{44}); (10^{28}, 10^{47})$
    \item In \textsw{Settings/Coordinates} wähle \textsw{Log} für x- und
        y-Koordinaten
    \item Für jede Punktkurve:
    \begin{itemize}
        \item In \textsw{Settings/Color Filter} wähle \textsw{Hue} und
            stelle den entspr.~Farbbereich ein
        \item Füge die zugehörigen Datenpunkte mit dem \emph{Point Match
            Tool} der Punktkurve hinzu
    \end{itemize}
    \item In \textsw{Settings/Export Format} wähle \textsw{Raw Xs and Ys}
        und \testsw{One curve on each line}
    \item Exportiere die Daten als \textsw{.csv}-Datei
    \item Im Terminal (unter Linux): \verb+csplit Q-vs-L.csv /^$/ {*}+
\end{itemize}

\subsection{Ergebnis}
\foreach \curve [count=\i] in {NLRG,BLRG,Quasar,LERG}{%
\begin{table}
    \begin{tabular}{S S}
        \csvreader[head to column names]{xx0\i.csv}
        $L_{radio}/\si{W.Hz^{-1}.Sr^{-1}}$ &
        $Q_{HS}/\si{erg.s^{-1}}$
        \\ \hline
        {\x & \\curve}
    \end{tabular}
\end{table}
}

\end{document}
