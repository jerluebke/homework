%! TEX program = lualatex
\documentclass[11pt,a4paper]{scrartcl}
\usepackage{fontspec}
\usepackage{polyglossia}
    \setdefaultlanguage{german}
\usepackage{csquotes}
\usepackage{lmodern}
\usepackage{fixcmex}
\usepackage{graphicx}
\usepackage{csvsimple}
\usepackage{siunitx}
\usepackage{tikz}   % foreach
\usepackage{float}

\usepackage[%
    backend=biber,
    natbib=true,
    sorting=nyt,
    url=false,
    doi=false,
    isbn=false,
    maxbibnames=3,
    maxcitenames=2,
    date=year,
    style=numeric-verb,
    citestyle=numeric-comp
]{biblatex}
\addbibresource{literatur.bib}

\usepackage[colorlinks]{hyperref}


\newcommand{\zB}{z.\,B.}


\title{}
\subject{Bearbeitung von EwA-Blatt 4}
\author{Jeremiah Lübke}
\date{\today}


\begin{document}

\maketitle

\section{Aufgabe 1: Engauge}
\subsection{Vorgehen}
\begin{itemize}
    \item Lade Abb. 2 von \cite{Godfrey2013} herunter (siehe
        \href{https://doi.org/10.1088\%2F0004-637x\%2F767\%2F1\%2F12}{Link})
    \item Öffne Abbildung in \texttt{Engauge Digitizer}
    \item Spezifiziere vier Punktkurven \texttt{NLRG, BLRG, Quasar, LERG}
    \item Erstelle drei Referenzpunkte, \zB~%
        $(10^{25}, 10^{47}); (10^{28}, 10^{44}); (10^{28}, 10^{47})$
    \item In \texttt{Settings/Coordinates} wähle \texttt{Log} für x- und
        y-Koordinaten
    \item Für jede Punktkurve:
    \begin{itemize}
        \item In \texttt{Settings/Color Filter} wähle \texttt{Hue} und
            stelle den entspr.~Farbbereich ein
        \item Füge die zugehörigen Datenpunkte mit dem \emph{Point Match
            Tool} der Punktkurve hinzu
    \end{itemize}
    \item In \texttt{Settings/Export Format} wähle \texttt{Raw Xs and Ys}
        und \texttt{One curve on each line}
    \item Exportiere die Daten als \texttt{.csv}-Datei
    \item Im Terminal (unter Linux): \texttt{./prepare-data.sh}
\end{itemize}

\subsection{Ergebnis}
\foreach \curve [count=\i from 0] in {NLRG, BLRG, Quasar, LERG} {%
    \begin{table}[H]
        \centering
        \begin{tabular}{cc}
            $L_{151}/\si{W.Hz.^{-1}.Sr^{-1}}$ &
            $Q_{HS}/\si{erg.s^{-1}}$ \\ \hline
            \csvreader[head to column names]{xx0\i.csv}{}%
            {\tablenum{\csvcoli} & \tablenum{\csvcolii} \\}
        \end{tabular}
    \caption{\curve}
    \end{table}
}

\section{Aufgabe 3: Git}
Bearbeitung der Aufgabe siehe Commit-Historie
\href{https://github.com/jerluebke/homework}{dieses Repos}. \par\noindent
Zum Zurücksetzen auf einen vorherigen Zeitpunkt:
\begin{verbatim}
    git checkout <commit-hash>
    git checkout -b <new-branch>
    <mache Änderungen>
    git commit -am "..."
    git checkout master
    git merge <new-branch>
\end{verbatim}

\printbibliography[title=Literaturverzeichnis]

\end{document}
