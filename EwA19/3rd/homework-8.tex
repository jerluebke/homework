\newcommand{\nextpart}{%
    \bigskip
    \noindent\rule{\textwidth}{1pt}
    \bigskip
}

\newcommand{\dn}[1]{%
    \ensuremath{\mathrm{d}^{#1}}
}

\newcommand{\TODO}[1]{%
    \textcolor{red}{\textbf{TODO} #1}
}

\newpage
\section{Aufgabe 8: Befehle definieren}
\textbf{Das neudefinierte Kommando}:
\begin{verbatim}
\newcommand{\dn}[1]{%
    \ensuremath{\mathrm{d}^{#1}}
}
\end{verbatim}

\textbf{Beispiel:} \textsw{dn[5]} $\Rightarrow \dn{5}$

\nextpart

\textbf{Ein TODO-Kommando}:
\begin{verbatim}
\newcommand{\TODO}[1]{%
    \textcolor{red}{\textbf{TODO} #1}
}
\end{verbatim}

\textbf{Beispiel:} \TODO{überlege ein gutes Beispiel}

\nextpart

\todo{QUACK!}

Mit dem Paket \textsw{todonotes} werden Notizen bspw.~bei folgenden
Konfigurationen entfernt:
\begin{verbatim}
\usepackage[...,draft,...]{scrartcl}
...
\usepackage[obeyDraft]{todonotes}
\end{verbatim}
oder
\begin{verbatim}
\usepackage[disable]{todonotes}
\end{verbatim}
