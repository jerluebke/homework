% Dies ist eine Vorlage für Eure Bachelorarbeit an der RUB,
% erstellt von Alexander Noack, zur freien Verwendung ;-)

\RequirePackage[l2tabu,orthodox]{nag} % checke common mistakes/outdated pkgs
% Die KOMA Klasse für article mit einer 11pt Schrift auf A4 Papier
\documentclass[11pt,
               a4paper,
               parskip=half,
               % draft,
%               bibliography=totoc,
               ]{scrartcl}

% ============ %
% Pakete laden %
% ============ %

% Damit Trennregeln aktualisiert werden
\usepackage[ngerman=ngerman-x-latest]{hyphsubst} % als erstes laden!

% Input encoding auf utf8 setzen, und die Ausgabe in T1 kodieren (Westeuropa)
\usepackage[utf8]{inputenc} % in einer aktuellen LaTeX Version nicht nötig
\usepackage[T1]{fontenc}

% geometry -- praktisch, wenn man genaue Vorstellungen/Vorgaben des Layouts hat
\usepackage[pass]{geometry} % Option pass übergibt die Werte von KOMA

% Kopf- und Fußzeile verändern
\usepackage{scrlayer-scrpage}

% Nutze latin modern font und fixe ein paar Probleme
\usepackage{lmodern}
\usepackage{fixcmex}

% Mikro-Typographie
\usepackage{microtype}

% Deutsche Sprache mit _n_euer Rechtschreibung zusätzlich zu englisch
\usepackage[english,ngerman]{babel}

% Komfortables Titel-, Autor-, .. -Handling
\usepackage{titling}

% Um zum Beispiel Grafiken einzubinden
\usepackage[usenames,dvipsnames]{xcolor}
\usepackage{graphicx} % lädt auch xcolor
    \graphicspath{{./img/}}

% Mathematik Pakete und extra Symbole
%\usepackage{amsmath}
\usepackage{mathtools} % lädt auch amsmath
\usepackage{amssymb}
\usepackage{textcomp}
\usepackage{gensymb}

% Ermöglicht komfortabel das Anpassen von Abständen, Labels etc. für itemize
\usepackage{enumitem}

% Für mehrere Bilder die zusammengehören praktisch
\usepackage{subcaption}

% Lade Tabellen Pakete
\usepackage{array}
\usepackage{booktabs}
\usepackage{multirow}
%\usepackage{tabularx}
%\usepackage{longtable}
\usepackage{csvsimple}

% Um Quellcode einzubinden
\usepackage{listings}

% Um mit LaTeX zu zeichnen und zu plotten
%\usepackage{tikz}
%\usepackage{pgfplots} % lädt auch tikz

% Physik- und formel- oder symbolbezogene Pakete
\usepackage{siunitx}
\usepackage{physics}
%\usepackage{braket}
\usepackage[thinc]{esdiff}

% Erleichtern das Leben beim Editieren, Probelesen und Arbeiten
\usepackage{blindtext}
\usepackage[colorinlistoftodos,obeyDraft]{todonotes}
\usepackage{lineno}

% Paket für wörtliche Zitate, URLs, einstellbaren Zeilenabstand
\usepackage{csquotes}
\usepackage{url}
\usepackage{setspace}

% Pakete, um User-defined-Macros zu erstellen, oder vorhandene zu verändern
\usepackage{xspace}
%\usepackage{xparse}

% für anklickbare Links (cross-referencing) ins Dokument und nach außen.
\usepackage[hidelinks,
            linktocpage=false,
            pdfusetitle]{hyperref} % als letztes (spät, exceptions) laden!
\usepackage{bookmark} % als letzteres laden, Optimierungen zu hyperref, \pdfbookmark

% =============== %
% Ende für Pakete %
% =============== %


% User-defined macros
\newcommand{\zB}{z.\,B.\xspace}
\newcommand{\ZB}{Z.\,B.\xspace}
\newcommand{\textsw}[1]{\texttt{#1}} % sw=software
\newcommand{\file}[1]{\texttt{#1}} % file names

% Header, Footer
\ihead{\thetitle} % inner head (einseitig links)
\chead{} % center head
\ohead{\rightmark} % outer head (einseitig rechts)
\automark{section} % setze \rightmark auf section name
\automark*{subsection} % falls subsection vorhanden auf diese
\setkomafont{pagehead}{\sffamily} % Kopfzeilen-Font

% list related (itemize, enumerate)
\setlist{nosep} % entferne alle Abstände
\setitemize[1]{label=\raisebox{.37ex}{\scalebox{.6}{$\bullet$}}} % kleinere bullet

% math related
\AtBeginDocument{% verkleinere Raum um Mathe Umgebungen
  \setlength{\abovedisplayskip}{6pt plus 3pt minus 3pt}%=11pt plus 3pt minus 6pt
  \setlength{\abovedisplayshortskip}{0pt plus 3pt}%=0pt plus 3pt
  \setlength{\belowdisplayskip}{6pt plus 3pt minus 3pt}%=11pt plus 3pt minus 6pt
  \setlength{\belowdisplayshortskip}{4pt plus 3pt minus 3pt}%=6.5pt plus 3.5pt minus 3pt
}
\newcolumntype{L}{>{$}l<{$}} % math-mode version of "l" column type
\newcolumntype{C}{>{$}c<{$}} % math-mode version of "c" column type

% table related
\setlength{\cmidrulekern}{.4em}

% Neue Einheiten für siunitx
\DeclareSIUnit{\erg}{erg}
\DeclareSIUnit{\Gauss}{G}

% Titel, Autor, etc
\title{Beispieldokument EWA} % Titel in Sprache der Arbeit
\newcommand{\theothertitle}{Example Document} % Titel in anderer Sprache
\newcommand{\bachelormaster}{Bachelor} % {Bachelor} oder {Master}
\newcommand{\sciencearts}{Science} % {Science} oder {Arts}
\author{Vorname Name} % Autor
\newcommand{\placeofbirth}{Deutschland} % Geburtsort
\newcommand{\location}{Bochum}
\date{2018} % Datum (Jahr)

% ================= %
% Ende der Preamble %
%  Beginn Dokument  %
% ================= %

\begin{document}

% Titelseite
% Benötigte Pakete in dieser Version:
% geometry, hyperref, bookmark, titling, setspace
\newgeometry{margin=2.5cm,bottom=4cm}%,bindingoffset=6mm}
\pdfbookmark[section]{Titelseite}{titlepage}
\begin{titlepage}
  \centering
  {\huge\titlefont\thetitle\par
                  \bigskip\bigskip
                  \theothertitle\par}
  \vspace{2cm}

  \begin{spacing}{0.8}
    {\LARGE \bachelormaster arbeit\par
            \bigskip\medskip
            im Studiengang\par
            ,,\bachelormaster{} of \sciencearts``\par
            im Fach Physik\par
            \bigskip\medskip
            an der Fakultät für Physik und Astronomie\par
            der Ruhr-Universität Bochum\par}

    \vfill

    {\LARGE von\par
            \theauthor\par
            \bigskip\medskip
            aus\par
            \placeofbirth\par}
  \end{spacing}

  \vspace{1.8cm}

  {\LARGE \location{} \thedate\par}
\end{titlepage}
\restoregeometry
\cleardoublepage

% Table of Contents
\pdfbookmark[section]{\contentsname}{toc}
\tableofcontents
\cleardoublepage

\section{Beispieldokument}
Das vorliegende Dokument ist ein Beispieldokument aus dem Kurs
\enquote{Einführung in das wissenschaftliche Arbeiten}.
Es dient \zB~dazu, aufzuzeigen,
wie die sogenannte (auf neudeutsch) \emph{Preamble}
Ihrer Bachelorarbeit aussehen könnte.
Es darf wild zitiert werden,
oder mit Formeln umher geworfen werden,
im Text $\sin^2 x + \cos^2 x = 1$,
so wie in den zugehörigen Formel-Umgebungen\dots

Es könnte sich als sinnvoll erweisen,
Aufzählungen zu verwenden:
\begin{itemize}
  \item Punkt 1
  \begin{itemize}
    \item Unterpunkt 1.1
    \item Unterpunkt 1.2
  \end{itemize}

  \item Punkt 2
  \begin{itemize}
    \item usw\dots
  \end{itemize}
\end{itemize}

Man kann auch \enquote{wörtliche Rede} verwenden.


\clearpage
\section{Wichtige Pakete}
%
\begin{description}
  \item[KoMa/KOMA-Script] Der Name hat natürlich nichts mit dem Zustand zu tun,
    sondern mit dem Namen des Autors dieser Pakete: Markus Kohm.
    Es handelt sich um eine Sammlung von Klassen und Paketen,
    die einem viele Dinge abnehmen und automatisieren,
    um die sich der unbedachte Physiker
    vielleicht gar keine Gedanken macht.
    So kann man damit zum Beispiel, mit minimaler Interaktion,
    Text in ideal große Bereiche auf einer Seite setzen,
    oder auch die Kopf- und Fußzeile beliebig abändern.
    Eine Auswahl der beinhalteten Pakete sind:
    \begin{itemize}
      \item \textsw{scrartcl}
      \item \textsw{scrlttr2}
      \item \textsw{scrbook}
      \item \textsw{scrlayer-scrpage}
    \end{itemize}

  \item[beamer] Dies ist namentlich inspiriert vom deutschen Gebrauch
    des im englischen nicht vorhandenen Worts. Es handelt sich aber,
    wie man, wenn man der deutschen Sprache mächtig ist, erahnen kann,
    um einen Projektor.
    Mit dieser Klasse kann man also Beamer-Präsentationen,
    die klassisch Powerpoint vorbehaltenen Präsentationen
    auch mit \LaTeX{} angehen und professionell gestalten.
    Der große Vorteil von \LaTeX,
    dass Inhalt und Layout getrennt sind (sein sollten),
    ist hier besonders Wert erwähnt zu werden.
    Man kann ein einheitliches Bild gut erreichen,
    auch wenn man Präsentationen kombiniert.
    Auch Formeln lassen sich in gewohnter Manier einbinden
    und professionell setzen.

  \item[babel] Hauptsächlich für die korrekte Worttrennung in der Sprache,
    aber auch für weitere Eigenheiten.
    So ist \zB~im Englischen nach dem Satzende durch einen Punkt
    ein längeres Leerzeichen zu setzen, als zwischen Wörtern.
    Anders ist es im Französischen und Deutschen.
    Bei solchen Details hilft \textsw{babel}.
    Unterstützt mehrere Sprachen, auch in einem Dokument.

  \item[hyphsubst] Im Deutschen gibt es Wörter,
    die falsch nach den Standardregeln von babel getrennt werden.
    So wurde bei mir z.B. Einzelstern als \enquote{Einzels-tern} getrennt.
    Um dem Abhilfe zu schaffen kann man dieses Paket
    (früh, auch vor der \verb+\documentclass+) wie folgt laden:
    \verb+\RequirePackage[ngerman=ngerman-x-latest]{hyphsubst}+

  \item[inputenc] Um auch Buchstaben mit Akzenten
    und nicht-lateinische Buchstaben
    korrekt aus den Quelldateien (\file{*.tex}) zu lesen,
    sollte ein input encoding mit dem Laden
    dieses Pakets vorgegeben werden.

  \item[fontenc] \LaTeX{} schafft es, alle möglichen Symbole,
    Buchstaben, mit Verzierungen, Akzenten etc. darzustellen,
    aber verwendet dafür nicht zwangsweise die vorhandenen Symbole
    aus dem Font, sondern setzt diese selbst zusammen.
    Das ist natürlich nicht gut aus dem PDF kopierbar.
    Um dem Abhilfe zu schaffen sollte man
    das jeweils regionale Font-Encoding (für Westeuropa T1)
    mit diesem Paket laden: \verb+\usepackage[T1]{fontenc}+

  \item[lmodern] Der Standardfont von \LaTeX
    ist der von Donald Knuth für \TeX entwickelte
    Font \enquote{Computer Modern}.
    Dieser wurde jedoch nur für bestimmte einzelne Größen
    zwischen 5\,pt und 17\,pt von Hand erstellt (5, 6, 7, 8, 9, 10, 12, 17),
    um die Strichstärken und relativen Proportionen
    auf die jeweilige Größe anzupassen.
    Um dieses Problem, wenn man Zwischengrößen
    oder außerhalb des Bereichs liegende Font-Größen braucht,
    kann man dieses Paket laden.
    Es verwendet immer noch diesen Standardfont, als Vektorgraphiken,
    skaliert auf die entsprechende Größe,
    ausgehend von der nächstgelegenen vorhandenen Schriftgröße.
    Natürlich kann man auch andere Fonts auswählen,
    aber dieses hier ist die einfache und klassische
    Rundum-Sorglos-Packung. Wird \textsw{lmodern} verwendet,
    so sollte man auch \textsw{fixcmex} verwenden.
    Alternative: \textsw{cm-super}

  \item[fixcmex] Wird \textsw{lmodern} zusammen mit
    \textsw{amsmath} geladen,
    so gehen ein paar Funktionalitäten von \textsw{amsmath} verloren.
    Um diese zu beheben, sollte
    (möglichst spät, auf jeden Fall nach Font-Paketen und Änderungen)
    dieses Paket zusätzlich geladen werden.

  \item[microtype] Es gibt eine Menge Infos über Typographie,
    bis hin zu vielen kleinen Details, die Mikrotypographie genannt werden.
    Was genau macht dieses Paket, ohne zu sehr auf Details einzugehen?
    Minimales Skalieren der Buchstaben, um eine Zeile Text
    doch noch in die vorgegebene Breite einzupassen
    (ohne die bei \LaTeX{} Usern so bekannten badboxes),
    leichtes Verschieben nach links bzw. rechts am linken bzw. rechten Rand,
    um einen optisch ansprechenderen Abschluss zu haben,
    und einige weitere Features,
    die man teils sehr fein einstellen kann. (Siehe Manual)

  \item[geometry] Nur, wenn man ganz konkrete Vorgaben
    (oder Vorstellungen :) hat, wie viel Platz zwischen Seitenelementen
    und zu den Seitenrändern sein soll.
    Man kann sehr fein einstellen, wie das Layout einer Seite aussehen soll.

  \item[siunitx] Gerade in der Physik und Naturwissenschaften
    sind Einheiten essentiell.
    Um diesen auch im Schriftsatz gerecht zu werden,
    sollten diese einheitlich gesetzt werden.
    Dafür ist \textsw{siunitx}.

  \item[array, booktabs, multirow, tabularx, longtable]
    \textsw{array} verbessert viel \enquote{unter der Haube}
    für Tabellen, aber bietet auch neue Funktionalität,
    wie selbst erstellbare Spalten.
    \textsw{booktabs} hübscht Tabellen weiter auf
    und bietet ebenfalls neue Funktionen,
    wie bestimmte Linien für obere und untere Grenze der Tabelle.
    Insbesondere das Handbuch ist lesenswert
    mit vielen guten Beispielen zu Stilfragen (als Style-Guide).
    \textsw{multirow} bietet einem den praktischen \verb+\multirow+-Befehl,
    der genau das tut, was der Name sagt,
    ähnlich zum \verb+\multicolumn+-Befehl.
    \textsw{tabularx} ist nützlich,
    wenn man eine Tabelle einer bestimmten Breite erzeugen will
    und die Spaltenbreite entsprechend angepasst werden soll.
    \textsw{longtable} ist vermutlich vor allem
    für Experimentalphysiker interessant,
    die eben mit langen Tabellen von \zB~Messdaten hantieren müssen.
    Eine Tabelle kann üblicherweise nicht auf mehrere Seiten verteilt werden,
    wenn sie zu viele Zeilen hat.
    \textsw{longtable} gibt einem diese Funktionalität.

  \item[csvsimple] Tabellen in \LaTeX{} von Hand zu erstellen,
    insbesondere mit vielen Werten und ggf.~Einheiten, kann eine Qual sein.
    Es gibt mittlerweile einige Tools, \zB~Webseiten,
    die Abhilfe schaffen können, aber besser ist,
    wenn man einfach eine \file{*.csv} Datei
    (kann aus Excel/Origin exportiert werden)
    einliest und als Tabelle darstellt.

  \item[xcolor, graphicx] Was wäre die (heutige) Welt ohne Farbe?
    Ohne Bilder?
    Möchte man dies in Form von \zB~Graphiken in seine Dokumente einbinden,
    braucht man diese Pakete\dots
    Hinweis: \textsw{xcolor} wird von vielen Paketen geladen
    und muss nur manuell geladen werden,
    falls sonst kein solches Paket (\zB~\textsw{graphicx}) verwendet wird.

  \item[hyperref] Jeder kennt heutzutage Links.
    Diese sind in Büchern meist nicht vorhanden,
    aber im digital anzusehenden PDF-Dokument.
    Um Links zu erzeugen,
    um \zB~einfach eine section aus der TOC anzuspringen,
    kann dieses Paket geladen werden.
    Damit werden URLs, wie auch Referenzen, clickable.
    Hinweis: Dieses Paket sollte als \emph{letztes} geladen werden.

  \item[url] Um URLs in seinem Dokument zu setzen,
    sollte das Paket \textsw{url} geladen werden.
    Es sorgt dafür, dass sie in einem typewriter Stil gesetzt
    und besser getrennt werden,
    falls sie über eine Zeile im Text hinausragen würden.

  \item[listings] Es gibt immer wieder Skripte,
    kleine oder größere, und Programme,
    die man an seine Arbeit anhängen möchte.
    Um Syntax-Highlighting und andere sinnvolle
    Features zu nutzen, kann man dieses Paket laden.
    Alternative: \textsw{minted}

  \item[setspace] Auch beim Zeilenabstand gilt:
    \LaTeX{} macht, was es tut, aus gutem Grund.
    Das ist auch gar nicht so schlecht,
    aber manchmal möchte man andere,
    als die Standard-Zeilenabstände.
    Dann empfiehlt es sich, dieses Paket zu laden,
    welches einem komfortabel ermöglicht,
    Zeilenabstände einzustellen.

  \item[csquotes] Möchte man Anführungszeichen verwenden,
    sollte man sich überlegen, wofür.
    Häufig wird wörtlich zitiert. Unter anderem dann bietet es sich an,
    dieses Paket zu laden, das den Befehl \verb+\enquote+ bereitstellt.
    Dieser kann beliebig variiert werden,
    je nach Sprache, Region und Vorliebe.

  \item[esdiff] Ja, die Mathematik kommt heutzutage
    wohl kaum ohne die von Leibniz eingeführte
    Differential-Schreibweise aus.
    Um sich das Leben so einfach wie möglich zu machen,
    empfiehlt es sich, ein Paket zu laden,
    statt selbst Befehle dafür zu schreiben.
    Alternativen: \textsw{physics}, \textsw{cool}

  \item[braket] Gibt einem Befehle an die Hand,
    um \textlangle Bra-Ket\textrangle Notation einfach zu verwenden.
    Alternativen: \textsw{physics}

  \item[physics] Viele kleine nützliche Tools,
    wie Differentiale, Bra-Ket,
    automatische Klammerskalierung nach Mathe-Befehlen
    (wie \verb+\sin+), \dots

  \item[amsmath, mathtools, amssymb] Die Mathematik\dots
    Das Paket \textsw{amsmath} sollte quasi immer geladen werden,
    wenn man mit mathematischen Ausdrücken hantiert.
    \textsw{mathtools} lädt \textsw{amsmath} selbstständig,
    und bietet einige nützliche erweiterte Funktionen.
    \textsw{amssymb} macht einige häufig verwendeten Symbole verfügbar.

  \item[appendix] Macht einem das Leben mit Anhang leichter\dots
    Bietet einige nützliche Zusatzfunktionen rund um Anhänge.

  \item[lineno] Kann, wie der Name vermuten lässt, Zeilennummern ausgeben.
    Sehr nützlich für Korrekturlesen oder Arbeitsversionen.

  \item[todonotes] Praktisch, wenn man im Dokument Todo-Notes haben möchte.
    Ermöglicht einem auch eine list of todos anzuzeigen, ähnlich zu TOC,
    verschiedene Farben\dots
    Alternativen: \textsw{easy-todo}, \textsw{fixme},
      \textsw{fixmetodonotes}, \textsw{todo}

  \item[blindtext] Erzeugt, was der Titel sagt.
    Vorteil gegenüber den anderen Paketen:
    Die Sprache, die mit Babel geladen wird,
    wird für den Blindtext verwendet --
    zumindest für den Fall der deutschen Sprache.
    Außerdem gibt es die Möglichkeit,
    Blindtexte mit Mathe zu erstellen.
    Alternativen: \textsw{lipsum}, \textsw{kantlipsum}

  \item[parskip] Verwendet man keine \textsw{KOMA} Klasse,
    so kann man alternativ den Abstand
    zwischen Paragraphen mithilfe dieses Pakets einstellen.

  \item[fancyhdr] Steht für \enquote{fancy header}.
    Sollte ausschließlich ohne \textsw{KOMA} verwendet werden.

  \item[nag] Gibt zusätzliche Warnungen aus,
    falls ein paar typische Fehler oder obsolete
    Pakete gefunden werden, ist jedoch ein wenig outdated.

  \item[xspace] Für variable Leerzeichen in user-defined Macros.

  \item[tikz, pgfplots] Für Skizzen, (Vektor-)Graphiken,
    und noch vieles mehr, \zB~auch dann, und überall dort,
    wo die vorhandenen Zeichen von \LaTeX{} nicht ausreichen.

  \item[biblatex (mit \textsw{biber} backend)]
    Das aktuellste Literaturpaket,
    welches bei neuen eigenen Arbeiten verwendet werden sollte --
    quasi ein Muss.

  \item[xparse] Ermöglicht das Ändern vorhandener Befehle
    und gibt Alternative zu \verb+\newcommand+.

  \item[caption] Einfaches Anpassen von captions (\zB~Font).

  \item[subcaption] Möglichkeit, mehrere Captions/Floats
    neben- und übereinander zu nutzen.

  \item[enumerate/enumitem] Einfaches Ändern vieler Einstellungen
    um Listen/itemizes. Kann zu Kompatibilitätsproblemen führen!

  \item[titling] Zugriff auf Titel und Autor, leichtes Anpassen.

  \item[floatflt] Text um Floats fließen lassen.

  \item[relsize] Ermöglicht relatives Schriftgröße ändern
    (nützlich für \zB~kleinere Inidizes).

  \item[standalone] Um \zB~standalone tikz Bilder einzubinden.

  \item[bookmark] Erweitert und verbessert Funktionalität von \textsw{hyperref}.

\end{description}
%
\section{\enquote{Dos and Don'ts} und Tipps}
Um von Beginn den korrekten Umgang mit \LaTeX{} zu üben,
folgt eine Liste mit \enquote{Dos and Don'ts},
sowie hilfreichen Tipps zur Verwendung von \LaTeX{} und Problembehebung.
\subsection{Dos and Don'ts}
%
\begin{itemize}
  \item Don't: Verwende keine manuell eingefügten Anführungszeichen {\verb+"+} im Fließtext.
  \item Do: Lade das Paket \verb+csquotes+ und verwende den Befehl \verb+\enquote+.
\end{itemize}

\begin{itemize}
  \item Don't: Verwende nicht \verb+\\+ im Text,
    um in einer neuen Zeile weiterzuschreiben.
    Eine Ausnahme bilden der Mathematikmodus und Tabellen.
  \item Do: Füge eine Leerzeile ein oder verwende den Befehl \verb+\par+.
\end{itemize}

\begin{itemize}
  \item Don't: Verwende bei Abkürzungen nicht das falsche Leerzeichen wie bei z. B.
  \item Do: Nutze den Befehl \verb+\,+ zwischen den Zeichen.
\end{itemize}

\begin{itemize}
  \item Don't: Verwende keine falschen Leerzeichen vor Verweisen wie \enquote*{siehe XY}.
  \item Do: Verwende non-breaking space \enquote*{siehe\textasciitilde XY}.
\end{itemize}

\begin{itemize}
  \item Don't: Ignoriere keine Warnungen und erst recht nicht Fehlermeldungen.
    Ebenfalls sollten badboxes nicht ignoriert werden.
  \item Do: Man lese sich die Warnung durch und versuche sie zu beheben.
    Sie werden nicht grundlos ausgegeben.
\end{itemize}

\begin{itemize}
  \item Don't: Mische die Klassen von \textsw{KOMA} nicht mit \textsw{fancyhdr}.
  \item Do: Verwende \textsw{scrlayer-scrpage} mit \textsw{KOMA}-Klassen.
\end{itemize}

\begin{itemize}
  \item Don't: Versuche nicht die Probleme lokal zu lösen.
  \item Do: Verwende überall die selben allgemeinen Befehle,
    um ein einheitliches Erscheinungsbild zu garantieren
    und gegebenenfalls leicht Änderungen einzufügen.

    Nutze \LaTeX{} als das, wofür es gedacht ist.
    Es trennt Inhalt von Layout.
    Nutze also keine Layout-Befehle im Text,
    sondern verwende sinnvolle preamble.

    Benenne Befehle nach Anwendung und Sinn,
    nicht beschreibend, was sie tun (Beispiel: \verb+\section+ statt \verb+\large\bfseries+).
\end{itemize}

\begin{itemize}
  \item Don't: Verwende keine Serifenschrift für Beamer-Präsentationen,
    die am Bildschirm gesehen werden.
  \item Do: Hier empfiehlt sich, wie es Voreinstellung ist,
    eine serifenlose Schrift zu verwenden.
\end{itemize}

\begin{itemize}
  \item Don't: Verwende nicht \verb+\sloppy+,
    weil es viele Warnungen und badboxes behebt. Das ist sloppy :)
  \item Do: Verwende \verb+\emergencystretch+, \textsw{microtype}, \textsw{url},
    manuelle Worttrennungen, \dots
\end{itemize}

\begin{itemize}
  \item Don't: Verwende nicht die kursiven Differentiale \verb+\frac{d}{dx}+.
  \item Do: Verwende aufrechte Differentiale mit \verb+\mathrm{d}+,
    oder besser nutze \zB~das Paket \textsw{esdiff}.
\end{itemize}

\begin{itemize}
  \item Don't: Verwende nicht (zu viele) vertikale Linien in Tabellen.
  \item Do: Besser ist es wenige horizontale Linien, gut gewählte Abstände,
    ggf. Hintergrundkolorierung zu verwenden (siehe \textsw{booktabs}).
\end{itemize}

\subsection{Weitere hilfreiche Tipps.}
Es ist sehr empfehlenswert, sich die Dokumentationen von Paketen,
insbesondere von \textsw{KOMA} durchzulesen.
Es gibt viele gute \LaTeX{} Grundlagenbücher (\zB~wikibooks),
die man zu Rate ziehen kann.
Im Internet gibt es Seiten wie \url{http://tex.stackexchange.com}
oder \url{http://texwelt.de}, auf denen einem kompetente User
und sogar \LaTeX{} Core-Entwickler helfend zur Seite stehen.

Für nahezu jeden Wunsch gibt es bereits Pakete.
Man muss sie nur finden oder ihren Namen ausfindig machen.
Dafür ist z.B.~\url{http://ctan.org} hilfreich.
Dort findet man zahlreiche Manuals und Dokumentationen.
Mit \textsw{tlmgr} lassen sich Pakete installieren.

Mit der App und Internetseite \textsw{detexify}
kann man ein Symbol malen und den zugehörigen \LaTeX-Befehl finden.

Mit \textsw{tikz} lassen sich sogar Graphen
ohne Zusatzsoftware im \LaTeX-Dokument zeichnen.

Der Befehl \verb+\space+ kann ein Leerzeichen erzwingen,
genauso wie mit \verb+\␣+, wobei \verb+␣+ ein Leerzeichen ist.
Es gibt auch Leerzeichen, die mehr oder weniger Platz einnehmen,
\zB~\verb+\,+, welches man bei Abkürzungen nach einem Punkt verwenden sollte.

Falls man mit all den Befehlen nur schwer zurecht kommt
und Microsoft Word vermisst, so kann man \textsw{LyX} ausprobieren und nutzen.
Das ist mehr ein Word-artiger WYSIWYM Editor.
Möchte man nicht dauernd auf Pakete zurückgreifen,
sondern eigene und einfachere Befehle erstellen,
kann man einen Blick auf \textsw{ConTeXt} werfen.

Möchte man beim Eintippen komplexer und langer Formeln
die Formel live sehen, kann man \zB~\textsw{TeXstudio} benutzen.

Zusätzlich ist es empfehlenswert,
ein Version Control System wie \textsw{git} zu verwenden.
Des Weiteren helfen Build/make/docker Scripts einem,
Graphiken und Plots aus R oder python zu automatisiert zu erstellen,
wenn sich etwas ändert.

% multirow shortcut (1 column) - needed to escape siunitx's tabular S
\newcommand{\mc}[1]{%
    \multicolumn{1}{c}{#1}
}
% multirow shortcut (2 rows)
\newcommand{\mr}[1]{%
    \multirow{2}*{#1}
}

\section{Aufgabe 4: Tabellen in \LaTeX}
\begin{table}[h]
    \centering
    \begin{tabular}{|c|c|c|c|c|}
        \firsthline
        & & & \multicolumn{2}{c|}{$A_e, \textrm{erg}^{-1}$} \\
        Objekt & $\delta$ & $p_e$ & $B=10\si{\micro G}$ & $B=100\si{\micro G}$
        \\
        \hline
        NGC 1068 & -0.69 & 2.38 & \num{2.1e+67} & \num{4.2e+65} \\
        NGC 4945 & -0.59 & 2.18 & \num{3.8e+65} & \num{9.6e+63} \\
        NGC 253 & -0.65 & 2.31 & \num{5.2e+65} & \num{1.2e+64} \\
        NGC 3034 & -0.39 & 1.78 & \num{1.2e+64} & \num{4.9e+62} \\
        \lasthline
    \end{tabular}
    \caption{Experimentelle Daten...}
    \label{tab:tab1}
\end{table}

\begin{table}[h]
    \centering
    % \begin{tabular}{c*{5}{S[table-format=4.3,table-number-alignment=center]}}
    \begin{tabular}{c*{5}{S}}
        \toprule
        \mr{Objekt} &
        %%%%%%%%%%%%%%%%%%%%%%%%%%%%%%%%%%%%%%%%%%%%%%%%%%%%%%%%%%%%%%%%%%%%%
        % NOTE: second and third `\multirow` enclosed in {...} !!!          %
        %%%%%%%%%%%%%%%%%%%%%%%%%%%%%%%%%%%%%%%%%%%%%%%%%%%%%%%%%%%%%%%%%%%%%
        {\mr{$\delta$}} &
        {\mr{$p_e$}} &
        \multicolumn{2}{c}{$A_e/\left(10^{65}\textrm{erg}^{-1}\right)$} \\
        \cline{4-5}
        & & &
        \mc{$B=10\si{\micro G}$} &
        \mc{$B=100\si{\micro G}$} \\
        \hline
        NGC 1068 & -0.69 & 2.38 & 2100 & 4.2 \\
        NGC 4945 & -0.59 & 2.18 & 3.8 & 0.096 \\
        NGC \enspace 253 & -0.65 & 2.31 & 5.2 & 0.12 \\
        NGC 3034 & -0.39 & 1.78 & 0.12 & 0.0049 \\
        \bottomrule
    \end{tabular}
    \caption{Experimentelle Daten... \emph{(verbessertes Layout)}}
    \label{tab:tab2}
\end{table}


\newpage
\section{Aufgabe 5: Abbildungen in \LaTeX}
\begin{figure}[h]
\centering
    \begin{subfigure}[h]{.4\textwidth}
        \includegraphics[width=\textwidth]{duck}
        \caption{duck}
        \label{fig:duck}
    \end{subfigure}
    ~
    \begin{subfigure}[h]{.4\textwidth}
        \includegraphics[width=\textwidth,angle=-45]{quack}
        \caption{QUACKKK}
        \label{fig:quack}
    \end{subfigure}
    \caption{Mit der Option \textsw{draft} werden diese Bilder durch Platzhalter ersetzt.}
    \label{fig:fig1}
\end{figure}


\newpage
\section{Aufgabe 6: Platzierung von Gleitobjekten}
Itaque et consequatur quibusdam aut id quos. Esse aut dolor omnis. Sit quis expedita asperiores praesentium est voluptatum dolor. Blanditiis quaerat corporis occaecati.

Aut hic cum aut. Tempora occaecati eum doloribus minima perferendis qui occaecati nostrum. Voluptatum odio consequatur soluta sapiente cumque quibusdam ut. Similique maxime deserunt veniam pariatur.

\begin{table}[h]
    \centering
    \begin{tabular}{|c|c|c|c|c|}
        \firsthline
        & & & \multicolumn{2}{c|}{$A_e, \textrm{erg}^{-1}$} \\
        Objekt & $\delta$ & $p_e$ & $B=10\si{\micro G}$ & $B=100\si{\micro G}$
        \\
        \hline
        NGC 1068 & -0.69 & 2.38 & \num{2.1e+67} & \num{4.2e+65} \\
        NGC 4945 & -0.59 & 2.18 & \num{3.8e+65} & \num{9.6e+63} \\
        NGC 253 & -0.65 & 2.31 & \num{5.2e+65} & \num{1.2e+64} \\
        NGC 3034 & -0.39 & 1.78 & \num{1.2e+64} & \num{4.9e+62} \\
        \lasthline
    \end{tabular}
    \caption{Experimentelle Daten...}
    \label{tab:tab3}
\end{table}

Laboriosam quasi eos est delectus dicta at quisquam. Neque recusandae quo corporis. Suscipit nobis ut autem quae eos qui. Esse nihil soluta quidem eum amet aliquam est iure. Pariatur quia laudantium temporibus perferendis praesentium et tempora.

\begin{figure}[p]
    \centering
    \includegraphics[width=.4\textwidth]{Unterschrift}
    \caption{Meine Unterschrift}
    \label{fig:sign}
\end{figure}
\clearpage  % flush everything and continue

Modi at accusamus esse vel assumenda. Qui enim aliquid sed natus ullam. Numquam et ut hic porro repellat blanditiis.

Est itaque ea aut tenetur consequatur dolorem eveniet modi. Modi dolores expedita nemo dolores eum est vero. Et tempore molestias quos maxime consequuntur consequuntur pariatur occaecati. Voluptate et autem incidunt qui. Velit molestiae quam unde velit vero. Voluptate velit accusamus harum veritatis eligendi dolor soluta.

Iste molestiae at temporibus delectus saepe debitis excepturi sit. Perspiciatis ipsam earum saepe nam sed aut omnis quo. Explicabo id sint praesentium aspernatur facilis vitae.

Et doloremque commodi soluta quam nihil perferendis ut aut. Sapiente et tempora est quia numquam maiores ipsum iure. Corporis sunt dolores iste asperiores. Quae consequatur id non veritatis necessitatibus quod. Qui occaecati dolore tempore. Ut dicta est quibusdam pariatur.

Numquam eos et blanditiis. Hic eius illo ipsa consequatur facilis. Minima aperiam dolore dolorem soluta animi. Totam quasi qui saepe optio. Quia laborum doloremque iusto dolore explicabo molestiae. Sapiente omnis dolorum omnis quo.

Est ipsum voluptatum et sapiente consectetur dolores aut. Reprehenderit impedit et velit enim. Fugiat vero accusantium rem dignissimos sint. Eum qui porro unde ipsum eaque consectetur. Omnis et eos eos. Sequi ipsa enim et dignissimos et asperiores ea aut.

Quia qui mollitia ut. Totam qui hic aspernatur sint sit est qui. Ullam corrupti placeat atque rem rerum. Non rem reprehenderit alias laboriosam consequuntur omnis.

Eligendi modi minima est dolorum asperiores. Necessitatibus et doloribus nihil dolor laborum deleniti nisi. Qui fuga optio quos maiores quibusdam. Iure fugiat ratione et in vel praesentium qui sit. Saepe odio incidunt est quis. Quia sed accusantium quis et.

Maiores eveniet quisquam pariatur voluptatem voluptatibus. Sed ut voluptas maiores alias illum libero dolores. Omnis labore laudantium dolorem iusto optio ducimus eaque voluptatem. Tempore quasi eos unde eaque. Nulla quibusdam sint doloremque nesciunt.

Magni enim corporis maiores maxime autem aliquid nisi ratione. Consequatur id quidem et pariatur rem dignissimos ratione qui. Veritatis aut vero aspernatur commodi. Tempore libero doloribus officiis non quos dolorem.

Iste sit qui voluptatibus quisquam sed non. Et nesciunt excepturi dolores. Et ipsam aut ut.

Eius nihil ea non ut placeat est dolorem. Cupiditate sit et labore consequatur et et deserunt. Tempore est exercitationem voluptatum. Ut quis harum consequatur omnis. Officia qui deleniti accusamus excepturi quibusdam.

Aliquam reprehenderit qui fugit. Eos illo doloribus exercitationem. Odio nulla similique quisquam ipsam autem voluptatem et omnis.

Odit maiores sequi aliquam ab hic qui rerum. Sunt laboriosam quis natus nesciunt adipisci dolorum veniam quidem. Mollitia placeat accusamus voluptates officiis aut et. Dolore eum consequatur veniam.

Eos consequatur quis voluptatem provident eligendi molestiae laudantium nisi. Exercitationem nulla culpa commodi repudiandae praesentium laboriosam ea laudantium. Qui adipisci nihil adipisci quia aut.

Ut hic labore facilis odit eius consequuntur corporis officiis. Reprehenderit dolor eum commodi maxime animi quis voluptate perferendis. Quod et soluta excepturi recusandae voluptatem. Quia commodi porro autem vitae enim. Voluptatum optio et aut harum harum saepe.

Laborum quaerat asperiores dolorem enim aut et est. Soluta asperiores et tenetur nisi laboriosam sed at rerum. Et hic qui officia fuga perferendis officia dolore. Dolorem a corporis et fugiat consequuntur perspiciatis culpa voluptates. Voluptas qui laboriosam tempora explicabo corrupti voluptatem. Velit harum animi velit modi molestiae hic enim ducimus.

Sed sed incidunt ipsa et. Vitae ipsam et et assumenda. Optio rerum ab iusto consequatur.

Ut qui dolore accusantium sequi aut reprehenderit distinctio. Iste vel beatae quis. Dignissimos ut earum quia quaerat doloremque.

Consequatur facere tenetur et sit officia. Eos iure in repellat soluta rerum illum optio ea. Voluptatibus nihil qui provident consequuntur consequatur dolor tenetur dolorum. Repellat consequatur est tempore voluptate quia necessitatibus praesentium maiores. Numquam aut sit illo. Temporibus modi molestiae et corrupti odit quia nihil placeat.

Consequatur deleniti fugit impedit est. Consequatur tempore mollitia non et eos tempora quas. Aliquid cum autem error repudiandae dolorem. Et est aliquid et tempore in magnam est. Harum dignissimos vero tempora animi.

Sunt ipsam iure nam beatae voluptatem. Porro dolorem sit repudiandae ex est ex facere molestias. Ad animi quis et ad natus consequatur aut aut. Veritatis et magnam optio debitis necessitatibus. Dolore dolore explicabo ad quam id.

\begin{table}[t]
    \centering
    % \begin{tabular}{c*{5}{S[table-format=4.3,table-number-alignment=center]}}
    \begin{tabular}{c*{5}{S}}
        \toprule
        \mr{Objekt} &
        %%%%%%%%%%%%%%%%%%%%%%%%%%%%%%%%%%%%%%%%%%%%%%%%%%%%%%%%%%%%%%%%%%%%%
        % NOTE: second and third `\multirow` enclosed in {...} !!!          %
        %%%%%%%%%%%%%%%%%%%%%%%%%%%%%%%%%%%%%%%%%%%%%%%%%%%%%%%%%%%%%%%%%%%%%
        {\mr{$\delta$}} &
        {\mr{$p_e$}} &
        \multicolumn{2}{c}{$A_e/\left(10^{65}\textrm{erg}^{-1}\right)$} \\
        \cline{4-5}
        & & &
        \mc{$B=10\si{\micro G}$} &
        \mc{$B=100\si{\micro G}$} \\
        \hline
        NGC 1068 & -0.69 & 2.38 & 2100 & 4.2 \\
        NGC 4945 & -0.59 & 2.18 & 3.8 & 0.096 \\
        NGC \enspace 253 & -0.65 & 2.31 & 5.2 & 0.12 \\
        NGC 3034 & -0.39 & 1.78 & 0.12 & 0.0049 \\
        \bottomrule
    \end{tabular}
    \caption{Experimentelle Daten... \emph{(verbessertes Layout)}}
    \label{tab:tab4}
\end{table}


\newpage
\section{Aufgabe 7: Matheumgebungen in \LaTeX}
\textbf{TODO...}


\newcommand{\nextpart}{%
    \bigskip
    \noindent\rule{\textwidth}{1pt}
    \bigskip
}

\newcommand{\dn}[1]{%
    \ensuremath{\mathrm{d}^{#1}}
}

\newcommand{\TODO}[1]{%
    \textcolor{red}{\textbf{TODO} #1}
}

\newpage
\section{Aufgabe 8: Befehle definieren}
\textbf{Das neudefinierte Kommando}:
\begin{verbatim}
\newcommand{\dn}[1]{%
    \ensuremath{\mathrm{d}^{#1}}
}
\end{verbatim}

\textbf{Beispiel:} \textsw{dn[5]} $\Rightarrow \dn{5}$

\nextpart

\textbf{Ein TODO-Kommando}:
\begin{verbatim}
\newcommand{\TODO}[1]{%
    \textcolor{red}{\textbf{TODO} #1}
}
\end{verbatim}

\textbf{Beispiel:} \TODO{überlege ein gutes Beispiel}

\nextpart

\todo{QUACK!}

Mit dem Paket \textsw{todonotes} werden Notizen bspw.~bei folgenden
Konfigurationen entfernt:
\begin{verbatim}
\usepackage[...,draft,...]{scrartcl}
...
\usepackage[obeyDraft]{todonotes}
\end{verbatim}
oder
\begin{verbatim}
\usepackage[disable]{todonotes}
\end{verbatim}


\newpage
\section{Aufgabe 9: Zitieren mit biblatex}
\textbf{TODO}


\newpage
\section{Aufgabe 10: Anhänge in \LaTeX}
\textbf{TODO}


\end{document}
