\newpage
\section{Aufgabe 7: Matheumgebungen in \LaTeX}
\begin{equation}
    x^2 + p\,x + q = 0 \Rightarrow x_{1/2} = \frac{-p \pm \sqrt{p^2 - 4\,q}}{2}
\end{equation}

\begin{equation}
    f(x) = \left|x-3\right| =
    \begin{cases}
        x-3 & \quad \text{für } x < 3 \\
        -x+3 & \quad \text{für } x \ge 3
    \end{cases}
\end{equation}

\begin{equation}
    \sum_{k=1}^n k = \frac{n\,(n+1)}{2}
\end{equation}

\begin{subequations}
\begin{align}
    \nabla \cdot \vec{E} &= \frac{\rho}{\epsilon_0} \\
    \nabla \cdot \vec{B} &= 0 \\
    \nabla \times \vec{E} &= -\mu_0 \frac{\partial \vec{B}}{\partial t} \\
    \nabla \times \vec{B} &= \frac{1}{c^2}\frac{\partial \vec{E}}{\partial t} +
    \mu_0\,\vec{J}
\end{align}
\end{subequations}

\begin{equation}
    \mathbf{A}\,\vec{b} = \vec{x} =
    \begin{pmatrix}
        a_{1,1} & a_{1,2} & \cdots & a_{1,n} \\
        a_{2,1} & a_{2,2} & \cdots & a_{2,n} \\
        \vdots & \vdots & \ddots & \vdots \\
        a_{m,1} & a_{m,2} & \cdots & a_{m,n}
    \end{pmatrix}
    \begin{pmatrix}
        b_1 \\ b_2 \\ \vdots \\ b_n
    \end{pmatrix}
    =
    \begin{pmatrix}
        x_1 \\ x_2 \\ \vdots \\ x_m
    \end{pmatrix}
\end{equation}

\begin{equation}
    \int_0^\infty \underbrace{1}_{u'(x)} \, \underbrace{\ln(x)}_{v(x)}
    \mathrm{d}x = \underbrace{x}_{u(x)} \, \underbrace{\ln(x)}_{v(x)}
    \bigg\rvert_0^\infty - \int_0^\infty \underbrace{x}_{u(x)} \,
    \underbrace{\frac{1}{x}}_{v'(x)} \mathrm{d}x
\end{equation}

