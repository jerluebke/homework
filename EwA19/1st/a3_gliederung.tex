%! TEX program = lualatex
\documentclass[11pt,a4paper]{scrartcl}
\usepackage[utf8]{inputenc}
\usepackage{polyglossia}
    \setdefaultlanguage{german}
\usepackage{lmodern}

\newcommand{\setpageno}[1]{%
    \setcounter{page}{#1}
    \newpage
}

\title{Biologie Kram}
\date{\today}
\author{Author McAuthorface}

\begin{document}

\maketitle
\thispagestyle{empty}
\newpage
% \clearpage

\tableofcontents
\thispagestyle{empty}
\newpage

\section{Einleitung}
\setpageno{1}

\section{Zum Organ}
\subsection{Allgemeiner Aufbau des Organs}
\subsubsection{Vorgehen bei der Vermessung}
\setpageno{2}
\subsubsection{Schema (Ergebnis der Vermessung)}
\subsection{Allgemeine Funktion des Organs}
\setpageno{5}
\subsection{Tierartspezifische Besonderheiten - Fallbeispiele aus verschiedenen ökologischen Nischen}
\subsubsection{Erste Tierart}
\setpageno{7}
\subsubsection{Zweite Tierart}
\setpageno{8}
\subsubsection{Dritte Tierart}
\setpageno{9}

\section{Verteilung von Rezeptoren und Neuronen in ausgewählten Arealen}
\setpageno{10}
\subsection{Vorgehen zur Dichtebestimmung}
\subsubsection{Rezeptoren}
\setpageno{11}
\subsubsection{Neuronen}
\setpageno{13}
\subsection{Verteilung von Rezeptoren und Neuronen in Tierart X (Schema)}
\setpageno{15}

\section{Diskussion der Ergebnisse}
\setpageno{18}

\section{Literatur}
\setpageno{19}

\setcounter{section}{0}
\renewcommand{\thesection}{\Alph{section}}

\section{Anhang}
\subsection{Mikroton}
\subsubsection{Aufbau}
\setpageno{20}

\subsubsection{Bedienung}
\setpageno{24}

\subsection{Technische Details der Auswertung}
\subsubsection{Koordinatensystem zur Rekonstruktion der Gewebestruktur}
\setpageno{28}

\subsubsection{Zur Auswertung verwendetes Programm}
\setpageno{32}

\end{document}
