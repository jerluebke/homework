%! TEX program = lualatex
\documentclass[11pt,a4paper,abstraction,notitlepage]{scrartcl}
\usepackage[utf8]{inputenc}
\usepackage{polyglossia}
    \setdefaultlanguage{german}
\usepackage{lmodern}

\subject{Eine Datenstruktur zur Modellierung der Bewegung geladener Teilchen im
Interstellaren Medium}
\title{}
\date{\today}
\author{Jeremiah Lübke}

\begin{document}

\maketitle

\begin{abstract}
Während Struktur und Zusammensetzung des Interstellaren Mediums gut verstanden
sind, bleiben die zurückgelegten Wege hochenergetischer geladener Teilchen
durch die turbolenten Magnetfelder des ISM weiterhin rätselhaft. Eine
Bestimmung der Quellen gemessener Teilchen erschwert sich dadurch deutlich. \\
Eine Möglichkeit um die zugrunde liefenden Prozesse besser zu verstehen bietet
die stochastische Modellierung jener Magnetfelder, was allerdings für große
Bereiche mit sehr vielen Gridpunkten einen immensen Rechenaufwand bedeutet.
Um diese Herausforderung anzugehen wurde eine Datenstruktur entwickelt, mit
welcher für die Berechnung der Magnetfelder und Teilchenbewegungen nur die
relevanten Unterbereiche berücksichtigt werden müssen. \\
Die Vorstellung dieser Datenstruktur und ein einfaches Anwendungsbeispiel sind
Gegenstand des Vortrags.
\end{abstract}

\thispagestyle{empty}

\end{document}
