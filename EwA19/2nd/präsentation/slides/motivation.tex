\section{Motivation}
\begin{frame}
    \frametitle{Motivation}
    \textbf{Consider the following...}
    \begin{itemize}
        \item A supernova emits a large amount of charged particles into the
            interstellar medium
        \item After their long journey through the turbolent magnetic field of
            the ISM some of those particles hit the Earth and can be measured
        \item But the scientists cannot easily determine the particles source,
            since there are many possible sources and the particles paths are
            just as turbolent as the magnetic field, through which they
            travelled
    \end{itemize}

    \textbf{The Problem...}
    \begin{itemize}
        \item Modelling the magnetic field of the ISM can be done
            stochastically for each grid point
        \item But the domain is very large...
    \end{itemize}
\end{frame}

\begin{frame}
    \textbf{How do we solve this?}
    \begin{itemize}
        \item One possibility: Trace the particles and compute the bfield only
            in the relevant areas
    \end{itemize}

    \textbf{Idea}
    \begin{itemize}
        \item use two grids:
        \begin{enumerate}
            \item fine particle grid
            \item coarse box grid
        \end{enumerate}
        \item A Box is a cell on the coarse grid containing $N^3$
            particle-gridpoints
        \item do bookkeeping concerning the occupied boxes (starting from a
            known configuration)
        \item compute magnetic field only in the occupied boxes
    \end{itemize}
\end{frame}

% vim: set ff=unix tw=79 sw=4 ts=4 et ic ai :
