\documentclass[11pt,a4paper]{scrartcl}
\usepackage[top=2.5cm,bottom=5cm,left=2cm,right=2cm]{geometry}
\usepackage{fontspec}
\usepackage{polyglossia}
    \setdefaultlanguage{english}
\usepackage{lmodern}
\usepackage{fancyhdr}
\usepackage{csquotes}
\usepackage{enumitem}
\usepackage{mathtools}
\usepackage{amssymb}
\usepackage{amsfonts}
\usepackage{siunitx}
    \sisetup{range-units=brackets}
\usepackage{physics}
\usepackage{wasysym}
\usepackage{booktabs}
\usepackage[%
    labelformat=simple,
    labelsep=none,
    textformat=empty,
    font={small,sc}
]{caption}
\usepackage{graphicx}
    \graphicspath{img}
\usepackage{pgfplots}
    \pgfplotsset{%
        compat=1.16,
        table/search path={data},
        label style={font=\tiny},
        tick label style={font=\tiny}
    }
\usepackage{todonotes}
\usepackage[%
    % colorlinks=true, linkcolor=blue,
    hidelinks
]{hyperref}


\newcommand{\tablehead}[1]{\multicolumn{1}{c}{#1}}
\newcommand*{\figref}[1]{(fig.~\ref{#1})}
\newcommand{\eg}{e.\,g.}

\newcommand{\course}{\textbf{Statistical Mechanics}}
\newcommand{\hwnumber}{2}
\newcommand{\nameA}{Jeremiah Lübke}
\newcommand{\matnumA}{108015230366}


\pagestyle{fancyplain}

\headheight 5\baselineskip
\lhead{%
    \nameA \\
    \matnumA
    \vspace*{2\baselineskip}
}

\chead{%
    \headrule
    \vspace*{\baselineskip}
    \textbf{\Large \course} \\\medskip
    \textbf{\Large Homework \hwnumber}
    \bigskip
    % \vspace*{\baselineskip}
}

\rhead{%
    \today
    \vspace*{3\baselineskip}
}

\cfoot{}

\headsep 1.5em


\newcommand{\nbar}{\ensuremath{\bar{n}}}


\begin{document}

\section*{H\hwnumber.1}

\begin{enumerate}[label=\textbf{\large(\alph*)}, itemsep=2\baselineskip]

\item
    The Hamiltonian of the system is given by
    $
    % \begin{equation*}
        H=\sum_{i=1}^{N}\hbar\omega\left(\hat{a}_{i}^{\dagger}\hat{a}_{i}+\frac{1}{2}\right)
    % \end{equation*}
    $
    where the \emph{number operator}
    $\hat{n}_{i}=\hat{a}_{i}^{\dagger}\hat{a}_{i}$ can be understood as
    yielding the number of excitations of the $i$-th oscillator (its energy
    level) $n_i$. The total energy of the system
    $E=\hbar\omega\left(M+\frac{N}{2}\right)$ imposes the additional condition
    $\sum_{i=1}^{N}n_i=M$. \\
    The above described setup could be visualized as $M$ \enquote{balls} (the
    single excitations) being distributed on $N$ intervals (the single
    oscillators), which are defined by $N-1$ seperators. Then finding the
    number of possible microstates for a given energy (encoded by $M$) is
    equivalent to finding the number of permutations of those $M$ balls and
    their $N-1$ seperators
    % : $(M+N-1)!$. Additionally,
    and
    taking into account their
    indistinguishability:
    % , one needs to divide by $M!(N-1)!$:
    \begin{equation*}
        % \implies
        \Omega = \frac{(M+N-1)!}{M!\,(N-1)!}
    \end{equation*}

\item
    Using Stirling's formula $N!\sim\left(\frac{N}{e}\right)^N$, $N-1\approx N$
    and $\nbar=\frac{M}{N}$:
    \begin{align*}
        \log(\Omega) =& \log(M+N-1)!-\log M!-\log(N-1)! \\
        \approx&
        (M+N-1)\log(M+N-1)-M-N+1\\&-M\log(M)+M-(N-1)\log(N-1)+N-1 \\
        \approx& M\log(\frac{M+N}{M})+N\log(\frac{M+N}{N}) \\
        =& N\nbar\log(\frac{\nbar+1}{\nbar})+N\log(\nbar+1)
    \end{align*}
    The entropy is then:
    \begin{equation*}
        S = k\log(\Omega)
        = k\,N\left(\nbar\log(\frac{\nbar+1}{\nbar})+\log(\nbar+1)\right)
    \end{equation*}

\item
    Making use of the chain rule:
    \begin{align*}
        \frac{1}{kT} = \frac{1}{k}\pdv{S}{E}
        &= \pdv{\nbar}{E}\pdv{\log(\Omega)}{\nbar} \\
        &= \frac{1}{N\hbar\omega}\left(N\log(\frac{\nbar+1}{\nbar})
        +N\nbar\left(-\frac{1}{\nbar^2}\right)\frac{\nbar}{\nbar+1}
        +N\frac{1}{\nbar+1}\right)\\
        &= \frac{1}{\hbar\omega}\log(\frac{\nbar+1}{\nbar})
    \end{align*}
    After inserting $\nbar=\frac{E}{N\hbar\omega}-\frac{1}{2}$, one can solve
    for the energy:
    \begin{align*}
        \frac{\hbar\omega}{kT} &=
        \log(1+\frac{1}{\frac{E}{N\hbar\omega}-\frac{1}{2}}) \\
        \implies \Aboxed{%
            E &=
            N\hbar\omega\left(\frac{1}{e^{\frac{\hbar\omega}{kT}}-1}+\frac{1}{2}\right)
        }
    \end{align*}

\end{enumerate}


\end{document}
