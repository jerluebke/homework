\documentclass[11pt,a4paper]{scrartcl}
\usepackage[top=2.5cm,bottom=5cm,left=2cm,right=2cm]{geometry}
\usepackage{fontspec}
\usepackage{polyglossia}
    \setdefaultlanguage{english}
\usepackage{lmodern}
\usepackage{fixcmex}
\usepackage{fancyhdr}
\usepackage{csquotes}
\usepackage{enumitem}
\usepackage{mathtools}
\usepackage{amssymb}
\usepackage{amsfonts}
\usepackage{siunitx}
    \sisetup{range-units=brackets}
\usepackage{physics}
\usepackage{textcomp}
\usepackage{gensymb}
\usepackage{wasysym}
\usepackage{array}
\usepackage{booktabs}
\usepackage[%
    labelformat=simple,
    labelsep=none,
    textformat=empty,
    font={small,sc}
]{caption}
\usepackage{graphicx}
    \graphicspath{img}
\usepackage{pgfplots}
    \pgfplotsset{%
        compat=1.16,
        table/search path={data},
        label style={font=\tiny},
        tick label style={font=\tiny}
    }
\usepackage[makeroom]{cancel}
\usepackage{todonotes}
\usepackage[%
    % colorlinks=true, linkcolor=blue,
    hidelinks
]{hyperref}


\newcommand{\tablehead}[1]{\multicolumn{1}{c}{#1}}
\newcommand*{\figref}[1]{(fig.~\ref{#1})}
\newcommand{\eg}{e.\,g.}
\newcommand{\ie}{i.\,e.}

\newcommand{\course}{\textbf{Statistical Mechanics}}
\newcommand{\hwnumber}{8}
\newcommand{\nameA}{Jeremiah Lübke}
\newcommand{\matnumA}{108015230366}


\pagestyle{fancyplain}

\headheight 5\baselineskip
\lhead{%
    \nameA \\
    \matnumA
    \vspace*{2\baselineskip}
}

\chead{%
    \headrule
    \vspace*{\baselineskip}
    \textbf{\Large \course} \\\medskip
    \textbf{\Large Homework \hwnumber}
    \bigskip
}

\rhead{%
    \today
    \vspace*{3\baselineskip}
}

\cfoot{\small\thepage}

\headsep 1.5em

\renewcommand{\thesection}{H\hwnumber.\arabic{section}}


\newcommand{\gs}{g_{s}}


\begin{document}

\section{}

Consider the hyperrelativistic fermi gas with $\epsilon_{p}=p\,c$.

\begin{enumerate}[label=\textbf{\large(\alph*)}, itemsep=2\baselineskip]

\item
    In order to compute the grand potential $\Omega=-k\,T\log{Z}$ consider at
    first the partition function for fermions:
    \begin{equation*}
        Z=\prod_{\nu}\delta_{\nu\nu}\left(1+e^{\beta(\mu-\epsilon_{\nu})}\right)
    \end{equation*}
    \begin{align*}
        \log{Z}&=-k\,T\sum_{nu}\delta{\nu\nu}\log(1+e^{\beta(\mu-\epsilon_{\nu})})
        \\
        &=\underbrace{\sum_{s}}_{=\gs}\int\dd[3]{p}\frac{V}{(2\pi\hbar)^3}
        \log(1+e^{\beta(\mu-pc)}) \\
        &=\frac{\gs\,V}{2\pi^2\hbar^3}\int\limits_{0}^{\infty}\dd{p}
        \underbrace{p^2}_{u'}
        \underbrace{\log(1+e^{\beta(\mu-pc)})}_{v} \\
        &=\frac{\gs\,V}{2\pi^2\hbar^3}\frac{\beta\,c}{3}\int\limits_{0}^{\infty}\frac{p^3\dd{p}}{e^{\beta(pc-\mu)}+1}
    \end{align*}
    Continue with the integral
    $\displaystyle
        I=\int\limits_{0}^{\infty}\frac{p^3\dd{p}}{e^{\beta(pc-\mu)}+1}
    $.
    Substituting $z=\beta(p\,c-\mu)$ yields:
    \begin{align*}
        (c\beta)^4\,I&=\int\limits_{-\beta\mu}^{\infty}\frac{(z+\beta\mu)^3}{e^z+1}\dd{z}
        \\
        &=\int\limits_{0}^{\infty}\frac{(\beta\mu+z)^3}{e^z+1}\dd{z}+\int\limits_{0}^{\beta\mu}\frac{(\beta\mu-z)^3}{e^{-z}+1}\dd{z}
        \qq{with}\frac{1}{e^{-z}+1}=1-\frac{1}{e^z+1}\qq{and}\beta\mu\gg{1} \\
        &=\int\limits_{0}^{\beta\mu}(\beta\mu-z)^3\dd{z}+\int\limits_{0}^{\infty}\frac{(\beta\mu+z)^3-(\beta\mu-z)^3}{e^z+1}\dd{z}
    \end{align*}
    The first integral is easily evaluated:
    \begin{equation*}
        \int\limits_{0}^{\beta\mu}(\beta\mu-z)^3\dd{z}=\int\limits_{0}^{\beta\mu}w^3\dd{w}=\frac{(\beta\mu)^4}{4}
    \end{equation*}
    And for the second one work out the numerator of the fraction
    \begin{equation*}
        (\beta\mu+z)^3-(\beta\mu-z)^3=6\,(\beta\mu)^2\,z+2\,z^3
    \end{equation*}
    and use
    \begin{gather*}
        \int\limits_{0}^{\infty}\frac{z^{a-1}}{e^z+1}\dd{z}=(1-2^{1-a})\,\Gamma(a)\,\zeta(a)
        \\
        \implies\int\limits_{0}^{\infty}\frac{z\dd{z}}{e^z+1}=\frac{\pi^2}{12}
        \qq*{,}
        \int\limits_{0}^{\infty}\frac{z^3\dd{z}}{e^z+1}=\frac{7\pi^4}{120}
    \end{gather*}
    Therewith:
    \begin{gather*}
        (c\beta)^4\,I=\frac{(\beta\mu)^4}{4}+\frac{(\beta\mu\pi)^2}{2}+\frac{7\pi^4}{60}
        \\
        \implies\log{Z}=\frac{\gs{V}\beta\mu^4}{24\pi^2\hbar^3c^3}\left(1+\frac{2\pi^2}{(\beta\mu)^2}+\frac{7}{15}\left(\frac{\pi}{\beta\mu}\right)^4\right)
        \\
        \implies\boxed{%
            \Omega=-k\,T\log{Z}=-\frac{\gs{V}\mu^4}{24\pi^2\hbar^3c^3}\left(1+\frac{2(\pi{k}T)^2}{\mu^2}+\order{T^4}\right)
        }
    \end{gather*}


\item
    For the thermal equation of state, integrate the energy per state with
    respect to the state number:
    \begin{gather*}
        E=\int\limits_{pc<\mu}\epsilon\dd{N}
        =\frac{\gs{V}c}{(2\pi\hbar)^3}4\pi
        \underbrace{%
            \int\limits_{0}^{\mu/c}\dd{p}p^3
        }_{=\frac{\mu^4}{4c^3}}
        =\frac{\gs{V}\mu^4}{8\pi^2\hbar^3c^3} \\
        \implies\Omega=-p\,V=-\frac{\gs{V}\mu^4}{24\pi^2\hbar^3c^3}=-\frac{1}{3}E
        \\
        \implies\boxed{3\,p\,V=E}
    \end{gather*}


\item
    In order to find the heat capacity, one needs to take the first correction
    into account (in the following: $C=C_V\approx C_p$, since
    $C_V-C_p\sim\order{T^3}$):
    \begin{gather*}
        S=-\pdv{\Omega}{T}=\frac{\gs{V}\mu^2k^2}{6\hbar^3c^3}T \\
        \boxed{C=T\pdv{S}{T}=\frac{\gs{V}\mu^2k^2}{6\hbar^3c^3}T}
    \end{gather*}

\end{enumerate}


\end{document}
