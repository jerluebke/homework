\documentclass[11pt,a4paper]{scrartcl}
\usepackage[top=2.5cm,bottom=5cm,left=2cm,right=2cm]{geometry}
\usepackage{fontspec}
\usepackage{polyglossia}
    \setdefaultlanguage{english}
\usepackage{fancyhdr}
\usepackage{csquotes}
\usepackage{enumitem}
\usepackage{mathtools}
\usepackage{amssymb}
\usepackage{amsfonts}
\usepackage{siunitx}
    \sisetup{range-units=brackets}
\usepackage{physics}
\usepackage{wasysym}
\usepackage{booktabs}
\usepackage[%
    labelformat=simple,
    labelsep=none,
    textformat=none,
    font={small,sc}
]{caption}
\usepackage{graphicx}
    \graphicspath{img}
\usepackage{pgfplots}
    \pgfplotsset{%
        table/search path={data},
        label style={font=\tiny},
        tick label style={font=\tiny}
    }
\usepackage[makeroom]{cancel}
\usepackage{todonotes}
\usepackage[%
    % colorlinks=true, linkcolor=blue,
    hidelinks
]{hyperref}


\newcommand{\tablehead}[1]{\multicolumn{1}{c}{#1}}
\newcommand*{\figref}[1]{(fig.~\ref{#1})}
\newcommand{\eg}{e.\,g.}

\newcommand{\course}{\textbf{Statistical Mechanics}}
\newcommand{\hwnumber}{1}
\newcommand{\nameA}{Jeremiah Lübke}
\newcommand{\matnumA}{108015230366}


\pagestyle{fancyplain}

\headheight 5\baselineskip
\lhead{%
    \nameA \\
    \matnumA
    \vspace*{2\baselineskip}
}

\chead{%
    \headrule
    \vspace*{\baselineskip}
    \textbf{\Large \course} \\\medskip
    \textbf{\Large Homework \hwnumber}
    \bigskip
}

\rhead{%
    \today
    \vspace*{3\baselineskip}
}

\cfoot{}

\headsep 1.5em

\setlength{\parindent}{0pt}


\begin{document}

\section*{H1.1}

Consider the Hamiltonian of the microcanonical ensemble of N harmonic oscillators:
\begin{equation*}
    H(p, q) = \sum_{i=1}^{N}\frac{\vec{p}_i^2}{2m} +
    \frac{m\omega^2}{2}\sum_{i=1}^{N}\vec{q}_i^2
\end{equation*}

The statistical weight with the phase space volume element:
\begin{equation*}
    \Omega = \lim_{\Delta E \to 0}\frac{1}{\Delta E} \int_{E\leq H(p,q)\leq
    E+\Delta E}\dd\Gamma, \quad
    \dd\Gamma = \frac{1}{N!}\prod_{i=1}^{N}\frac{\dd[3]p_i\dd[3]q_i}{(2\pi\hbar)^{3}}
\end{equation*}

In order to compute $\Omega$ for the given Hamiltonian, consider the
integral (the $\delta$ appears after taking the limit $\Delta E \to 0$):
\begin{align*}
    \lim_{\Delta E \to 0}\frac{1}{\Delta E}\int_{E\leq H(p,q)\leq E+\Delta E}\dd\Gamma &=
    \frac{1}{N!}\frac{1}{(2\pi\hbar)^{3N}}\int\dd[3N]{p}\dd[3N]{q}
    \delta\left(H(p,q)-E\right) \\
    &= \frac{1}{N!}\frac{1}{(2\pi\hbar)^{3N}}\int\dd[3N]{p}\dd[3N]{q}
    \delta\left(\sum_{i=1}^{N}\left(\frac{\vec{p}_i^2}{2m}+\frac{m\omega^2\vec{q}_i^2}{2}\right)-E\right)
\end{align*}

Substitute
\begin{equation*}
    \widetilde{\vec{p}_i}\longrightarrow\frac{\vec{p}_i}{\sqrt{2m}}, \quad
    \widetilde{\vec{q}_i}\longrightarrow\sqrt{\frac{m}{2}}\omega\vec{q}_i
    \implies \dd[3N]{p} = \left(2m\right)^{\frac{3N}{2}}\dd[3N]{\widetilde{p}},
    \quad \dd[3N]{q} = \left(\frac{2}{m\omega^2}\right)^{\frac{3N}{2}}\dd[3N]{\widetilde{q}}
\end{equation*}
and drop the prefactor for now:
\begin{equation*}
    \int\dd[3N]{p}\dd[3N]{q}\delta\left(H(p,q)-E\right) =
    \left(\frac{2}{\omega}\right)^{3N}
    \int\dd[3N]{\widetilde{p}}\dd[3N]{\widetilde{q}}
    \delta\Bigg(
    \underbrace{\sum_{i=1}^{N}\left(\widetilde{\vec{p}}_i^2+\widetilde{\vec{q}}_i^2\right)}_{\equiv R^2}
    -E\Bigg) =
\end{equation*}

Switch to spherical coordinates and apply the relation
$\delta(x^2-a^2)=\frac{1}{2\abs{a}}\left(\delta(x-a)+\delta(x+a)\right)$:
\begin{equation*}
    = \int R^{6N-1}\dd{R}\dd{\Omega_{6N-1}}\frac{1}{2\sqrt{E}}\Big(\delta\left(R-\sqrt{E}\right)
    +\cancelto{0,\ R\geq 0}{\delta\left(R+\sqrt{E}\right)}\Big)
    = \frac{1}{2}\int\dd{\Omega_{6N-1}}E^{3N-1}
\end{equation*}

Use the equation for the D-dimensional sphere to obtain:
\begin{equation*}
    V_D(R) = \int_0^R r^{D-1}\dd{r} \int\dd{\Omega_{D-1}} = \frac{R^D}{D} \int\dd{\Omega_{D-1}}
    = \frac{2\pi^{\frac{D}{2}}R^D}{D\,\Gamma\left(\frac{D}{2}\right)}
    \implies \int\dd{\Omega_{D-1}} = \frac{2\pi^{\frac{D}{2}}}{\Gamma\left(\frac{D}{2}\right)}
\end{equation*}

Finally, put everything together:
\begin{gather*}
    \Omega=\frac{1}{N!}\frac{1}{(2\pi\hbar)^{3N}}\left(\frac{2}{\omega}\right)^{3N}
    \frac{E^{3N-1}}{2}\frac{2\pi^{3N}}{\Gamma(3N)} \\
    \implies\boxed{%
        % \Omega&=\frac{1}{N!}\frac{1}{(\hbar\omega)^{3N}}\frac{E^{3N-1}}{\Gamma(3N)}
        \Omega=\frac{E^{3N-1}}{N!(\hbar\omega)^{3N}\Gamma(3N)}
    }
\end{gather*}


\end{document}
