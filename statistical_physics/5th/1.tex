\documentclass[11pt,a4paper]{scrartcl}
\usepackage[top=2.5cm,bottom=5cm,left=2cm,right=2cm]{geometry}
\usepackage{fontspec}
\usepackage{polyglossia}
    \setdefaultlanguage{english}
\usepackage{lmodern}
\usepackage{fixcmex}
\usepackage{fancyhdr}
\usepackage{csquotes}
\usepackage{enumitem}
\usepackage{mathtools}
\usepackage{amssymb}
\usepackage{amsfonts}
\usepackage{siunitx}
    \sisetup{range-units=brackets}
\usepackage{physics}
\usepackage{textcomp}
\usepackage{gensymb}
\usepackage{wasysym}
\usepackage{array}
\usepackage{booktabs}
\usepackage[%
    labelformat=simple,
    labelsep=none,
    textformat=empty,
    font={small,sc}
]{caption}
\usepackage{graphicx}
    \graphicspath{img}
\usepackage{pgfplots}
    \pgfplotsset{%
        compat=1.16,
        table/search path={data},
        label style={font=\tiny},
        tick label style={font=\tiny}
    }
\usepackage[makeroom]{cancel}
\usepackage{todonotes}
\usepackage[%
    % colorlinks=true, linkcolor=blue,
    hidelinks
]{hyperref}


\newcommand{\tablehead}[1]{\multicolumn{1}{c}{#1}}
\newcommand*{\figref}[1]{(fig.~\ref{#1})}
\newcommand{\eg}{e.\,g.}
\newcommand{\ie}{i.\,e.}

\newcommand{\course}{\textbf{Statistical Mechanics}}
\newcommand{\hwnumber}{5}
\newcommand{\nameA}{Jeremiah Lübke}
\newcommand{\matnumA}{108015230366}


\pagestyle{fancyplain}

\headheight 5\baselineskip
\lhead{%
    \nameA \\
    \matnumA
    \vspace*{2\baselineskip}
}

\chead{%
    \headrule
    \vspace*{\baselineskip}
    \textbf{\Large \course} \\\medskip
    \textbf{\Large Homework \hwnumber}
    \bigskip
}

\rhead{%
    \today
    \vspace*{3\baselineskip}
}

\cfoot{\small\thepage}

\headsep 1.5em

\renewcommand{\thesection}{H\hwnumber.\arabic{section}}


\begin{document}

\section{Systems of identical particles}

The entropy is an extensive quantity,
\ie~$S(\mathrm{System}_1+\mathrm{System}_2)=S(\mathrm{System}_1)+S(\mathrm{System}_2)$.

Verify this property, first using the na\"ive partition function $Z_N'=(\alpha
kT)^\frac{3N}{2}V^N$ and then taking into account particles being indistinguishable
$Z_N=\frac{1}{N!}(\alpha kT)^\frac{3N}{2}V^N$, where $\alpha\equiv 2\pi m$ and for
simplicity $V_1=V_2=V$ and $N_1=N_2=N$.

The entropy can be obtained from both partition functions via:
\begin{align*}
    S&=-\pdv{F}{T}=\pdv{T}kT\log Z_N(T) \\
    &=k\log Z_N(T)+\frac{3}{2}kN
\end{align*}

\begin{enumerate}[label=\textbf{(\roman*)}]
    \item Using $Z_N'$:
        \begin{align*}
            &S(V_1, N_1)+S(V_2, N_2)=2S(V, N)
            =2Nk\log V+3Nk\log\alpha kT+3Nk \\
            &S(V_1+V_2, N_1+N_2)=S(2V, 2N)
            =2Nk\log 2V +3Nk\log\alpha kT+3Nk \\&\\
            &\implies 2S(V, N)\neq S(2V, 2N)
        \end{align*}

    \item On the other hand with $Z_N$ (using Sterling's formula):
        \begin{align*}
            &S=Nk\log V +\frac{3N}{2}k\log\alpha kT-Nk\log N +\frac{5N}{2}k
            \\&\\
            &2S(V, N)=2Nk\log\frac{V}{N}+3Nk\log\alpha kT+5Nk \\
            &S(2V, 2N)=2Nk\log\frac{2V}{2N}+3Nk\log\alpha kT+5Nk \\&\\
            &\implies \boxed{2S(V, N)=S(2V, 2N)}
        \end{align*}
\end{enumerate}

\newpage

\section{Relativistic particles}

Consider the Hamiltonian $H=\sqrt{p^2c^2+m^2c^4}$ and compute the partition
function:
\begin{align*}
    Z_N&=\frac{1}{N!}\int\prod_{j=1}^{N}\frac{\dd[3]{p_j}\dd[3]{q_j}}{(2\pi\hbar)^3}
    e^{-\beta c\sqrt{p^2+m^2c^2}}\\
    &=\frac{1}{N!}\left(\frac{4\pi V}{(2\pi\hbar)^3}
    \int\limits_{0}^{\infty}\dd{p}p^2e^{-\beta c^2 m\sqrt{\frac{p^2}{m^2
    c^2}+1}}\right)^N
\end{align*}
Using:
\begin{equation*}
    \int\limits_{0}^{\infty}\dd{p}p^2e^{-a\sqrt{p^2+1}}
    =\int\limits_{1}^{\infty}\dd{u}u\sqrt{u^2-1}e^{-au}
    =\frac{K_2(a)}{a}
\end{equation*}
where $K_{\alpha}$ is the modified Bessel function

\begin{equation*}
    \implies Z_N=\frac{1}{N!}\left(\frac{4\pi Vm^2c}{(2\pi\hbar)^3
    \beta}K_2(\beta m c^2)\right)^N
\end{equation*}
Using the asymptotical expansion:
\begin{equation*}
    K_2(x)\sim\sqrt{\frac{\pi}{2x}}e^{-x}\left(1+\frac{15}{8x}+\cdots\right)
    \qas x\to\infty
\end{equation*}
which yields:
\begin{gather*}
    \implies\boxed{Z_N^{\mathrm{rel}}=Z_N^{\mathrm{cl}}e^{-N\frac{mc^2}{kT}}\left(1+\frac{15kT}{8mc^2}\right)^N}
    \qas \frac{1}{c}\to0 \\
    \qq*{with}Z_N^{\mathrm{cl}}=\frac{1}{N!}\left[V\left(\frac{2\pi
    mkT}{(2\pi\hbar)^2}\right)^{3/2}\right]^N
\end{gather*}

\newpage

\section{Quantum corrections}

Compute the canonical partition function for the Hamiltonian
$\hat{H}=c\,\abs*{\hat{\vec{p}}}$ using properly symmetrized basis states:
\begin{align*}
    Z_N &= \tr e^{-\beta\hat{H}} \\
    &=
    \int_{V}\dd[3N]{r}\ev{e^{-\beta\hat{H}}}{\vec{r}_{1}\cdots\vec{r}_{N}} \\
    &= \int_{V}\dd[3N]{r}\int_{\mathbb{R}}\dd[3N]{p}\int_{\mathbb{R}}\dd[3N]{p'}
    \braket{\vec{r}_{1}\cdots\vec{r}_{N}}{\vec{p}_{1}\cdots\vec{p}_{N}}
    \mel{\vec{p}_{1}\cdots\vec{p}_{N}}{e^{-\beta\hat{H}}}{\vec{p'}_{1}\cdots\vec{p'}_{N}}
    \braket{\vec{p'}_{1}\cdots\vec{p'}_{N}}{\vec{r}_{1}\cdots\vec{r}_{N}} \\
    &= \frac{1}{N!}\sum_{\hat{P}\in\mathrm{Perms}}(\pm1)^{\eta_{\hat{P}}}
    \int_{V}\dd[3N]{r}\int_{\mathbb{R}}\dd[3N]{p}e^{-\beta
        E(\vec{p}_{1},\cdots,\vec{p}_{N})}
    \braket{\vec{r}_{1}\cdots\vec{r}_{N}}{\vec{p}_{1}\cdots\vec{p}_{N}}
    \braket{\vec{p}_{1}\cdots\vec{p}_{N}}{\vec{r}_{\hat{P}1}\cdots\vec{r}_{\hat{P}N}}
    \\
    &=
    \frac{1}{N!\,(2\pi\hbar)^{3N}}\sum_{\hat{P}\in\mathrm{Perms}}(\pm1)^{\eta_{\hat{P}}}
    \prod_{j=1}^{N}\int_{V}\dd[3]{r_j}\int_{\mathbb{R}}\dd[3]{p_j}
    \exp{-\beta\,c\,\abs{\vec{p}_j}
    +\frac{i}{\hbar}\,\vec{p}_{j}\cdot\left(\vec{r}_{j}-\vec{r}_{\hat{P}j}\right)}
\end{align*}

Evaluate the $p$-integral, using $\vec{p}\cdot\vec{r}=p\,r\,\cos\measuredangle(\vec{p},
\vec{r})$, letting $a=\beta\,c$ and
$b=\frac{\abs*{\vec{r}_{j}-\vec{r}_{\hat{P}j}}}{\hbar}$ and rotating the
coordinate system in such a way that $\theta=\measuredangle(\vec{p},
\vec{r}_{j}-\vec{r}_{\hat{P}j})$ holds:
\begin{align*}
    &\int_{\mathbb{R}}\dd[3]{p}\exp{-\beta\,c\,\abs{\vec{p}}+\frac{i}{\hbar}\,\vec{p}\cdot\left(\vec{r}_{j}-\vec{r}_{\hat{P}j}\right)}
    =\\&=
    \int\limits_{0}^{\infty}\int\limits_{-1}^{1}\int\limits_{0}^{2\pi}\dd{p}\dd{\cos\theta}\dd{\varphi}
    p^2\exp{-p\left(a-i\,b\cos\theta\right)} \\
    &= \frac{4\pi}{b}\int\limits_{0}^{\infty}\dd{p}p\,e^{-ap}\,\sin b\,p \\
    &= \frac{8\pi a}{\left(a^2+b^2\right)^2}
\end{align*}

Now one has the expression:
\begin{equation*}
    \sum_{\hat{P}\in\mathrm{Perms}}(\pm1)^{\eta_{\hat{P}}}\prod_{j=1}^{N}\int_V\dd[3]{r_j}
    f(\vec{r}_{j}-\vec{r}_{\hat{P}j})\qc
    f(\vec{r})=\frac{8\pi\hbar^4\beta c}{\left((\hbar\beta c)^2+\vec{r}^2\right)^2}
\end{equation*}

When expanding this device, the 0th order term is given by the identity
permutation $\hat{P}\equiv\mathrm{Id}$, which yields:
\begin{equation*}
    \prod_{j=1}^{N}\int_{V}\dd[3]{r_j}f(0)=\left(\frac{8\pi V}{(\beta
    c)^3}\right)^N
\end{equation*}

For the 1st order term consider the permutation, which only switches one pair
$\hat{P}i=
\begin{cases}
    a\qif i=b\\
    b\qif i=a\\
    i\qotherwise
\end{cases}
$:
\begin{gather*}
    \prod_{i=1}^{N}\int_{V}\dd[3]{r_j}f(\vec{r}_{i}-\vec{r}_{\hat{P}i})
    =\left(\frac{8\pi V}{(\beta
    c)^3}\right)^{N-2}\int_{V}\dd[3]{r_a}\int_{V}\dd[3]{r_b}f^2(\vec{r}_a-\vec{r}_b)
    \\
    \int_{V}\dd[3]{r_a}\int_{V}\dd[3]{r_b}f^2(\vec{r}_a-\vec{r}_b)\approx
    4\pi V\int\limits_{0}^{\infty}\dd{u}
    \frac{\left(8\pi\hbar^4\beta c\right)^2}{\left((\hbar\beta
    c)^2+u^2\right)^4} \\
    \qq*{with}
    \int\limits_{0}^{\infty}\frac{\dd{x}}{\left(1+x^2\right)^4}=\frac{5\pi}{32}\\
    \implies
    \int_{V}\dd[3]{r_a}\int_{V}\dd[3]{r_b}f^2(\vec{r}_a-\vec{r}_b)\approx
    \frac{40\pi^4\hbar V}{(\beta c)^5}
\end{gather*}

Since there are $\binom{N}{2}$ such permutations, one reveives (with
$N!\sim\left(\frac{N}{e}\right)^N$ and $\rho=\frac{N}{V}$):
\begin{align*}
    \binom{N}{2}\prod_{i=1}^{N}\int_{V}\dd[3]{r_j}f(\vec{r}_{i}-\vec{r}_{\hat{P}i})
    &=\binom{N}{2}\left(\frac{8\pi V}{(\beta
    c)^3}\right)^{N-2}\frac{40\pi^4\hbar V}{(\beta c)^5} \\
    &\approx\left(\frac{8\pi V}{(\beta c)^3}\right)^{N}\frac{5\pi\hbar\beta c
    N\rho}{16 e^2}
\end{align*}

Finally, one can write down the partition function with first order quantum
corrections:
\begin{gather*}
    \boxed{%
        Z_{N}^{\mathrm{qm}} = Z_{N}^{\mathrm{cl}}\left(1\pm\frac{5\pi\hbar c N\rho}{16e^2kT}\right)
    }\\
    \qq*{with} Z_{N}^{\mathrm{cl}}=\frac{1}{N!}\left(8\pi V\left(\frac{kT}{2\pi\hbar
    c}\right)^3\right)^N
\end{gather*}

\end{document}
