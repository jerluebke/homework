\documentclass[11pt,a4paper]{scrartcl}
\usepackage[top=2.5cm,bottom=5cm,left=2cm,right=2cm]{geometry}
\usepackage{fontspec}
\usepackage{polyglossia}
    \setdefaultlanguage{english}
\usepackage{lmodern}
\usepackage{fixcmex}
\usepackage{fancyhdr}
\usepackage{csquotes}
\usepackage{enumitem}
\usepackage{mathtools}
\usepackage{amssymb}
\usepackage{amsfonts}
\usepackage{siunitx}
    \sisetup{range-units=brackets}
\usepackage{physics}
\usepackage{textcomp}
\usepackage{gensymb}
\usepackage{array}
\usepackage{booktabs}
\usepackage[%
    labelformat=simple,
    labelsep=none,
    textformat=empty,
    font={small,sc}
]{caption}
\usepackage{graphicx}
    \graphicspath{img}
\usepackage{pgfplots}
    \pgfplotsset{%
        compat=1.16,
        table/search path={data},
        label style={font=\tiny},
        tick label style={font=\tiny}
    }
\usepackage[makeroom]{cancel}
\usepackage{todonotes}
\usepackage[%
    % colorlinks=true, linkcolor=blue,
    hidelinks
]{hyperref}


\newcommand{\tablehead}[1]{\multicolumn{1}{c}{#1}}
\newcommand*{\figref}[1]{(fig.~\ref{#1})}
\newcommand{\eg}{e.\,g.}
\newcommand{\ie}{i.\,e.}

\newcommand{\course}{\textbf{Statistical Mechanics}}
\newcommand{\hwnumber}{6}
\newcommand{\nameA}{Jeremiah Lübke}
\newcommand{\matnumA}{108015230366}


\pagestyle{fancyplain}

\headheight 5\baselineskip
\lhead{%
    \nameA \\
    \matnumA
    \vspace*{2\baselineskip}
}

\chead{%
    \headrule
    \vspace*{\baselineskip}
    \textbf{\Large \course} \\\medskip
    \textbf{\Large Homework \hwnumber}
    \bigskip
}

\rhead{%
    \today
    \vspace*{3\baselineskip}
}

\cfoot{\small\thepage}

\headsep 1.5em

\renewcommand{\thesection}{H\hwnumber.\arabic{section}}


\newcommand{\thermpart}[3]{\left(\pdv{#1}{#2}\right)_{#3}}


\begin{document}

\section{}

Consider the Hamiltonian:
\begin{equation*}
    H(r,p)=\sum_{i=1}^{N}\frac{\vec{p}_{i}^{2}}{2m}+\sum_{j=1}^{N}\sum_{i=1}^{j-1}U(\vec{r}_{i}-\vec{r}_{j})
\end{equation*}
with an interaction potential $U(\vec{r})=\frac{\alpha}{r^n}$
and compute the partition function:
\begin{align*}
    Z&=\frac{1}{N!}\int\prod_{i=1}^{N}\frac{\dd[3]{r_i}\dd[3]{p_i}}{(2\pi\hbar)^3}e^{-\beta H}
    \\
    &=\frac{1}{N!\lambda^{3N}}\int\prod_{i=1}^{N}\dd[3]{r_i}e^{-\beta\sum_{j<k}U(\vec{r}_{j}-\vec{r}_{k})}
    \qq{with}\lambda=\sqrt{\frac{\beta(2\pi\hbar)^2}{2\pi{m}}}
\end{align*} \\

The product $\displaystyle\prod_{j<k}e^{-\beta U_{jk}}$ is not easily expanded,
since its factors are close to 1 for large $r$. Instead consider
$f_{jk}=e^{-\beta U_{jk}}-1$:
\begin{equation*}
    \prod_{j<k}(1+f_{jk})=1+
    \underbrace{\sum_{j<k}f_{jk}}_{\binom{N}{2}\text{ summands}}
    +\sum_{\substack{j<k\\l<m}}f_{jk}f_{lm}+\dotsb
\end{equation*}
with
\begin{align*}
    \int\prod_{i=1}^{N}\dd[3]{r_i}\sum_{j<k}f_{jk}
    &=\sum_{j<k}\prod_{\substack{i\ne{j}\\i\ne{k}}}
    \int\dd[3]{r_i}\int\dd[3]{r_j}\int\dd[3]{r_k}f(\vec{r}_{j}-\vec{r}_{k}) \\
    &=\frac{N(N-1)}{2}V^{N-2}\iint\dd[3]{r_j}\dd[3]{r_k}f(\vec{r}_{j}-\vec{r}_{k})
    \\
    &\approx\frac{N(N-1)}{2}V^{N-2}\,V\int\dd[3]{r}f(r) \\
    &\approx-N^2\,V^{N-1}\,B(T)
\end{align*}
where $\displaystyle B(T)=-\frac{1}{2}\int\dd[3]{r}f(r)=
\frac{1}{2}\int\dd[3]{r}\left(1-e^{-\beta U(r)}\right)$ is the first
\emph{virial coefficient}. \\
Further:
\begin{equation*}
    \implies\int\prod_{i=1}^{N}\dd[3]{r_i}\prod_{j<k}(1+f_{jk})
    =V^N-N^2\,V^{N-1}\,B(T)+\dotsb
\end{equation*} \\

Finally, the complete partition function:
\begin{equation}
    \boxed{%
    Z=\frac{1}{N!}\left(\frac{V}{\lambda^3}\right)\left(1-\frac{N^2}{V}B(T)\right)}
    \label{eq:Z}
\end{equation}
and from $p=-\thermpart{F}{V}{T}$, $F=-kT\log{Z}$ the equation of state:
\begin{equation}
    \boxed{p\,V=N\,k\,T\left(1+\frac{N}{V}B(T)\right)}
    \label{eq:EOS}
\end{equation}


\begin{enumerate}[label=\textbf{\large(\alph*)}, itemsep=2\baselineskip]

\item
    Computing the first virial coefficient:
    \begin{equation*}
        B(T)=2\pi\int\limits_{0}^{\infty}\dd{r}r^2\left(1-e^{-\beta{U}(r)}\right)
    \end{equation*}

    Approximate the potential for suitable $r_0$:
    \begin{equation}
        U(r)=\frac{\alpha}{r^n}\approx
        \begin{cases}
            \infty\qc r<r_0 \\
            \frac{\alpha}{r^n}\qc r\geq r_0
        \end{cases}
        \label{eq:U}
    \end{equation}
    and expand the integrand around $r_0$ (where $\gamma\equiv\alpha\beta$):
    \begin{equation*}
        1-e^{-\frac{\gamma}{r^n}}=1-1+\frac{\gamma}{r^n}-\order{\frac{\gamma^2}{2r^{2n}}}
    \end{equation*}

    Then:
    \begin{equation}
        B(T)\approx
        2\pi\int\limits_{0}^{r_0}\dd{r}r^2+2\pi\gamma\int\limits_{r_0}^{\infty}\frac{\dd{r}}{r^{n-2}}
        \label{eq:Bint}
    \end{equation}

    From \eqref{eq:U} it is clear, that
    $r_0^2=\frac{\gamma}{r_{0}^{n-2}}\implies r_0=\gamma^{1/n}$. \\
    It is then straightforward to integrate \eqref{eq:Bint}, which yields:
    \begin{equation}
        \boxed{%
        B(T)=\frac{2\pi}{3}\frac{n}{n-3}\left(\frac{\alpha}{kT}\right)^{3/n}}
        \label{eq:B}
    \end{equation}
    And for the derivative of \eqref{eq:B}, one finds:
    \begin{equation*}
        \pdv{B}{T}=-\frac{3B}{nT}
    \end{equation*}


\item
    Computing the heat capacity for constant volume (with
    $S=-\thermpart{F}{T}{V}$, $F=-kT\log{Z}$):
    \begin{equation*}
        c_{V}=T\thermpart{S}{T}{V}=kT\pdv[2]{(T\log{Z})}{T}
        =kT\left(2\pdv{\log{Z}}{T}+T\pdv[2]{\log{T}}{T}\right)
    \end{equation*}
    One finds for \eqref{eq:Z}:
    \begin{align*}
        \pdv{\log{Z}}{T}&=\frac{3}{nT}\frac{N^2}{V-N^2B} \\
        \pdv[2]{\log{Z}}{T}&=-\frac{3}{nT^2}\frac{N^2B}{V-N^2B}\left(1+\frac{3}{n}\frac{N^2B}{V-N^2B}\right)
    \end{align*}
    Which gives:
    \begin{equation}
        \boxed{%
        c_{V}=\frac{3k}{n}\frac{N^2B}{V-N^2B}\left(1+\frac{3}{n}\frac{N^2B}{V-N^2B}\right)}
        \label{eq:cv}
    \end{equation} \\

    The heat capacity for constant pressure can be obtained from the relation:
    \begin{equation*}
        c_{P}-c_{V}=-T\frac{\thermpart{P}{T}{V}^2}{\thermpart{P}{V}{T}}
    \end{equation*}
    Working out the derivatives of \eqref{eq:EOS}:
    \begin{align*}
        \thermpart{P}{T}{V}&=\frac{Nk}{V}\left(1+\frac{n-3}{n}\frac{NB}{V}\right)
        \\
        \thermpart{P}{V}{T}&=-\frac{NkT}{V^3}\left(V+2NB\right)
    \end{align*}
    Leading to:
    \begin{equation}
        c_{P}=\frac{3k}{n}\frac{N^2B}{V-N^2B}\left(1+\frac{3}{n}\frac{N^2B}{V-N^2B}\right)+N\frac{\left(V+\frac{n-3}{n}NB\right)^2}{V^2+2NVB}
        \label{eq:cp}
    \end{equation}


\item
    In order to compute the isothermic compressibility
    \begin{equation*}
        \beta_{T}=-\frac{1}{V}\thermpart{V}{P}{T}
    \end{equation*}
    one can (ab-)use the fact that in thermodynamics all variables are
    considered as functions of each other, which allows to use the chainrule in
    this way:
    \begin{equation*}
        \pdv{V}{V}=\pdv{V}{P}\pdv{P}{V}=1\implies\pdv{V}{P}=\frac{1}{\pdv{P}{V}}
    \end{equation*}

    Therefor one can use \eqref{eq:EOS} directly and with
    $\thermpart{P}{V}{T}=-\frac{NkT}{V^3}(V+2NB)$ write:
    \begin{equation}
        \beta_{T}=-\frac{1}{V\thermpart{P}{V}{T}}=\frac{V^2}{NkT(V+2NB)}
        \label{eq:bt}
    \end{equation} \\

    For the adiabatic compressibility, one can utilize the relation:
    \begin{equation*}
        \gamma=\frac{\beta_{T}}{\beta_{S}}=\frac{c_{P}}{c_{V}}\iff\beta_{S}=\frac{\beta_{T}\,c_{V}}{c_{P}}
    \end{equation*}
    And after inserting \eqref{eq:cv}, \eqref{eq:cp} and \eqref{eq:bt}, one
    receives:
    \begin{equation}
        \boxed{%
        \beta_{S}=\frac{\frac{3k}{n}\frac{N^2B}{V-N^2B}\left(1+\frac{3}{n}\frac{N^2B}{V-N^2B}\right)\frac{V^2}{NkT(V+2NB)}}
        {\frac{3k}{n}\frac{N^2B}{V-N^2B}\left(1+\frac{3}{n}\frac{N^2B}{V-N^2B}\right)+Nk\frac{\left(V+\frac{n-3}{n}NB\right)^2}{V^2+2NVB}}
        }
        \label{eq:bs}
    \end{equation}

    Now one could try to further work out \eqref{eq:bs} in order to find a more
    concise expression, if one had nothing else to do.

\end{enumerate}

\end{document}
